%                 Plain TeX file
%                 ==============
%
%-------------------------------------------------------------------------
%
% additional macros required

\font\fourtbf=cmbx10 scaled \magstep2
\def\ie{{\it i.e.}}
\def\cf{{\it cf.}}
\def\etal{{\it et al.}}
\def\CP{{\bf CP}}
\def\nwl{\hfill\break}
\def\ref{\par\noindent
	\hangindent=0.7 true cm
	\hangafter=1}
\def\note #1]{{\bf #1]}}
%  section macros, designed to avoid new page after section headings.
\def\mainsect #1\par{\interlinepenalty=100\bigskip\bigskip\noindent
\hangindent=1 true cm
\hangafter=1
{#1}\par%
\penalty10000\par\bigskip\par\penalty10000\noindent}
%\def\mainsect #1\par{\bigskip\bigskip\noindent{#1}\par%
%\penalty10000\par\bigskip\par\penalty10000\noindent}

\def\subsect #1\par{\bigskip\noindent{#1}\par%
\penalty10000\bigskip\par\penalty10000\noindent}
\def\intsect #1\par{\par\penalty10000\bigskip\penalty10000\noindent{#1}\par%
\penalty10000\par\bigskip\par\penalty10000\noindent}


% \input /dsk2/users/jcd/tex/loc-macro.tex
\magnification=\magstephalf
\raggedbottom

\def\solar{\odot}
\def\Plus{{\displaystyle +}}
\def\H{{\rm H}}
\def\He{{\rm He}}
\def\dd{{\rm d}}

\centerline{\fourtbf File format for}
\centerline{\fourtbf GONG solar model project.}

\bigskip

\centerline{J{\o}rgen Christensen-Dalsgaard}
\centerline{Teoretisk Astrofysik Center; and Institut for Fysik og Astronomi,
 Aarhus Universitet}

\bigskip
%
\mainsect
{\bf 1. Introduction.}

The present notes provide a definition of the data format
used for comparison of solar models within the GONG models team.
They essentially correspond to Section 9 of
{\it Computational procedures for GONG model project}
(in the following {\CP}) and have been extracted here for
convenience of reference.
For details on the GONG model comparisons project, involving simplified
physics, {\CP} should be consulted.
The focus here is on the exchange of realistic solar models,
involving detailed physics.

\mainsect
{\bf 2. File structure, and variables}

The exchange of models will take place by means of ASCII files,
to avoid problems with different binary formats.
This makes it possible to send the data electronically without
problems (other than caused by the amount).

For each model the file consists of a header, with descriptive
information, a set of global variables, and a set of variables
given at each mesh point.

The set of variables is likely to develop, particularly in the
context of the use of the format for transferring realistic
solar and stellar models.
The actual set used is defined by the variable {\tt ivers}
included in the header to the file.
In addition, the total number of global parameters, meshpoints
and variables at each meshpoint are given in the header,
so that the information required to read the file is available.

The variables should be given in {\it cgs} units, unless otherwise stated.

\subsect
{\it 2.1 Text header.}

The first record should contain the name of the model,
its date, and an identification of its origin (such as name or
institute). 
In addition the header may contain text further describing
the calculation, and information about the remaining data.

An example of a header may be the following
(describing Model S of the {\it Science} series of articles):
\medskip
{\obeylines \tt
L5BI.D.15.PRES.950912.AARHUS
Level 5 physics, present Sun. (OPAL opacity, OPAL EOS). He, Z diffusion.
Age of present Sun: 4.6 Gyr.

}

\subsect
{\it 2.2 Global parameters.}

These are set up in the array {\tt glob(i)}, {\tt i = 1, ..., iconst},
with the following definition:

\medskip
{\obeylines
% \ind{5.0em}
 1: $M$  	(total mass).
 2: $R$  	(photospheric radius).
 3: $L_{\rm s}$	(surface luminosity).
 4: $Z$ 	(heavy element abundance).
 5: $X_0$ 	(initial hydrogen abundance).
 6: $\alpha = \ell /H_p$ (mixing-length parameter; %
 $H_p$ is pressure scale height)
 7: $\phi$  	(another convection theory parameter)$^{\rm a)}$
 8: $\xi$  	(yet another convection theory parameter)$^{\rm a)}$
 9: $\beta$	(parameter in surface pressure condition)$^{\rm b)}$
10: $\lambda$	(parameter in surface luminosity condition)$^{\rm b)}$
11: $\displaystyle{{R^2 \over p_{\rm c}} {\dd^2 p_{\rm c}  \over \dd r^2}}$ %
(at centre)$^{\rm c)}$
12: $\displaystyle{{R^2 \over \rho_{\rm c}} %
{\dd^2 \rho_{\rm c} \over \dd r^2}}$ (at centre)$^{\rm c)}$
13: Model age (in years)$^{\rm d)}$
14 -- 15: Unused, so far.
}
\medskip\noindent
Notes:
\medskip
\item{a)}
The parameters $\phi$ and $\xi$ are defined in {\CP}.
For the B\"ohm-Vitense formulation, $\phi = 9/4$ and $\xi = 1/162$.
\item{b)} 
These parameters are used in boundary conditions for
models with simplified physics ({\cf} {\CP}).
For realistic models, $\beta = \lambda = 1$.
\item{c)}
These second derivatives may be needed
in the central boundary conditions for oscillation calculations.
\item{d)}
Added 30/7/96.
If the calculation includes pre-main-sequence evolution, the definition
of zero age must be specified, in the character header or in
accompanying notes.
It would be useful to arrive at a common definition;
suggestions are welcome.

\subsect
{\it 2.3 Model variables at each mesh point.}

These are set up in the array {\tt var(i, n)}, 
{\tt i = 1, ..., ivar}, {\tt n = 1, ..., nn}.
The current set of variables, as defined in July 1996,
is characterized by having {\tt ivers} = 250.
\medskip
{\obeylines\parskip=2pt
 1: $r$ (distance to centre)
 2: $\ln q , \quad  q = m/M$ ($m$ is mass interior to $r$ and $M$ is total mass)
 3: $T$ (temperature)
 4: $p$ (pressure)
 5: $\rho$ (density)
 6: $X$ (hydrogen abundance by mass)
 7: $L(r)$ (luminosity at distance $r$ from centre)
 8: $\kappa$ (opacity)
 9: $\epsilon$ (energy generation rate per unit mass)
10: $\displaystyle {\Gamma_1 = %
\left({\partial \ln p \over \partial \ln \rho} \right)_{\rm ad}}$
11: $\displaystyle{\nabla_{\rm ad} = %
\left({ \partial \ln T \over \partial \ln p} \right)_{\rm ad}}$
12: $\displaystyle{\delta = %
- \left( {\partial \log \rho  \over \partial \log T} \right)_p}$
13: $c_p$ (specific heat at constant pressure)
14: $\mu_{\rm e}^{-1}$ [see note ii) below]
15: $ \displaystyle{{1 \over \Gamma_1} {\dd  \log p  \over \dd \log r} - %
{\dd  \log \rho  \over \dd \log r}}$
16: $r_X$ (rate of change in $X$ from nuclear reactions)
17: $Z$ (heavy-element abundance per unit mass)
18: $R - r$
19: $\epsilon_{\rm g}$ (rate of gravitational energy release)
20: $L_{\rm g}$ (local gravitational luminosity; this has only %
been included in J. Reiter's models so far)
21: $X({}^3{\rm He})$ (${}^3{\rm He}$ abundance by mass)
22: $X({}^{12}{\rm C})$ (${}^{12}{\rm C}$ abundance by mass)
23: $X({}^{13}{\rm C})$ (${}^{13}{\rm C}$ abundance by mass)
24: $X({}^{14}{\rm N})$ (${}^{14}{\rm N}$ abundance by mass)
25: $X({}^{16}{\rm O})$ (${}^{16}{\rm O}$ abundance by mass)
26: $\displaystyle \left({\partial \ln \Gamma_1 \over \partial \ln \rho} %
     \right)_{p, Y}$
27: $\displaystyle \left({\partial \ln \Gamma_1 \over \partial \ln p} %
     \right)_{\rho, Y}$
28: $\displaystyle \left({\partial \ln \Gamma_1 \over \partial Y} %
     \right)_{p, \rho}$
29 -- 30: Currently not used.
}
\medskip\noindent
Comments:
\medskip
\item{i)}
In variable 2, $\ln q$ is used instead of $m$ to give a better
indication of variation close to the surface.
Here ``ln'' is natural logarithm.
\item{ii)}
$\mu_{\rm e}^{-1} = N_{\rm e} m_{\rm u}$,
where $N_{\rm e}$ is the number of free electrons per unit mass
and $m_{\rm u}$ is the atomic mass unit;
thus $\mu_{\rm e}$ is the mean molecular weight per electron.
It has been included to give some indication of the ionization state.
\item{iii)} 
Variable 18 was introduced Aug. 6 1993, to avoid problems with interpolation
in $r$ near surface. (For versions before {\tt ivers} = 210,
but after that date, $R - r$ was set in variable 17.)
\item{iv)}
The derivatives of $\Gamma_1$ were introduced principally to allow
calculation of kernels involving $Y$, in realistic models;
they are less essential for the comparison of models.
\item{v)}
Calculations that do not follow all details of the CNO cycle
may not have available some of the CNO abundances.
In that case, the corresponding abundances can be set to zero.
(An example might be a calculation following the conversion of
${}^{16}{\rm O}$ into ${}^{14}{\rm N}$ but ignoring the
details of carbon burning into ${}^{14}{\rm N}$.)
\item{vi)}
More generally, for a given code there might be variables that are
difficult to obtain. 
In that case, the above numbering should be maintained, but the
missing variables may be set identially to zero.

\medskip\noindent
The present set of variables has been chosen to give a reasonably
comprehensive basis for comparing evolution models, and to be
adequate for the computation of adiabatic oscillations.
In particular these variables should define completely
the stellar structure equations, and so permit a check of the
accuracy to which the equations are satisfied.
For a more detailed comparison even more variables may be needed
(such as ionization levels), but I suggest that that be arranged
separately. 

For non-adiabatic calculations, which we should eventually
get to, more variables are certainly needed.
Typical examples are the derivatives of $\kappa$ and $\epsilon$
with respect to $p$ and $T$, and possibly variables relating
to the perturbation of the convective flux.
They can be included later, by extending the basic set
given above.

\subsect
{\it 2.4 Format for data transfer.}

The data exchange should be carried out by means of formatted
ASCII files, using the following structure:
\medskip
{\obeylines
{\bf Record 1:} %
Name of model (as a character string)
{\bf Record 2 -- 4:} %
Explanatory text, in free format
{\bf Record 5:} %
{\tt nn, iconst, ivar, ivers}
{\bf Record 6 -- 8:} %
{\tt glob(i), i = 1, ..., iconst}
{\bf Record 9 --  :} %
{\tt var(i,n), i = 1, $\ldots$, ivar, n = 1, $\ldots$ nn}.
}
\medskip\noindent
Here {\tt nn} is the number of mesh points in the model,
{\tt iconst} is the number of global variables (given in the array
{\tt glob}), and {\tt ivar} is the number of variables at each mesh point
(given in the array {\tt var}).
Thus with the present set {\tt iconst} = 15 and {\tt ivar} = 30, but there is
room for expansion.
The version number {\tt ivers} was discussed above.

The integer variable Record 5 should be written with the format 
\medskip\noindent
\qquad {\tt 4i10}
\medskip\noindent
and the real variable Records 6 and up with the format
\medskip\noindent
\qquad {\tt 1p5e16.9}

\mainsect
{\bf 3. Summary of earlier versions}

The format has been in used for almost a decade, with various extensions
or modifications;
these are characterized by the version number, as summarized below:

\medskip
\item{--} {\bf Version 100}:
{\tt var(1) - var(16)} defined as above.
In some models {\tt var(17)} was set to $R - r$, 
and {\tt var(19)} and {\tt var(20)} were set as above;
otherwise these variables were set to zero;
{\tt ivar} = 20.
\item{--} {\bf Version 200}:
{\tt var(1) - var(16)} and {\tt var(19) - var(25)} defined as above,
{\tt var(17)} set to $R - r$; 
{\tt ivar} = 25.
\item{--} {\bf Version 210}:
{\tt var(1) - var(25)} defined as above;
{\tt ivar} = 25.

\end
