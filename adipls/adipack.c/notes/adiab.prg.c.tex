%\input /home/hao/jcd/tex/loc-macro.tex
\magnification=\magstephalf
\tolerance=2000
\def\nwl{\hfill\break}
\def\note #1]{{\bf #1]}}
\def\etal{{\it et al.}}
\def\eg{{\it e.g.}}
\def\cf{{\it cf.}}
\def\ie{{\it i.e.}}
\def\etc{{\it etc.}}
\def\viz{{\it viz.}}
\def\newpage{\vfill\eject}
\def\dd{{\rm d}}
\def\DD{{\rm D}}
\def\f{{\rm f}}
\def\Msun{\,{\rm M}_\odot}
\def\Rsun{\,{\rm R}_\odot}
\def\pturb{p_{\rm turb}}
\def\pg{p_{\rm g}}
\def\Gammar{\Gamma_1^{\rm r}}
\def\draft{\headline{\bf File: \jobname\hfill DRAFT\hfill\today}}
%\def\source#1\par{\medskip{\obeylines\smallskip\tt\parindent=40pt #1}
%\medskip\noindent}
\def\ms{\medskip}
\def\msni{\medskip\noindent}
\def\source{\obeylines\tt\smallskip\parindent=40pt}
\def\param{\par\noindent\hangindent\parindent\hangafter=1}
\def\pparam{\par\indent\hangindent2\parindent\hangafter=1}
\def\ppparam{\par\indent\indent\hangindent3\parindent\hangafter=1}
\def\defindent{\par\indent\indent\parindent=8.5em}
\def\defaults{\defindent \obeylines (default: \tt}
\def\muHz{\,\mu{\rm Hz}}
\def\fig #1
{\midinsert
\leftskip=3em
\rightskip=3em
\vskip #1
\noindent}
\def\gif{\endinsert}


\def\today{\ifcase\month\or
  January\or February\or March\or April\or May\or June\or
  July\or August\or September\or October\or November\or December\fi
  \space\number\day, \number\year}
\font\twenbf=cmbx10 scaled 2000
\font\fourtbf=cmbx10 scaled \magstep2
\font\twelvebf=cmbx10 scaled \magstep1

%  section macros, designed to avoid new page after section headings.
\def\mainsect #1\par{\interlinepenalty=100\bigskip\bigskip\noindent
\hangindent=1 true cm
\hangafter=1
{#1}\par%
\penalty10000\par\bigskip\par\penalty10000\noindent}
%\def\mainsect #1\par{\bigskip\bigskip\noindent{#1}\par%
%\penalty10000\par\bigskip\par\penalty10000\noindent}

\def\subsect #1\par{\bigskip\noindent{#1}\par%
\penalty10000\bigskip\par\penalty10000\noindent}
\def\intsect #1\par{\par\penalty10000\bigskip\penalty10000\noindent{#1}\par%
\penalty10000\par\bigskip\par\penalty10000\noindent}
\def\ref{\par\noindent
\hangindent=0.7 true cm
\hangafter=1}
\def\det{{\rm det \,}}
\def\sign{{\rm sign \,}}
\def\xf{x_{\rm f}}
\def\xs{x_{\rm s}}
%\def\revision#1{{\bf --- REVISION #1 ---:}}
\def\revision#1{\null}

\def\contentfill{\leaders\hbox to 2em{\hss.\hss}\hfill}
\def\contentitem#1#2{{\interlinepenalty=100
\noindent
\hangindent=1 true cm
\hangafter=1
{#1{\rm \hskip 2em \contentfill \hskip 2em #2}\par}}}

\def\chaptercont #1 ... #2
{\bigskip\contentitem{{#1}}{{#2}}
\medskip}
\def\mainsectcont #1 ... #2
{\medskip\contentitem{#1}{#2}
}
\def\subsectcont #1 ... #2
{\leftskip=15truept
\smallskip\contentitem {#1} {#2}
\vskip 0pt\leftskip=0pt}

\pageno=-1
\nopagenumbers

\centerline {\twenbf Notes on adiabatic oscillation programme}
\medskip
\centerline{\fourtbf Sixth edition}
\medskip
\centerline{\today}
\bigskip

\centerline{ J{\o}rgen Christensen-Dalsgaard}
\centerline{ Institut for Fysik og Astronomi, Aarhus Universitet}
\bigskip
\vfill
\hrule
\medskip

\noindent
{\bf Notes on revisions:} These notes were originally written at HAO in 1983. 
The fourth edition was produced in September 1997,
following a major reorganization of the code involving also
the addition or change of several input parameters.
This was further updated in a fifth edition in 2002,
including the extension of the code to include rotational effects. 
This still needs further testing, however.

The present edition is aimed to correspond to the version of the code
{\tt adipls.c.d.x}, as of May 2010.
This is the version that is distributed in the updated distribution of
the code ({\tt adipack.c}) and used in the general package combining the
evolution and pulsation codes, which has a rather more restricted distribution.
Relative to the previous distribution of the oscillation code from
1997 ({\tt adipack.n}), several additions and updates have been made,
including:
\medskip
\item{--} The use of fourth-order integration in the shooting method,
flagged by {\tt mdintg = 5}.

\item{--} The use of the Takata (2006b) scheme for labelling dipolar modes,
flagged by {\tt irsord = 20}.

\item{--} The possibility of computing first-order rotational effects on 
frequencies and eigenfunctions, following Soufi et al. (1998). This involves a 
new block of input parameters, {\tt rot}, and is flagged by {\tt irotsl = 1}.
{\bf This option still needs testing.}

\item{--} The option of producing (generally discontinuous) solutions
at arbitrary frequency, flagged by {\tt itmax = 0} and {\tt nfmode $>$ 10}.
This also includes the option for output of solutions during scan in
frequency, with {\tt iscan $>$ 1}, if {\tt nfmode $>$ 0}.

%In general, it is recommended to obtain and install the most
%recent version of the code.
%However, significant changes in the code and the notes
%have been flagged in the text, indicating the time where
%the modification was carried out.
%\note [Probably no longer?]

It is recommended to consider also the notes given
in the source to subroutine {\tt adipls}, particularly as far as
input parameters to the programme are concerned, since the
notes in the programme source are generally updated when the
programme is changed.

\medskip
\hrule
\newpage
\null
\footline={\hss\tenrm\folio\hss}
\newpage
\centerline{\twelvebf Contents}
\bigskip
\chaptercont
{\twelvebf 1 Introduction} ... 1
\chaptercont
{\twelvebf 2 Notation. Equations and boundary conditions} ... 2
\mainsectcont
{\bf 2.1 Variables and equations} ... 2
\mainsectcont
{\bf 2.2 Removing the $x^l$ variation near the centre} ... 3
\mainsectcont
{\bf 2.3 Inner boundary conditions} ... 4
\subsectcont
{\it 2.3.1 Full model} ... 4
\subsectcont
{\it 2.3.2 Truncated model} ... 4
\mainsectcont
{\bf 2.4 Surface conditions} ... 5
\subsectcont
{\it 2.4.1 Singular surface} ... 5
\subsectcont
{\it 2.4.2 Simple surface conditions} ... 5
\subsectcont
{\it 2.4.3 Match to the solution for an isothermal atmosphere} ... 6
\chaptercont
{\twelvebf 3 Numerical techniques}  ... 7
\mainsectcont
{\bf 3.1 The shooting method}  ... 7
\mainsectcont
{\bf 3.2 The relaxation technique} ... 9
\mainsectcont
{\bf 3.3 The transition from $\hat y_i$ to $y_i$}  ... 10
\mainsectcont
{\bf 3.4 Richardson extrapolation}  ... 11
\mainsectcont
{\bf 3.5 Operational experience} ... 12
\chaptercont
{\twelvebf 4 Results of the calculation}  ... 13
\mainsectcont
{\bf 4.1 Frequency correction in the Cowling approximation} ... 13
\mainsectcont
{\bf 4.2 Mode order} ... 13
\mainsectcont
{\bf 4.3 Mode energy, scaled eigenfunctions} ... 14
\mainsectcont
{\bf 4.4 Variational frequency} ... 15
\mainsectcont
{\bf 4.5 Kernels} ... 15
\chaptercont
{\twelvebf 5 Equilibrium model variables} ... 16
\chaptercont
{\twelvebf 6 Notation and data storage in the programme} ... 17
\mainsectcont
{\bf 6.1 Model storage} ... 17
\mainsectcont
{\bf 6.2 Storage of solution} ... 18
\mainsectcont
{\bf 6.3 Terminal input, and terminal and printed output}  ... 18
\chaptercont
{\twelvebf 7 Input to the programme}  ... 18
\mainsectcont
{\bf 7.1 The equilibrium model}  ... 18
\mainsectcont
{\bf 7.2 Control parameters}  ... 19
\subsectcont
{\it 7.2.1 Input of control parameters}  ... 19
\subsectcont
{\it 7.2.2 Assignment of unit numbers to files} ... 20
\subsectcont
{\it 7.2.3 Description of control parameters} ... 21
\mainsectcont
{\bf 7.3 Comments on the input parameters} ... 28
\subsectcont
{\it 7.3.1 Notes on model} ... 28
\subsectcont
{\it 7.3.2 Notes on $l$ and trial eigenfrequency} ... 28
\subsectcont
{\it 7.3.3 Remaining parameters}  ... 30
\mainsectcont
{\bf 7.4 The user-supplied routines {\tt modmod} and {\tt spcout}} ... 31
\chaptercont
{\twelvebf 8 Output from the programme}  ... 32
\mainsectcont
{\bf 8.1 Printed output} ... 32
\mainsectcont
{\bf 8.2 The grand summary} ... 33
\mainsectcont
{\bf 8.3 The short summary} ... 35
\mainsectcont
{\bf 8.4 Output of eigenfunction to file} ... 35
\mainsectcont
{\bf 8.5 Output of rotational kernel to file ({\cf} equation 4.7)} ... 36
\mainsectcont
{\bf 8.6 Output of $\Gamma_1$ kernel to file ({\cf} equation 4.8)} ... 36
\mainsectcont
{\bf 8.7 Iteration status log} ... 37
\chaptercont
{\twelvebf 9 The main programme}  ... 37
\chaptercont
{\twelvebf References} ... 38
\chaptercont
{\twelvebf Appendix A. Equations} ... 40
\mainsectcont
{\bf A.1 Equations in the standard case} ... 40
\mainsectcont
{\bf A.2 Equilibrium variables and oscillation equations in 
plane-parallel case} ... 40
\mainsectcont
{\bf A.3 The treatment of turbulent pressure} ... 41
\subsectcont
{\it A.3.1 Neglect of the Eulerian perturbation in the turbulent body force
({\tt iturpr = 1)}} ... 42
\subsectcont
{\it A.3.2 Neglect of the Eulerian perturbation in the turbulent pressure
({\tt iturpr = 2)}} ... 44
\subsectcont
{\it A.3.3 Neglect of the Lagrangian perturbation in the turbulent pressure}
... 45
\mainsectcont
{\bf A.4 Inclusion of rotational effects} ... 46
\subsectcont
{\it A.4.1 Equations with no explicit inclusion of rotation} ... 47
\subsectcont
{\it A.4.2 Including first-order rotation, according to Soufi et al.} (1998)
... 48
\newpage

\pageno=1
\mainsect
\centerline{\twelvebf 1 Introduction} 

These notes describes a programme for calculation of adiabatic 
oscillations of stellar models. This has been developed over
several years, with a number of different uses in mind, and
is therefore fairly complex both in structure and in the
control parameters possible. In particular it has not been
designed for general use, and so it has not been thoroughly
tested in all conceivable cases. On the other hand the
programme is quite flexible, both in the physical situations
it may consider, and in the numerical schemes it employs.
Thus both models with vanishing surface pressure (and hence
a singularity at the surface), and physical models terminating
at a finite pressure can be considered; and it is possible to consider
models truncated at a finite distance from the centre, as well
as complete models.
A description of the code was given by Christensen-Dalsgaard (2008).

A principal goal of the present notes has been to give a precise
definition of the calculations carried out by the programme.
The notes also provide information intended to be sufficiently
detailed to permit easy use of the programme, describing
the notation used, the form needed for the equilibrium
model that must be provided to the programme, and the output
produced, as well as some notes on the experience gained so far
in running the programme. 
%No discussion is given of the organization
%of the programme; indeed 
%\note [Some remarks about notes on organization in Appendix].
However, it is sufficiently complex 
that modifications by users other than the author should 
be attempted only with care.

The programme uses equilibrium models in the form
of so-called {\tt amdl} files, produced by the JC-D evolution code.
They consist of an essentially minimal set of variables
for computing adiabatic oscillations.
The variables are described in Section 5, and the input format
is provided in Section 7.1.
The programme is controlled by a vast number of control parameters,
described in Section 7.2. 
In most cases, particularly for computing p-mode frequencies
of solar models, these parameters barely need to be changed.
Samples of suitable input files can be provided.

The output from the programme is described in Section 8.
The most important output is the set of frequencies and other
mode data, and the mode eigenfunctions.
The most extensive set of mode data is provided in
binary form in the {\it grand summary file} (Section 8.2),
which also gives diagnostics on the computation.
A more compact set of mode data, also in binary form,
is given in the {\it short summary} (Section 8.3).
The eigenfunctions can be provided as a comprehensive set,
a set containing just the displacement amplitudes,
or a set giving the density-weighted displacement amplitudes
(Section 8.4); the latter set is the most appropriate for
setting up kernels.
The eigenfunction data contain the grand summary.
Note that, as the eigenfunction files are rather large,
it is advisable to output them only if explicitly needed.
The programme may also output kernels for frequency
spitting caused by spherically symmetric rotation (Section 8.5)
and, less usefully, kernels for the effect on frequencies of
changes in the adiabatic exponent $\Gamma_1$ at fixed density $\rho$
(Section 8.6).

In case of problems, diagnostic output is provided in a 
log file with the default name {\tt adipls-status.log}
(see Section 8.8; the name may be changed in the input file).
The programme also produces output summarizing the calculation,
and providing further diagnostics.
This is typically less useful, except for debugging;
in particular, reports of problems should be accompanied
by this output file (see Section 8.1).

In Section 2 the equations and boundary conditions are briefly 
discussed; this section also serves to define the notation used.
A more detailed description of the equations is contained in
Appendix A. Section 3 describes the numerical techniques used to
solve the equations. Section 4 defines the quantities computed by
the programme. 
%Section 5 describes the equilibrium variables that
%must be supplied to the programme. 
Section 6 provides a brief 
discussion of the notation and data storage used in the programme,
mainly serving to define the terms used in Section 7.
% which describes the input required.
% Section 8 discusses the output from the programme.
Finally Section 9 describes the main programme that must
be supplied by the user to set up storage for the calculation. 

Very few references are given here. This evidently does not imply that
the methods discussed are original. Detailed discussions of non-radial
stellar oscillations, as well as numerous references, may be found,
{\eg}, in Unno {\etal} (1989) and Aerts {\etal} (2010).

I finally note that a separate short set of notes are available
on the use of this and other related programmes:
{\it Notes on using the solar models and adiabatic pulsations package}.

\mainsect
\centerline{\twelvebf 2 Notation. Equations and boundary conditions} 

The notation used here in general
follows that of Christensen-Dalsgaard (1981, 1982, 2008; in the following
CD81, CD82, CD08);
however, it is described in part here.
Words entirely in {\tt typewriter type} refer
to names of variables in the programme.

\subsect
{\bf 2.1 Variables and equations} 

As usual the perturbations are assumed to be separated in $\theta$ and
$\phi$ in terms of spherical harmonics, and the time dependence
to be given as a harmonic function. Thus the displacement vector
is written as
$$
\vec{\delta r} = {\rm Re} \left\{ \left[ \xi_r (r) Y_l^m ( \theta ,
\phi ) {{\bf a} }_r + \xi_h (r) \left( { \partial Y_l^m 
\over \partial \theta } {{\bf a} }_{\theta} + {1 \over \sin  \theta} 
{\partial Y_l^m  \over \partial \phi} {{\bf a} }_{\phi} \right)
\right] \exp (-i \omega t ) \right\} \; .
\eqno (2.1)
$$
Similarly the perturbation in, e.g. pressure, may be written
$$
\delta p  =  {\rm Re} \left[ \delta p(r) Y_l^m ( \theta , \phi ) \exp (- i 
\omega t  ) \right] \; .
\eqno (2.2)
$$
Here $Y_l^m$ is a spherical harmonic, ${{\bf a}}_r$, ${{\bf a}}_{\theta}$, and ${{\bf a}}_{\phi}$ are unit vectors in the
$r$, $\theta$, and $\phi$ directions and $\omega$ is the frequency.
As the oscillations are adiabatic (and only conservative boundary
conditions are considered) $\omega$ is real, and the amplitude functions
$\xi_r (r)$, $\xi_h (r)$, $\delta p (r)$, etc. can be chosen to be real.

The equations for non-radial oscillations can be found in a number
of references ({\eg} Unno {\etal} 1989).
However, the programme permits a modification of 
Poisson's equation for the perturbations, which is written as
$$
{1 \over r^2}   {\dd \over \dd r }\left( r^2 {\dd \Phi^{\prime}
 \over \dd r }\right)  -  {l ( l + 1) 
\over r^2} \Phi^{\prime} = - 4 \pi \lambda G \rho^{\prime}
\eqno (2.3)
$$
(the ordinary equation clearly corresponds to having $\lambda = 1$).
This was done to investigate the effect of making a gradual transition
between the Cowling approximation, where $\Phi^{\prime}$ is neglected,
to the full set of equations (Christensen-Dalsgaard \& Gough, in preparation);
this can clearly be accomplished by varying $\lambda$ continuously from 0
to 1.

Nonradial oscillations may be considered both in the Cowling approximation
(corresponding to a second-order system of equations) and in the full
case (corresponding to a fourth-order set). For radial oscillations
the perturbation in the gravitational potential can be eliminated 
analytically, so that here the basic, complete set of equations is
of second order. 

A further restriction is that
nonradial oscillations of a truncated model can only be treated in the
Cowling approximation; this is almost inevitable, as there seems to
be no natural way to specify the boundary conditions for the perturbation
in the gravitational potential at the base of a truncated model.

In some equilibrium models a contribution to the equation of hydrostatic
support from ``turbulent pressure'' may be included. This is the case, {\eg},
for semi-empirical models of the solar atmosphere
or models based on averages of hydrodynamical simulations.
This raises the
question of the appropriate treatment of the turbulent pressure in
the perturbation equations. 
The programme allows various options for treating
(or, more precisely, neglecting) effects of turbulent pressure.
These are controlled by the definition of the equilibrium-model
variables and/or the parameter {\tt iturpr}.
The details are discussed in Section A.3.
It should be noted that effects of turbulent pressure
may well dominate the differences between the observed solar
frequencies and frequencies of solar models where turbulent
pressure is neglected ({\eg} Rosenthal {\etal} 1999).

The degree $l$ is treated as a real variable (clearly only integer
values have physical meaning). The transition between the nonradial
and the radial equations is defined in the programme
to take place at $l  =  10^{-6}$.

In the calculation the frequency is expressed in terms of the dimensionless
squared frequency 
$$
\sigma^2  =  {R^3 \over G M }  \omega^2 \; ,
\eqno (2.4)
$$
where $M$ and $R$ are mass and photospheric radius of the equilibrium model.
The eigenfunctions are defined in terms of a set of dimensionless functions
$y_i (x)$ where $x  =  r / R$. For nonradial oscillations ($l > 0$)
$$
\eqalign{
y_1 & = {\xi_r  \over R } \; , \cr
y_2 & = x  \left( {p^{\prime}  \over \rho } -  \Phi^{\prime} \right)
{l ( l + 1)  \over \omega^2 r^2} = {l ( l + 1)  \over R }  \xi_h \; , \cr
y_3 & =  x  {\Phi^{\prime}  \over gr } \; , \cr
y_4  & = x^2  {\dd \over \dd x }\left( {y_3   \over x }\right) \; . \cr
}
\eqno (2.5a)
$$

For radial oscillations only $y_1$ and $y_2$ are used, where
$y_1$ is defined as above, and 
$$
y_2 = {p^{\prime}  \over \omega^2 R^2 \rho} \; .
\eqno (2.5b)
$$

\subsect
{\bf 2.2 Removing the $x^l$ variation near the centre} 

The dependent variables $y_i $ have been chosen in such a way that
for $l > 0$ they all vary as $x^{l - 1}$ for $x \rightarrow 0$.
For large $l$
a considerable (and fundamentally unnecessary) computational effort would
be needed to represent this variation sufficiently accurately with, {\eg},
a finite difference technique, if these variables were to be used in
the numerical integration. Instead we introduce a new set of dependent
variables by
$$
{\hat y}_i = x^{- l +1}  y_i , \qquad i  = 1, 2, 3, 4 \; .
\eqno (2.6)
$$
These variables are then $O(1)$ in $x$ near the centre.
They are used in the region where the variation in the
$y_i$ is dominated by the $x^{l - 1}$ behaviour.
Specifically one obtains from JWKB theory that
$$
y_1 (x)  \sim \exp \left( \int^x k \dd x \right) \; ,
\eqno (2.7)
$$
where
$$
k^2 = {l ( l +1) \over x^2} \left( 1 - {{\tilde N}^2 
\over \sigma^2} \right) \left( 1 - {\sigma^2  \over {\tilde S }_l^2} \right) 
\; ,
\eqno (2.8)
$$
and $\tilde N$ and $\tilde S_l$ are the dimensionless buoyancy and
Lamb frequencies (see Section 5 below).
Here ${\tilde N}^2$ and ${\tilde S}_l^{-2} \to  0$ as $x \to 0$.
Thus close to the centre 
$k^2 \simeq l ( l + 1)/x^2$, and, to this order,
the JWKB result is that $y_1  \sim x^{l +1/2}$. Writing
$k^2 = h(x) l ( l + 1)/x^2$ this behaviour is clearly
dominant as long as $h(x) > 1/4$.
Thus the transition from ${\hat y}_i$ to $y_i$ at a value $x_{\rm ev}$
of $x$ where $h(x) = 1/4$, such that the exponentially
growing solution dominates for $x > x_{\rm ev}$.
The default is to choose the smallest value of $x$ where $h(x) = 1/4$;
however, there is the option of restricting the search to
be outside a given value of $x$.

This transformation is applied when non-radial oscillations 
are computed. It permits calculating modes of arbitrary high degree
in a complete model; when the ${\hat y}_i$ are multiplied by
$x^{l - 1}$ to obtain the $y_i$ 
this is done in a special routine that replaces 
$x^{l - 1}$ by zero when it is below the (machine-dependent) limit
for underflow.

\subsect
{\bf 2.3 Inner boundary conditions} 

The inner conditions depend on whether one considers a full model,
including the centre, or an envelope model.
In the former case, the boundary conditions are
regularity conditions ({\eg} Christensen-Dalsgaard, Dilke \& Gough 1974).
In the latter case, there is considerably more flexibility in
the choice of condition.

\subsect
{\it 2.3.1 Full model}

When the centre is included in the equilibrium model, the solution must
satisfy regularity conditions at the innermost meshpoint. These are
obtained by expansion around the centre, to terms that are $O ( x^2 )$
times the leading terms.
It is useful to note that it follows from the expansion that
$$
\xi_r \simeq l \xi_h \qquad {\rm or} \qquad
y_2 \simeq ( l + 1) y_1 \quad {\rm for}  \quad x \to  0 \; ,
\eqno(2.9)
$$
and
$$
{\dd \Phi '  \over \dd r } \simeq {l \over r }\Phi ' \qquad {\rm or} \qquad
y_4 \simeq ( l - 2 ) y_3 \quad {\rm for}  \quad x  \to 0 \; .
\eqno(2.10)
$$

\subsect
{\it 2.3.2 Truncated model}

In a model truncated at a finite distance from
the centre it is assumed that the calculation is
for radial oscillations, or in the Cowling approximation.
Here the programme allows four types of conditions,
flagged by the input parameter {\tt ibotbc} ({\cf} Section 7.2.3 below):

\medskip
\item{i)}
({\tt ibotbc = 0})
Applicable when the innermost meshpoint $x_1$ 
is in an evanescent region.
The condition sets the relation between 
$y_1$ and $y_2$ so as to select the solution that decreases
exponentially towards the interior.
From the JWKB relation for $y_1$, equations (2.7) and (2.8), and
the differential equation for $y_1$ one finds
that approximately
$$
y_2 ( x_1 ) \simeq 
{x_1 | k(x_1 ) | \over 1 - \sigma^2 / {\tilde S }_l^2 (x_1 )} y_1 (x_1 ) \; .
\eqno (2.11)
$$
This condition is mainly useful for modes of relatively large degree.

\medskip
\item{ii)}
({\tt ibotbc = 1})
Specification of the ratio between $y_1 $ and $y_2$ 
at the innermost mesh point $x_1$ by
$$
y_1 (x_1 ) = \alpha ( 1 - F_{\rm BC} ) , \qquad
y_2 (x_1 ) = \alpha F_{\rm BC}  \; , 
\eqno (2.12)
$$
where $\alpha$ is an arbitrary scale factor.
Note that this condition can also be written as
$$
F_{\rm BC} y_1(x_1) - (1 - F_{\rm BC}) y_2(x_1) = 0 \; .
\eqno(2.12a)
$$
Here $F_{\rm BC}$ is determined by the input parameter
{\tt fcttbc} ({\cf} Section 7.2.3).

\medskip
\item{iii)}
({\tt ibotbc = 2})
Setting 
$$
{\dd y_1 \over \dd x} = 0 \qquad \hbox{\rm at} \quad x = x_1 \; .
\eqno(2.13)
$$
This can be expressed as a relation between $y_1(x_1)$ and
$y_2(x_1)$ by using the differential equation for $y_1$
({\cf} Appendix A).

\medskip
\item{iv)}
({\tt ibotbc = 3})
Setting 
$$
{\dd \over \dd x}\left({ y_1 \over x} \right)  = 0
\qquad \hbox{\rm at} \quad x = x_1 \; .
\eqno(2.14)
$$
This can be expressed as a relation between $y_1(x_1)$ and
$y_2(x_1)$ by using the differential equation for $y_1$
({\cf} Appendix A).

\subsect
{\bf 2.4 Surface conditions}

The surface conditions depend on whether or not the surface pressure of
the model vanishes. For vanishing surface pressure 
(as occurs, for example, in complete polytropic models)
the surface is a singular point where regularity conditions must be imposed.
When the pressure is non-vanishing, as in realistic stellar models
truncated at a suitable point in the atmosphere,
various conditions can be used. 
These are selected by the input parameters {\tt istsbc} and {\tt fctsbc}.

For nonradial oscillations the condition on the gravitational
potential perturbation is common to the singular and nonsingular cases.
It is obtained by demanding continuity
of $\Phi^{\prime}$ and its first derivative; in terms of the variables
used here this may be expressed as
$$
y_4 (\xs ) = -  [ l + U(\xs )]  y_3 + \lambda U(\xs ) y_1 \; ,
\eqno (2.15)
$$
where $\xs$ is the value of $x$ at the surface, and 
$U = 4 \pi \rho r^3 / m$.
For vanishing surface density, as would normally be the case
for a singular surface, the terms in $U(\xs)$ vanish.
However, for the iso-pycnic model (corresponding to constant density,
{\ie}, to a polytrope of index 0), $U(\xs) = 3$.

\subsect
{\it 2.4.1 Singular surface}

When the surface pressure vanishes, the solution in the
vicinity of the singular point is obtained by expansion
around this singularity, retaining terms that are
$O ( t )$ times the leading order term, where $t$ is the depth below the
surface (it might be noted here that the expansion depends qualitatively
on whether or not the surface density vanishes as well, {\ie},
whether or not the effective polytropic index $\mu$ at the surface 
is non-zero).
Details of this procedure, including expressions for the
expansion coefficients, were given by Christensen-Dalsgaard \& Mullan (1994).

\subsect
{\it 2.4.2 Simple surface conditions}

When the surface pressure is non-vanishing, as is the case for a 
realistic stellar model, the programme allows various simple
expressions for the surface pressure condition.
These are selected by setting {\tt istsbc = 0}.
One is to use the condition $\delta p = 0$, which in terms of the
variables used here becomes
$$
y_2 (\xs ) = {l ( l + 1)  \over \sigma^2} [ y_1 (\xs ) - y_3 ( \xs ) ]
\eqno (2.16a)
$$ 
in the nonradial case, or
$$
y_2 (\xs ) = {1 \over \sigma^2} y_1 (\xs ) 
\eqno (2.16b)
$$
in the radial case.

To permit the use of different surface boundary conditions, 
a boundary condition on the form
$$
y_2 (\xs ) = {l ( l + 1)  \over \sigma^2} 
[ (1 - F_{\rm SBC} ) y_1 (\xs ) - y_3 ( \xs ) ]
\eqno (2.16c)
$$ 
in the nonradial case, or
$$
y_2 (\xs ) = {1 \over \sigma^2} (1 - F_{\rm SBC} ) 
 y_1 (\xs ) 
\eqno (2.16d)
$$
in the radial case, may be used.
Here $F_{\rm SBC}$ is an input parameter to the programme.
For $F_{\rm SBC} = 0$ we recover the conditions (2.16a) or (2.16b),
{\ie}, $\delta p = 0$, whereas for $F_{\rm SBC} = 1$ the condition
is equivalent to $p ' = 0$, {\ie}, the vanishing of the Eulerian
pressure perturbation.
$F_{\rm SBC}$ is determined by the input parameter {\tt fctsbc}.

\subsect
{\it 2.4.3 Match to the solution for an isothermal atmosphere}

A more realistic condition is obtained by matching the solution
onto the energetically confined of the two 
(analytically known) solutions for adiabatic waves in an isothermal
atmosphere matched to the model ({\cf} Unno {\etal} 1989, pp 163 {\it ff}).
This condition is selected by setting {\tt istsbc = 1}.
It has been implemented in the following form: Introduce
$$
V_g = {G m \rho  \over \Gamma_1 p r} \; , \qquad
A_{\rm i} =  V_g( \Gamma_1 - 1) \; ,
\eqno (2.17)
$$
and set
$$
\gamma = (A_{\rm i} + 4 - V_g )^2 + 4 ( \sigma^2 
x^3 - A_{\rm i}) \left( {l ( l + 1)  \over \sigma^2 x^3} - V_g \right) \; ,
\eqno (2.18)
$$
and
$$
C = {{1 \over 2 }( \gamma^{1/2} + V_g - A_{\rm i} ) - 2 
\over V_g - {\displaystyle l ( l + 1)  \over \displaystyle \sigma^2 x^3 }} \; ,
\eqno (2.19)
$$
evaluated at $x = \xs$; then the condition in the nonradial case is
$$
y_2 (\xs )
= {l ( l + 1)  \over \sigma^2 \xs^2}
\left\{ C y_1 (\xs ) -
\left[ 1 + \left( {l ( l + 1)  \over \sigma^2 \xs^3} - l - 1 \right) 
(V_g + A_{\rm i} )^{-1} \right] y_3 (\xs ) \right\} \; , 
\eqno (2.20a)
$$
and in the radial case 
$$
y_2 (\xs ) = {C \over x^2 \sigma^2} y_1 (\xs ) \; .
\eqno (2.20b)
$$
Note that in the nonradial case the term in $y_3$ has been
approximated by assuming that both the frequency and the degree is
``small'' ({\cf} Unno {\etal} 1989);
if this is not the case $y_3$ is small and
the term is almost certainly negligible
(it may well have negligible influence on the solution in any case). 

It should be noted that the expression for $A_{\rm i}$ is the
value of $A$ ({\cf} equation 5.1) in the isothermal case,
where $\dd \ln \rho / \dd \ln r = \dd \ln p / \dd \ln r$.
This is the form specifically used in the programme.

To use this condition we must clearly require that 
$\gamma \ge 0$; this approximately corresponds to requiring that
the frequency be below the local acoustical cut-off frequency at
$x = \xs$. If this were not the case, the solution in the
isothermal atmosphere would have a running-wave component, and the
system would no be longer conservative; thus the eigenfrequency and the
eigenfunctions become complex, and the problem can no longer
be treated with the present programme.
If a solution is attempted above this limit,
the condition corresponding to {\tt istsbc = 0} is used,
and a warning message is printed.
Furthermore, the value of {\tt istsbc} used in setting {\tt icase}
({\cf} Section 8.2) is set to zero.

\mainsect
\centerline{\twelvebf 3 Numerical techniques} 

The numerical problem can be formulated generally as that of solving
$$
{\dd y_i  \over \dd x} = \sum_{j=1}^I a_{ij} (x) y_j (x) \; , \qquad
\hbox{\rm for }  i = 1, \ldots , I \; ,
\eqno (3.1)
$$
with the boundary conditions
$$
\sum_{j=1}^I b_{ij} y_j (x_1 ) = 0 \; , \qquad
\hbox{\rm for }  i = 1, \ldots \; ,  I/2 \; ,
\eqno (3.2)
$$
$$
\sum_{j=1}^I c_{ij} y_j (\xs ) = 0 \; , \qquad
\hbox{\rm for }  i =  1 \; , \ldots,  I/2 \; .
\eqno (3.3)
$$
Here the order $I$ of the system is 4 for the full nonradial case, and
2 for radial oscillations or nonradial oscillations in the Cowling 
approximation. This system only allows non-trivial solutions for
selected values of $\sigma^2$ which is thus an eigenvalue of
the problem.

The programme permits solving these equations with two basically different
techniques, each with some variants. The first is a shooting method,
where solutions satisfying the boundary conditions are integrated 
separately from the inner and outer boundary, and the eigenvalue
is found by matching these solutions at a suitable inner fitting point
$\xf$ (defined by the input parameter {\tt xfit}).
The second technique is to solve the equations, together with a 
normalization condition and either all, or all but one, of the
boundary conditions using a relaxation technique; the eigenvalue is
then found by requiring continuity of one of the eigenfunctions at
an interior matching point when all the boundary conditions are
satisfied, or by requiring that the remaining boundary condition
be satisfied.
The choice of integration method is controlled by the input
parameter {\tt mdintg} ({\cf} Section 7.2.3).

For simplicity we do not distinguish between $\hat y_i$ and
$y_i$ ({\cf} Section 2.2) in the bulk of this Section.
It is implicitly understood
that the dependent variable (which is denoted $y_i$) is
$\hat y_i$ for $x < x_{\rm ev}$ and $y_i$ for $x \ge x_{\rm ev}$.
The numerical treatment of the transition between $\hat y_i$ and 
$y_i$ is discussed in Section 3.3.

\subsect
{\bf 3.1 The shooting method} 

It is convenient here to distinguish between $I$ = 2 and $I$ = 4.
For $I$ = 2 the differential equations (3.1) have a unique (apart
from normalization) solution $y_i^{\rm (i)}$ satisfying the
inner boundary conditions (3.2), and a unique solution $y_i^{\rm (o)}$
satisfying the outer boundary conditions (3.3). These may be obtained
by numerical integration of the equations. 
The final solution can then
be represented as $y_j = C^{\rm (i)} y_j^{\rm (i)} = C^{\rm (o)} y_j^{\rm (o)}$.
The eigenvalue is 
obtained by requiring that the solutions agree at a suitable matching
point $x = \xf$, say. Thus
$$
\eqalign{
& C^{\rm (i)} y_1^{\rm (i)} (\xf ) = C^{\rm (o)} y_1^{\rm (o)} (\xf ) \; , \cr
& C^{\rm (i)} y_2^{\rm (i)} (\xf ) = C^{\rm (o)} y_2^{\rm (o)} (\xf ) \; . \cr
}
\eqno (3.4)
$$
These equations clearly have a non-trivial solution 
$(C^{\rm (i)} , C^{\rm (o)} )$
only when their determinant vanishes, 
{\ie}, when
$$
\Delta \equiv y_1^{\rm (i)} ( \xf ) y_2^{\rm (o)} ( \xf )
- y_2^{\rm (i)} ( \xf ) y_1^{\rm (o)} ( \xf ) = 0 \; .
\eqno (3.5)
$$
Equation (3.5) is therefore the eigenvalue equation.

For $I$ = 4 there are two linearly independent solutions satisfying
the inner boundary conditions, and two linearly independent solutions
satisfying the outer boundary conditions. The former set $\{y_i^{\rm (i,1)} ,
y_i^{\rm (i,2)}\}$ is chosen by setting
$$
\eqalign{
& y_1^{\rm (i,1)} (x_1 ) = 1 \; , \qquad
  y_3^{\rm (i,1)} (x_1 ) = 0 \; , \cr
& y_1^{\rm (i,2)} (x_1 ) = 1 \; , \qquad
  y_3^{\rm (i,2)} (x_1 ) = 1 \; , \cr
}
\eqno (3.6)
$$
and the latter set $\{y_i^{\rm (o,1)} , y_i^{\rm (o,2)}\}$ is chosen by setting
$$
\eqalign{
& y_1^{\rm (o,1)} (\xs ) = 1 \; , \qquad
  y_3^{\rm (o,1)} (\xs ) = 0 \; , \cr
& y_1^{\rm (o,2)} (\xs ) = 1 \; , \qquad
  y_3^{\rm (o,2)} (\xs ) = 1 \; . \cr
}
\eqno (3.7)
$$
The inner and outer boundary conditions are such that, given $y_1$
and $y_3$, $y_2$ and $y_4$ may be calculated from them;
thus equations (3.6) and (3.7) completely specify the solutions,
which may then be obtained by integrating from the inner or outer
boundary. 
The final solution can then
be represented as 
$$
y_j = C^{\rm (i,1)} y_j^{\rm (i,1)} + C^{\rm (i,2)} y_j^{\rm (i,2)} =
C^{\rm (o,1)} y_j^{\rm (o,1)} + C^{\rm (o,2)} y_j^{\rm (o,2)} \; .
$$
At the fitting point $\xf$ continuity of the solution
requires that
$$
C^{\rm (i,1)} y_j^{\rm (i,1)} (\xf ) + C^{\rm (i,2)} y_j^{\rm (i,2)} (\xf ) =
C^{\rm (o,1)} y_j^{\rm (o,1)} (\xf ) + C^{\rm (o,2)} y_j^{\rm (o,2)} (\xf )
\qquad j = 1, 2, 3, 4 \; . 
\eqno (3.8)
$$
This set of equations only has a non-trivial solution if
$$
\Delta = \det \left\{ \matrix {
y_{1,\f}^{\rm (i,1)}& 
y_{1,\f}^{\rm (i,2)}& 
y_{1,\f}^{\rm (o,1)}& 
y_{1,\f}^{\rm (o,2)}& \cr
\noalign{\vskip3pt}
y_{2,\f}^{\rm (i,1)}& 
y_{2,\f}^{\rm (i,2)}& 
y_{2,\f}^{\rm (o,1)}& 
y_{2,\f}^{\rm (o,2)}& \cr
\noalign{\vskip3pt}
y_{3,\f}^{\rm (i,1)}& 
y_{3,\f}^{\rm (i,2)}& 
y_{3,\f}^{\rm (o,1)}& 
y_{3,\f}^{\rm (o,2)}& \cr
\noalign{\vskip3pt}
y_{4,\f}^{\rm (i,1)}& 
y_{4,\f}^{\rm (i,2)}& 
y_{4,\f}^{\rm (o,1)}& 
y_{4,\f}^{\rm (o,2)}& \cr}\right\} = 0 \; ,
\eqno(3.9)
$$
where, {\eg},
$y_{j,\f}^{\rm (i,1)} \equiv y_j^{\rm (i,1)} (\xf )$.
Thus equation (3.9) is the eigenvalue equation in this case.

Clearly $\Delta$ as defined in either equation (3.5) or equation (3.9) is
a smooth function of $\sigma^2$, and the eigenfrequencies are
found as the zeros of this function. This is done in the programme using
a standard secant technique: given two values $\sigma_{i-1}^2$
and $\sigma_i^2$ of $\sigma^2$ and the associated values of
$\Delta$, the new value of $\sigma^2$ is found as
$$
\sigma_{i+1}^2 = \sigma_i^2 -
\Delta_i {\sigma_i^2 - \sigma_{i-1}^2   \over \Delta_i - \Delta_{i-1} } \; ,
\eqno (3.10)
$$
where $\Delta_j \equiv \Delta ( \sigma_j^2 ) $.
However, the
programme also has the option for scanning through a given interval
in $\sigma^2$ to look for change of sign of $\Delta$, possibly
iterating for the eigenfrequency at each change of sign.
Thus it is possible to search a given region 
of the spectrum completely automatically.

The programme allows the use of three different techniques for solving
the differential equations. One is the standard second-order
centred difference technique, where the differential equations are
replaced by the difference equations
$$
{y_i^{n+1} - y_i^n \over x^{n+1} - x^n} = 
{1 \over 2 } \sum_{j=1}^I \left[ a_{ij}^n  y_j^n + a_{ij}^{n+1}  y_j^{n+1}
\right] , \quad i = 1, \ldots, I \; .
\eqno (3.11)
$$
Here we have introduced a mesh 
$x_1 = x^1 < x^2 < \cdots < x^N = \xs$ in $x$,
where $N$ is the total number of mesh points; 
$y_i^n \equiv y_i ( x^n )$, and 
$a_{ij}^n \equiv a_{ij} (x^n )$. These equations allow
the solution at $x = x^{n+1}$ to be determined from the
solution at $x = x^n$.

The second technique was
proposed by Gabriel \& Noels (1976); here on each mesh interval
$(x^n , x^{n+1})$ we consider the equations 
$$
{\dd y_i^{(n)}  \over \dd x} = \sum_{j=1}^I  {\bar a}_{ij}^n  
y_j^{(n)} (x), \qquad
\hbox{\rm for } i = 1 \; , \ldots, I \; ,
\eqno (3.12)
$$
with constant coefficients, where 
${\bar a}_{ij}^n \equiv 1/2 ( a_{ij}^n + a_{ij}^{n+1} )$.
These equations may
be solved analytically on the mesh intervals, and the complete
solution is obtained by continuous matching at the mesh points.
This technique clearly permits the computation of solutions varying
arbitrarily rapidly, 
{\ie}, the calculation of modes of arbitrarily high order. On the other
hand solving equations (3.12) involves finding the eigenvalues
and eigenvectors of the coefficient matrix, and therefore becomes
very complex and time consuming for higher-order systems. Thus in
practice it has only been implemented for systems of order 2, 
{\ie}, radial oscillations or non-radial oscillations in the Cowling
approximation.

The third technique uses the fourth-order difference scheme developed
by Cash \& Moore (1980).
This involves evaluating the right-hand side of the equations at points
intermediate between the meshpoints.
To do so the code interpolates the model quantities linearly between
the meshpoints (higher-order interpolation, to be formally consistent with
the order of the scheme, would create problems in the vicinity of sharp
variations, {\eg}, in the composition in the model).
Although the scheme is therefore not formally of high order, in practice
it appears to work very well for high-order modes where the variation in
the solution is far more rapid than the variation of the equilibrium
quantities.
Unlike the scheme developed by Gabriel \& Noels (1976), this scheme can
also be used for the full fourth-order system.

\subsect
{\bf 3.2 The relaxation technique} 

Two variants of this technique have been implemented. In the first
one of the boundary conditions (to be definite we assume here
that it is the first surface condition) is set aside to be used
for determining the eigenfrequency. The difference equations (3.11)
for $n = 1, 2, \ldots, N - 1$, the
boundary conditions (3.2), the normalization condition $y_1
( \xs ) = 1$, and, for $I$ = 4, the remaining surface
boundary condition, are then solved to give the solution
$\{y_i^n\}$ at each mesh point. Notice that the equations
and boundary conditions constitute a set of linear equations
for the solution, and this set may be solved efficiently by
forward elimination and backsubstitution 
({\eg} Baker, Moore \& Spiegel 1974).
The eigenvalue is then found by requiring that the remaining
boundary condition be satisfied; thus we evaluate
$$
\Delta = {\sum_{j=1}^I  c_{1j}  y_j (\xs )  
\over  \left[ \sum_{j=1}^I  \left( y_j^{\rm (norm)} \right)^2
\right]^{1/2} } \; ,
\eqno (3.13)
$$
where $y_j^{\rm (norm)} = y_j ( x^{\rm (norm)} )$ is the
solution at a suitably chosen normalization point; in the present
case, where a surface boundary condition is used, $x^{\rm (norm)}$
is typically the innermost mesh point (the added flexibility of
being able to vary $x^{\rm (norm)}$ has occasionally been found to
be useful). $\Delta$ is then a smooth function of $\sigma^2$,
and the eigenvalues may be found as the zeros of $\Delta$,
essentially as described in connection with the shooting technique.

As both boundaries, at least in a complete model, are either
singular or very nearly singular, the removal of one of the
boundary conditions tends to produce solutions that are somewhat
ill-behaved, in particular for modes of high degree. This in turn
is reflected in the behaviour of $\Delta$ as a function of $\sigma^2$.
This problem is avoided in the second variant of the relaxation
technique. Here the difference equations are solved separately
for $x \le \xf$ and $x \ge \xf$, by introducing a double
point $\xf^- = x^{n_f} = x^{n_f + 1} = \xf^+$ in the mesh.
The solution is furthermore required to
satisfy the boundary conditions (3.2) and (3.3), a suitable
normalization condition ({\eg}
$y_1 ( \xs ) = 1$), and continuity of all but one of the
variables at $x = \xf$, {\ie},
$$
\eqalign{
& y_1 ( \xf^- ) = y_1 ( \xf^+ ) \; , \cr
& y_3 ( \xf^- ) = y_3 ( \xf^+ ) \; , \cr
& y_4 ( \xf^- ) = y_4 ( \xf^+ ) \; , \cr
}
\eqno (3.14)
$$
(when $I$ = 2 clearly only the first continuity condition is used)
We then set
$$
\Delta = y_2 ( \xf^- ) - y_2 (\xf^+ ) \; ,
\eqno (3.15)
$$
and the eigenvalues are found as the zeros of $\Delta$, regarded as
a function of $\sigma^2$. 
It should be noticed, however, that with
this definition, $\Delta$ may have singularities with discontinuous sign 
changes that are not associated with an eigenvalue. Assuming that the
normalization is at the outer boundary, as used above, this 
occurs when $y_1 (\xf^- )$ is very close to a zero. Then
the continuous fitting to $y_1 ( \xf^+ )$, which is
in general not close to a zero, forces the interior solution, and
hence $\Delta$, to be very large. Thus care is needed when searching for
the eigenvalues by stepping through a range in $\sigma^2$,
when this method is used. However, close to an eigenvalue $\Delta$ is
generally well-behaved, and the secant iteration ({\cf}
equation 3.10) may be used without problems.

There is a third variant of the relaxation technique, which has 
not, however, been implemented in the current version of the programme 
({\cf} Unno {\etal} 1989, pp. 167 -- 178).
Here the difference equations, the boundary conditions
and a normalization condition are solved simultaneously for the
solution $\{y_i^n\}$ and the eigenvalue $\sigma^2$. As the
eigenvalue is included as an unknown when solving the equations, 
the system is non-linear and must be solved by Newton-Raphson iteration.
This method in principle gives quadratic convergence towards the
solution (provided analytical derivatives of the equations and
boundary conditions with respect to the $y_i^n$ and $\sigma^2$
are used), as opposed to the somewhat slower convergence obtained with
the secant iteration. However, it requires an initial guess for both the
eigenvalue and the eigenfunction; experience with this method in a
different programme suggests that this guess has to be fairly close to
the correct solution, at least for modes that are predominantly trapped
in a restricted region of the model. Furthermore there is no natural way to 
scan the spectrum. Thus this method in practice almost certainly has to be
combined with a version of one of the methods discussed above, to provide
an initial estimate of the eigenvalue and eigenfunction. 
It could then be used with some
advantage for the final iterations towards the solution, where its
faster convergence could be exploited. This would be a fairly 
straightforward extension of the existing programme, and may be
implemented in future.

\subsect
{\bf 3.3 The transition from $\hat y_i$ to $y_i$} 

In practice, the variables $\hat y_i$ are defined as
$$
\hat y_i (x) = \left( {x \over x_{\rm ev}} \right)^{- l +1} y_i (x) \; ,
\eqno (3.16)
$$
rather than by equation (2.6). Also the transition point $x_{\rm ev}$ is 
taken at an existing point in the mesh. To simplify the logic in the
programme $x_{\rm ev}$ is forced to be closer to the centre than the 
fitting point $\xf$; if $\xf$ has been specified such that
this puts constraints on $x_{\rm ev}$ , a
warning is printed (note that, as discussed in Section 3.5 below,
the fitting point should not be placed in the evanescent region).

From the definition (3.16) it follows that 
$$
\hat y_i (x_{\rm ev} )= y_i (x_{\rm ev} ) \; .
\eqno (3.17)
$$
When using the shooting technique the integration
is carried out separately for $x$ between $x_1$ and $x_{\rm ev}$, and
$x$ between $x_{\rm ev}$ and $\xf$, and equations (3.17) are used to
provide initial conditions for the integration on the latter region.
When the relaxation technique is used, the integration is carried
out over the entire region between $x_1$ and $\xf$; to ensure
that the conditions (3.17) are satisfied a double point 
$(x_{\rm ev}^- , x_{\rm ev}^+)$ is introduced where $x_{\rm ev}^- = x_{\rm ev}$,
$x_{\rm ev}^+ = x_{\rm ev} + \epsilon$, and $\epsilon$ should be
chosen close to machine accuracy; in this way the difference
equations (3.11) approximately ensures that equations (3.17) are
satisfied. It would be possible, at the price of making the programme
somewhat more complicated, to enforce equations (3.17) strictly, but
this is unlikely to have any significant effect on the results.

\subsect
{\bf 3.4 Richardson extrapolation} 

The difference scheme (3.11), which is used by one
version of the shooting technique, and the relaxation technique,
is of second order.
Consequently the truncation errors in the eigenfrequency and eigenfunction
scale as $N^{-2}$.
If $\omega ( {1\over 2} N )$ and $\omega (N)$ are the eigenfrequencies obtained
from solutions with ${1\over 2} N $ and $N$ meshpoints, the leading order
error term therefore cancels in 
$$
\sigma_{\rm Ri} = {1 \over 3} [ 4 \sigma (N) - \sigma ( {1\over 2} N ) ] \; .
\eqno (3.18)
$$
This procedure, known as {\it Richardson extrapolation},
was used by Shibahashi \& Osaki (1981).
It provides an estimate of the eigenfrequency that is substantially
more accurate than $\sigma ( N )$, although of course at some
added computational expense.

%\revision{22/7/94}
It is obviously essential that the iteration converges to the
same mode in the calculations with $N$ and $N/2$ meshpoints.
To check this, the order of the solution obtained in the two cases
is calculated, according to the procedure described in 
equation (4.1) below.
If the two solutions have different orders, the combined solution
should in principle be rejected.
In practice, the situation may sometimes be more complex,
due to problems with the definition of the order.
It happens, particularly for $l = 1$ in evolved models,
that the computed order jumps during evolution, due to
a slight shift in the eigenfunction which eliminates a zero
(see section 4.2; as discussed there this problem may largely be 
avoided by using a determination of the order developed by Takata).
Similarly, corresponding differences between the eigenfunctions
computed with $N/2$ and $N$ points can lead to different orders
being inferred in the two cases, even though this is in fact the same
mode, and Richardson extrapolation therefore justified.
This is handled by the programme in the following way:
if different orders are detected, the difference between the
eigenfunctions is estimated, in terms of the norm defined
by the energy integral (equation 4.3), relative to the norm
of one of the eigenfunctions.
If this norm is comparable with the difference between the frequencies,
in a suitable sense, the mode is accepted, otherwise it is rejected.
In the former case a warning message, in the latter an error message,
is written to {\tt adipls-status.log}.
The details of the tests (which are handled by the subroutine {\tt testri})
may still have to be fine-tuned.
For high-order modes problems may also occur if the thinned-out mesh is
not sufficient to resolve the eigenfunction, identifying a mode that
is actually of different order in the vicinity of the frequency based on
the full mesh.
In such cases, it may be necessary to forego the Richardson extrapolation.


\subsect
{\bf 3.5 Operational experience} 

As implemented here the shooting technique is considerably faster than
the relaxation technique, and so should be used whenever possible
(notice that both techniques may use the difference equations (3.11)
and so they are numerically equivalent, in regions of the spectrum
where they both work). For {\it second-order systems} the shooting technique can
probably always be used; the integrations of the inner and outer solutions
should cause no problems, and the matching determinant in equation (3.5)
is well-behaved. For  {\it fourth-order systems},
however, this need not be the case. For modes where the perturbation
in the gravitational potential has little effect on the solution, the
two solutions $y_j^{\rm (i,1)}$ and $y_j^{\rm (i,2)}$, and similarly the
two solutions $y_j^{\rm (o,1)}$ and $y_j^{\rm (o,2)}$, are almost
linearly dependent, and so the matching determinant nearly vanishes for any
value of $\sigma^2$. This is therefore the situation where the relaxation
technique may be used with advantage. 
The point where the shooting method fails depends on the precision. 
With 4-byte real variables, such as is used in the single-precision
version on the Aarhus Alliant, 
experience has shown that for a solar
model the shooting technique may give problems for the full fourth order
system for 5 min modes of degree higher than 4.
On the NCAR Cray, with 8 byte real variables,
it was possible to go as high as $l$ = 40
for 5 min solar p modes; for strongly
trapped g modes there were problems at degree higher than about 15.
However, in general a little experimentation may be required to
determine the conditions under which one method should be preferred
over another.

It should be noticed that the conditions under which the shooting technique
for the full system gives problems are precisely the conditions where
the Cowling approximation (of neglecting the perturbation in the
gravitational potential) might be expected to be applicable. Thus
here the Cowling approximation may be adequate, and the resulting
second-order system can be solved using the shooting technique; the
resulting eigenfrequency may, if desired, be corrected for the
effects of the perturbations in the gravitational potential by
using Cowling's perturbation expression, and the perturbation
in the potential may be computed from the eigenfunction by using
the integral solution of Poisson's equation (this option is 
built into the programme). If a solution of the
full system is desired, a reasonable strategy is to first compute
the solution in the Cowling approximation with the shooting technique
(possibly stepping in $\sigma^2$ to search for the eigenvalues),
and to then solve the full set using the relaxation technique,
starting from the eigenvalues found in the Cowling approximation.

The ability to specify the fitting point $\xf$ gives considerably
flexibility in the calculation, and correct choice of $\xf$ is
often essential to obtain a solution. As a rule of thumb when
using the shooting technique $\xf$
should be chosen near the general maximum in the eigenfunction, so that
the integration both from the inner and from the outer 
boundary is in a direction where the solution is predominantly
increasing. However, the fitting point may not placed too close to
the surface, as otherwise the almost singular solution at the
surface may cause problems. 

%\revision{22/7/94}
Problems are almost certain to arise if the fitting point is placed
deep within the region where the solution is evanescent.
In particular, in older versions of the code the point
$x_{\rm ev}$ where the transition from scaled to
unscaled variables takes place was restricted to lie deeper
than the fitting point.
For very high-degree modes, where the turning point is within
a fraction of a per cent of the radius from the surface,
this may easily happen, unless $\xf$ is very close to unity.
In the present version of the code, this can be controlled in two
slightly different ways:
\medskip
\item{i)} if the parameter {\tt irsevn} is set to 2,
the fitting point is automatically reset to
$x_{\rm ev}$ if it is found that $\xf < x_{\rm ev}$.
The fitting point is reset to the input value before the
next mode is calculated.
\item{ii)}
if the parameter {\tt xfit} is set to -1, the fitting point
is always set to the boundary of the evanescent region.
\medskip

Roughly the same rules apply when using
the relaxation technique with fitting, although here the rate (or even success)
of convergence appears to somewhat more sensitive to the proper
choice of $\xf$. 
It has been found that the programme occasionally fails to
converge to a single mode, out of a large sample
or converges to a neighbouring mode.
The missing mode can then usually be found by choosing a slightly
different value for $\xf$.
%The programme has an option
%that automatically puts the fitting point at $x_{\rm ev}$.
%However, it is not clear that this is in fact in general a useful choice.
%There is still room for experimentation.
%\revision{22/7/94}
The programme has the option of automatically resetting $\xf$
if the iteration fails or the wrong mode is obtained,
by setting the parameter {\tt nftmax} to a value greater than one.
If this succeeds, a warning message is written to
{\tt adipls-status.log}.
After completing the given mode, $\xf$ is reset to the original value.

(As indicated previously the use of the relaxation technique, removing
one of the boundary conditions, can cause problems; in particular
it seems to give trouble for precisely the modes where the 
relaxation technique is indicated, {\ie}, modes of high degree.
This option was implemented before the relaxation
technique with interior matching, and has been left in the programme to allow
possible future experiments with its properties; however, for
practical calculations the relaxation technique with interior matching should
be preferred.)

\mainsect
\centerline{\twelvebf 4 Results of the calculation} 

The principal results of the calculation are the eigenfrequency $\sigma$
and the eigenfunction $y_i (x)$. However, the programme computes and
prints a number of other quantities, which are described in this section.

\subsect
{\bf 4.1 Frequency correction in the Cowling approximation}

When the equations are solved in the Cowling approximation for a
complete model, the correction
$\delta \sigma^2$ to $\sigma^2$ caused by the perturbation in the
gravitational potential is calculated from Cowling's (1941) perturbation
expression, as well as the corrected squared frequency 
$\sigma_{\rm c}^2 = \sigma^2 + \delta \sigma^2$;
here $\sigma^2$ denotes the value obtained as an eigenvalue.
The calculation is carried out by the subroutine {\tt gravpo}.
In addition the perturbation of the gravitational potential, in
the form of $y_3$ is calculated from the integral solution to
Poisson's equation; currently $y_4$ is not set.

\subsect
{\bf 4.2 Mode order}

The programme finds the order of the mode according to the Scuflaire (1974)
definition (see also Unno {\etal} 1989, p. 149 -- 158),
through calling subroutine {\tt order}.
Specifically the order is defined by
$$
n =
- \sum_{x_{z1}> 0} \sign \left( y_2  {\dd y_1 \over \dd x }\right) + n_0 \; .
\eqno (4.1)
$$
Here the sum is over the zeros $\{x_{z1}\}$ in $y_1$ (excluding the
centre), and $\sign$ is the sign function, $\sign(z) = 1$ if $z > 0$
and $\sign(z) = -1$ if $z < 0$. The value of $n_0$ depends on the
behaviour of the solution close to the innermost boundary: if $y_1$ and
$y_2$ have the same sign at the innermost mesh point, excluding the
centre, $n_0 = 0$, otherwise $n_0 = 1$. In particular, for a
complete model that includes the centre, it follows from the expansion
of the solution at the centre that $n_0 = 1$ for radial oscillations
and $n_0 = 0$ for non-radial oscillations. Thus the fundamental 
radial oscillation has order $n = 1$. Although this is contrary to the
commonly used convention of assigning order 0 to the fundamental radial
oscillation, the convention used here is in fact the more reasonable,
in the sense that it ensures that $n$ is invariant under a continuous
variation of $l$ from 0 to 1. With this definition $n > 0$ for p modes, 
$n = 0$ for f modes, and $n < 0$ for g modes.

It may be shown that this labelling is mathematically satisfactory in
the Cowling approximation, in the sense of being invariant under 
continuous variations of the equilibrium model, or under continuous
changes in $l$
(Gabriel \& Scuflaire 1979; Christensen-Dalsgaard 1980).
However, this is not always the case for modes 
corresponding to the full set of equations. 
In particular, for sufficiently centrally condensed models
an unreasonable order is calculated for the lowest-order p modes with $l = 1$;
such models include models of the present Sun and more highly evolved models, 
as well as polytropic models of sufficiently high 
polytropic index (see Christensen-Dalsgaard \& Mullan 1994).
This problem affects a very broad range of dipolar modes in models
of red giants.
The correct mode labelling can in principle be inferred for
such models by following the modes through
a sequence of models starting from a less centrally condensed model, {\eg},
from a ZAMS model where such problems apparently do not occur. 

The programme contains the option for correcting the order
set as in equation (4.1),
controlled by the input parameter {\tt irsord}:
if $1 \le {\tt irsord} \le 10$,
the order 
$n$ as calculated using equation (4.1) is replaced by $\tilde n = n + 1$
when $l = 1$ and $0 \le n \le {\tt irsord}$.
Thus {\tt irsord = 1} makes the correct resetting 
in the case of a traditional model of the present Sun.
It is important to realize, however, that the correction 
depends strongly on the nature of the model.
In particular, for more evolved $1 \Msun$
models the problems extend to higher order.

An alternative method for setting the order was proposed by
Lee (1985), on the basis of $\delta \Phi$ and $p'$.
It may be obtained in the programme by setting 
{\tt irsord = 11} or {\tt -11}.
The success with this approach has been less than convincing so far.
However, in a major breakthrough Takata (2005, 2006a,b) showed that
the eigenfunctions of dipolar modes satisfy an identity which allows
the oscillation equations to be cast as a second-order system.
This allows a rigorous determination of the mode order based on the
functions
$$
{\cal Y}_1 = {1 \over g} \left[ {\delta \Phi \over r}
- \delta \left({\dd \Phi \over \dd r} \right) \right] \; , \qquad
{\cal Y}_2 = {\delta p \over p} \; .
\eqno (4.2)
$$
The scheme has been implemented in the code by G\"ulnur Do{\u g}an
and appears to produce reliable orders of dipolar modes for a broad range
of models, including models of red giants.
It is used by setting {\tt irsord = 20}.


\subsect
{\bf 4.3 Mode energy, scaled eigenfunctions}

The programme calls subroutine {\tt ekin} to calculate
a dimensionless measure of the kinetic energy of pulsation.
Two different normalizations may be chosen, depending on the value of
the parameter {\tt iekinr}. 
For {\tt iekinr} $= 0$ (the default) the quantity calculated is
$$
E = {\int_{r_1}^{R_{\rm s}} [ \xi_r^2 + l ( l + 1) \xi_h^2 ]
 \rho  r^2  \dd r \over M  \xi_r (R_{\rm s} )^2}
= { \displaystyle \int_{x_1}^{\xs} \left[ y_1^2
+ y_2^2  / l ( l + 1) \right]  q U \dd x / x 
\over 4 \pi  y_1 (\xs )^2} \; ,
\eqno(4.3a)
$$
whereas for {\tt iekinr} $= 1$, 
$$
E = {\int_{r_1}^{R_{\rm s}} [ \xi_r^2 + l ( l + 1) \xi_h^2 ]
 \rho  r^2  \dd r \over 
 M [ \xi_r (R_{\rm phot} )^2 + l(l+1) \xi_h (R_{\rm phot} )^2]}
= { \int_{x_1}^{\xs} \left[ y_1^2
+ y_2^2  / l ( l + 1) \right]  q U \dd x / x  
\over 4 \pi [ y_1 (x_{\rm phot} )^2 + y_2 (x_{\rm phot} )^2/l(l+1)] } \; .
\eqno(4.3b)
$$
(For radial modes the terms in $y_2$ are not included.)
Here $r_1 \equiv  R  x_1$, $R_{\rm s} \equiv R  \xs$
and $R_{\rm phot} \equiv R $ are the
distance of the innermost mesh point from the centre and the surface
and photospheric radii of the model, respectively; 
$q = m/M$ and $U = 4 \pi \rho r^3 / m$.
Note that, as defined here, $E$ is related to the commonly used 
{\it mode mass} $M_{\rm mode}$ by $E = M_{\rm mode}/(4 \pi M)$.

To indicate the region where the mode predominantly resides 
(in an energetical sense) 
$$
\eqalign{
& z_1 (x) = \left( {4 \pi r^3 \rho  \over M }\right)^{1/2}  y_1 (x) 
= \left( {4 \pi r^3 \rho  \over M }\right)^{1/2}  {\xi_r(r) \over R} \; ,\cr
& z_2 (x) = {1 \over \sqrt {l ( l + 1) } }
\left( {4 \pi r^3 \rho  \over M }\right)^{1/2} y_2 (x) 
= \sqrt {l ( l + 1) } 
  \left( {4 \pi r^3 \rho  \over M }\right)^{1/2} { \xi_h(r) \over R} \; , \cr
}
\eqno (4.4)
$$
are calculated (for radial modes only $z_1$ is found).
These are defined in such a way that
$$
E = {1 \over 4 \pi  y_1 (\xs )^2} 
\int_{x_1}^{\xs} [ z_1^2 + z_2^2 ]  {\dd x \over x } \; 
\eqno(4.5)
$$
[assuming the definition (4.3a) of $E$].
For output purposes the normalized functions
$$
\hat z_1 = {z_1 \over z_{1, \rm max} } \qquad
\hbox{\rm and } \qquad
\hat z_2 = {z_2 \over z_{1, \rm max} } \; ,
\eqno(4.6)
$$
where $z_{1, \rm max}$ is the maximum value of $z_1$,
are generally used.

\subsect
{\bf 4.4 Variational frequency}

The programme has the option of calculating the frequency and period
based on the variational expression for $\omega^2$,
in the subroutine {\tt varfrq}
[see Christensen-Dalsgaard (1982) for details]. 
As discussed there,
different formulations have to be used for p and for g modes,
selected by the input parameter {\tt ivarf}.
It might also be noticed that, as implemented, the calculation
is based on only $y_1$ and $y_2$, with the contribution
from the perturbation in the gravitational potential being 
calculated using the integral solution to Poisson's equation.
Thus even for modes calculated in the Cowling approximation
the perturbation of the gravitational potential is taken into
account, essentially by using the perturbation technique,
in the calculation of the variational period.

Notice that the programme normally uses a different formulation
for radial and for nonradial modes.
There is an option (selected by setting the parameter
{\tt ivarf} to 3), where essentially the nonradial
formulation of the variational integral (in its p-mode form)
may be used for radial modes; this may be useful to get strict consistency
between the calculation for radial and non-radial modes.
However, this formulation apparently leads to fairly severe
cancellation and consequent loss of accuracy. 
Thus it probably cannot be recommended, although the modes for
which it might be of use must be investigated.

\subsect
{\bf 4.5 Kernels}

For a rotation law depending on $r$ only the rotational
splitting may be written
$$
\Delta \omega_{nlm} = m \beta_{nl} \int_0^{\xs}
K_{nl} (x)  \Omega (x) \dd x \; ,
\eqno (4.7)
$$
where $\Omega (x)$ is the angular velocity.
The programme has an option of
calculating and printing $\beta_{nl}$ and $K_{nl}$;
this is done by subroutine {\tt rotker} when the parameter
{\tt irotkr} is set to 1.

The programme contains also the option for computing the kernel
for computing the frequency change caused by a change in $\Gamma_1$
at fixed $p$ and $\rho$.
Specifically, the kernel $K_{nl}^{(\Gamma_1 )}$
is defined such that 
$$
{\delta \omega  \over \omega } = \int_0^{\xs}
K_{nl}^{( \Gamma_1 )} \delta \Gamma_1 \dd x \; .
\eqno (4.8)
$$
It is calculated by subroutine {\tt gm1ker} when the parameter
{\tt igm1kr} is set to 1.

\mainsect
\centerline{\twelvebf 5 Equilibrium model variables} 

The following variables are needed at each mesh point in the model:
$$\eqalign{
x &\equiv r / R \; , \cr
A_1 &\equiv q / x^3  , \qquad \hbox{\rm where } q = m / M \; , \cr
A_2 &= V_g \equiv -  {1 \over \Gamma_1} {\dd \ln p  \over \dd \ln r}
= {G m \rho   \over \Gamma_1 p r} \; , \cr
A_3 &\equiv \Gamma_1 \; , \cr
A_4 &= A \equiv {1 \over \Gamma_1} {\dd \ln p \over \dd \ln r} - 
{\dd \ln \rho   \over \dd \ln r} \; , \cr
A_5 &= U \equiv {4 \pi \rho r^3  \over m } \; .\cr
}
\eqno (5.1)
$$
These are clearly all dimensionless.
In addition we use the following ``global'' quantities for the model:
$$
\eqalign{
& D_1 \equiv M \; , \cr
& D_2 \equiv R \; , \cr
& D_3 \equiv p_{\rm c} \; , \cr
& D_4 \equiv \rho_{\rm c} \; , \cr
& D_5 \equiv 
- \left( {1 \over \Gamma_1 p }  {\dd^2 p \over \dd x^2} \right)_{\rm c} \; , \cr
& D_6  \equiv 
- \left(  {1 \over \rho } {\dd^2 \rho  \over \dd x^2} \right)_{\rm c} \; , \cr
& D_7  \equiv \mu \; , \cr
& D_8: \hbox{see below} \; .\cr
} \eqno(5.2)
$$
Here $R$ and $M$ are photospheric radius and mass of the model (the photosphere
being defined as the point where the temperature equals the effective
temperature). In a complete model 
$p_{\rm c}$ and $\rho_{\rm c}$ are central pressure and 
density, and $D_5$ and $D_6$ are evaluated at the centre.
In an envelope model (that does not include the centre) $D_3$ and
$D_4$ should be set to the values of pressure and density at the
innermost mesh point, and $D_5$ and $D_6$ may be set to zero.
The dimensional variables ({\ie},
$D_1 - D_4$) must be given in $cgs$ units.

For a model with vanishing surface pressure $\mu$ is the effective
polytropic index at the surface, so that in particular $\mu$ is the
polytropic index of a complete polytrope
(the polytropic index elsewhere is not needed in the calculation, and
so mixed polytropes can be considered); models with non-vanishing
surface pressure are flagged by having $\mu < 0$.
The notation is otherwise standard. The model may include an
atmosphere (for solar models a simplified atmosphere
extending out to roughly the temperature minimum is typically used). Thus
at the surface possibly $x > 1$.

These variables are convenient when the equations are formulated as
by {\eg} Dziembowski; but it should be possible to derive any set
of variables required for {\it adiabatic} oscillation calculations from them.
$D_5$ and $D_6$ are needed
for the expansion of the solution around the centre. In models with
vanishing surface pressure $D_7$ is used in the expansion at the
surface singular point.

For a plane-parallel equilibrium model, 
or when the equilibrium model is affected by turbulent pressure,
the definition of the equilibrium
variables must be somewhat modified. These modifications are discussed
in Sections A.2 and A.3, respectively.

The quantity $D_8$ is used to flag for a different number
of variables in the file, or otherwise a different structure.
Currently the non-standard options are to have $D_8 = 10$ or $20$.
In both cases, the file contains 6 variables $A_1 - A_6$ at each
mesh point.
For $D_8 = 10$, $A_6$ is related to the gradient in turbulent pressure
(see Section A.3.2);
for $D_8 = 20$, $A_6$ is related to the ratio between the full and
reduced gravitational acceleration, in cases with rotation
(see Section A.4).
The value of $D_8$ is checked when the file is opened in
subroutine {\tt readml}; hence the structure must be the
same for all models in a given file.

Relations between the variables defined here and
more ``physical'' variables are often useful.
Here we provide relations valid when $A_6$ is not included;
for the case including rotation, see Section A.4.
We obtain:
$$
p = {G M^2  \over 4 \pi R^4}  {x^2 A_1^2 A_5 \over A_2 A_3} \; , \qquad
{\dd p \over \dd r }= - {G M^2  \over 4 \pi R^5} x A_1^2 A_5 \; , \qquad
\rho = {M \over 4 \pi R^3}  A_1 A_5 \; . 
\eqno (5.3)
$$
We may also express the characteristic frequencies for adiabatic
oscillations in terms of these variables. Thus if $N$ is the
buoyancy frequency, $S_l$ is the Lamb frequency and $\omega_a$ is the acoustical cut-off frequency for an isothermal
atmosphere, we have
$$
N^2 \equiv {G M \over R^3} {\tilde N}^2 =  {G M \over R^3}  A_1 A_4 \; , 
\eqno (5.4)
$$
$$
S_l^2 \equiv {l ( l + 1) c^2  \over r^2}
\equiv {G M \over R^3} {\tilde S}_l^2 = 
{G M \over R^3} {l ( l + 1) A_1  \over  A_2} \; , 
\eqno (5.5)
$$
and
$$
\omega_a^2 \equiv {c^2  \over 4 H_p^2}
= {1 \over 4 } {G M \over R^3}  A_1 A_2 A_3^2 \; , 
\eqno (5.6)
$$
where $c$ is the adiabatic sound speed, and $H_p = p / ( g \rho )$ is
the pressure scale height.
Finally it may be noted that the squared sound speed is given by
$$
c^2 = {G M \over R }x^2 {A_1  \over A_2} \; .
\eqno (5.7)
$$

\mainsect
\centerline{\twelvebf 6 Notation and data storage in the programme} 

\subsect
{\bf 6.1 Model storage}

Only a limited description of the notation in the programme is given in
this section, aimed at facilitating the later discussion of input to
and output from the programme.
Further comments on the organization of the programme are given in Appendix B.
However, a  detailed description of the code 
is outside the scope of these notes.

In the programme a mesh $\{ x^n \}$ is used, with $x^1 $
being the point closest to the centre and 
$x^{NN}$ at the surface; this is stored in a 
one-dimensional array ${\tt x(n)} = x^n ,  n = 1, 2, \ldots, NN$.
The $A_i$ are stored in the array {\tt aa(1:iaa,1:nn)}.
Here the first dimension {\tt iaa} is currently set to 10.
The variables are stored as
$$
{\tt aa(i,n)} = A_i ( x^n ) ,  \qquad i = 1,\ldots,{\tt ia}, 
\quad n = 1,\ldots, {\tt nn} \; ,
$$
where {\tt ia} is normally 5, but may be 6 in models including
one possible treatment of turbulent pressure ({\cf} section A.3.2);
{\tt aa(10,n)} is used for additional storage related to 
a different way of handling turbulent pressure.
Finally the $D_i$ are stored in the array {\tt data(1:8)} as
$$
{\tt data(i)} = D_i , \qquad  i = 1,\ldots,7  \; ;
$$
{\tt data(8)} is used only as a flag on model input 
({\cf} Section 5).

\subsect
{\bf 6.2 Storage of solution}

The {\it squared} eigenfrequency $\sigma^2$ is denoted by 
{\tt sig} (and {\tt sig1}, {\tt sig2}, {\tt sigp}, $\ldots$)
in the programme. Furthermore the degree $l$ of the modes is stored
in {\tt el};
as discussed in Section 2.1 $l$ is treated as a real variable.

The solution at {\tt x(n)} is set into {\tt y(i,n)}, 
as
$$
{\tt y(i,n)} = y_i ( x^n ) ,  \quad
{\tt i = 1,  \ldots,  ii, \quad  n = nw1,  \ldots,  nn} \; .
$$
Here {\tt ii} is the order of the system, {\ie},
2 for radial oscillations or non-radial oscillations in the Cowling
approximation, and 4 for non-radial oscillations with the full system.
Further {\tt nw1} = 1 if the model is not truncated in the interior, 
otherwise {\tt nw1 = ntrnct}, the truncation mesh point;
{\tt nnw = nn - nw1 + 1} is used for
the total number of points in the solution.

The inner boundary conditions are applied at the point {\tt nibc}.
When {\tt nw1} = 1, and {\tt x(1)} is at the centre of the model,
{\tt nibc} = 2.
Otherwise {\tt nibc = nw1}.

A more detailed description of the data storage is given in the source
for the programme.

\subsect
{\bf 6.3 Terminal input, and terminal and printed output} 

The unit numbers for these types of input and output
are defined in
\ms
{
\source
common/cstdio/ istdin, istdou, istdpr
}\msni
Here {\tt istdin} is the unit number for input (typically 5),
{\tt istdou} is the unit number for output to the terminal,
and {\tt istdpr} is the unit number for printed output.
For batch processing, one would obviously have
{\tt istdou = istdpr}.
These quantities are initialized in 
\ms
{
\source
block data blstio
}\msni
which must be set, depending on the installation.
{\tt istdpr} is contained in the list of control parameters,
and may be changed during the operation of the programme.
This allows, say, sending a short summary of the operation of
the programme, to verify that it is successful, to the terminal,
and the detailed printed output to a file.

\mainsect
\centerline{\twelvebf 7 Input to the programme} 

\intsect
{\bf 7.1 The equilibrium model} 

The equilibrium model must be supplied to the programme in a single,
unformatted record, to be read by the statement
\ms
{
\source
\ read(imlds) nmod, nnmodl, (data(i), i=1,8), (x(n), (aa(i,n), i=1,ia),
*  n=1,nnmodl)
}\msni
Here {\tt nmod} is an identification number of the model, which is not
used in the calculation.
Also, {\tt ia} should be 5, unless anything else is flagged by setting
{\tt data(8)}.
Note that the model storage must start at the centre.

It is important that the model supplied to the programme must have a reasonable
distribution of mesh points. There is currently no possibility for
resetting the mesh in the programme, apart from possibly reducing
the number of points by taking every {\tt in}-th point ({\cf} Section 7.2).
This is of course particularly important for high-order
modes, where it is essential to use different meshes for p and
for g modes. A separate programme exists for resetting the mesh
from that supplied by, {\eg}, the evolution programme.

\subsect
{\bf 7.2 Control parameters} 

\intsect
{\it 7.2.1 Input of control parameters} 

In the original version of the programme control parameters were 
input by means of {\tt namelist}.
However, {\tt namelist} is not a standard Fortran77 feature,
unfortunately; in particular, certain implementations
under UNIX have not permitted {\tt namelist}.
Also {\tt namelist} may not be optimal for interactive use
of the programme.

Thus the programme uses list-directed input.
This takes place in separate
lines, containing up to 9 variables.
On {\it input}, each line is prompted by printing
the names of the variable, and the current values,
as in
\ms
{
\source
itrsig, sig1, istsig, inomde, itrds ?
0 10.00 1 1 10 
}\msni
The new values are then input.
Note that if a value is not given for a variable, it is
not changed.
Thus by inputting, {\eg},
\ms
{
\source
,5.,,2,,
}\msni
in response, {\tt sig1} is set to 5.0, {\tt inomde} to 2, and the
remaining variables are unchanged.
If a line consisting solely of commas is input, all variables are unchanged.

Although it is possible in this way to enter the data from the terminal,
this is somewhat impractical, given the large number of parameters.
The usual way of providing input to the code is through the use
of an input file.
This may conveniently be given the following structure:
\ms
{
\source
\qquad .
\qquad .
mod:
  ifind, xmod, imlds, in, nprmod,
,,,,,,,   @
  xtrnct, ntrnsf, imdmod,
0.,,,,,,,,,,,,,,,,,,,,,, @
osc:
  el, nsel, els1, dels, dfsig1, dfsig2, nsig1, nsig2
   ,4,0,1,0,0,,,,,,,     @
  itrsig, sig1, istsig, inomde, itrds,
    1, 5   0 ,    10,10,,,,,,,,   @
  dfsig, nsig, iscan, sig2,
,2,50,30,,,,,,,,,,,,,,,,,,,,,,,     @
eltrw1, eltrw2, sgtrw1, sgtrw2
,,0,-1,,,,,,,,,,,,,,,,,,,,,,,,,,,,  @
\qquad .
\qquad .
}
\msni
Here lines of headers giving variable names are followed by lines
setting their values.
The latter are indicated by the character ``{\tt @}'' at the end of the line.
Only these data lines should be passed to the programme.
This is achieved by passing the input file through a filter,
before the data are read in.\footnote*
{This extraction is trivially done under UNIX by using
the {\tt grep} programme.
More generally, it would be simple to write a Fortran programme
to perform the same task.
}
In practice the extraction is handled automatically by the
UNIX script which is invoked to run the programme
({\cf} {\it Notes on using the solar models and adiabatic pulsations package}).

The {\it output} gives a line with the variable names,
and a line giving their values, as
\ms
{
\source
itrsig, sig1, istsig, inomde, itrds
0 5.0 1 2 10    @
}\msni
Note the addition of the character ``{\tt @}''.
This is included in all output lines that correspond
to a parameter input line to the programme;
in this way, the output contains essentially the information
required to repeat the given run.

As the programme has a very large number of control
parameters (about 85!), these are separated into 5
groups.
The first control parameter, {\tt cntrd}, is a string which
controls which of these groups are modified.
Each group is indicated by a three-letter string, defined by
\medskip
{\obeylines\smallskip
{\tt mod:} read model controls.
{\tt osc:} read controls for mode selection.
{\tt rot:} read controls for including rotational effects
{\tt cst:} read new values of fundamental constants (currently %
just gravitational constant)
{\tt int:} read controls for type of equations or integration procedure.
{\tt out:} read controls for output.
{\tt dgn:} read controls for diagnostics.}
\bigskip

The fields may be separated by, e.g., ``.'' (but  {\it not} comma
or blank, as they act as variable separators in the
list-directed input).
Thus if {\tt cntrd} is ``{\tt mod.osc.int}'', the groups defining the model,
mode selection and integration procedures are read.
Note that this input quantity {\it must} be specified explicitly
({\ie}, it cannot be set to default values by means of ``,,,'').

\subsect
{\it 7.2.2 Assignment of unit numbers to files} 

In the current version of the programme, files are 
accessed by unit numbers which are assigned once and for all.
This is therefore very similar to the procedure used on
traditional main-frames, where the assignment takes place
with the appropriate JCL statements. -- 
In future a more flexible scheme may be implemented.

The assignment takes place in the programme, to make
it as installation independent as possible.
It is defined by means of a list, read from normal input, on the form
\ms
{
\source
   $<$unit number$>$    $<$file name$>$
}\msni
and terminated by
\ms
{
\source
   -1 ''
}\msni
As an example, the following block of an input file
\ms
{
\source
   2    'model' \qquad @
  10    '/usr/jcd/agsm/gsm' \qquad @
  11    'new.gsm' \qquad @
  15    'new.ssm' \qquad @
 -1 '' \qquad @
}\msni
assigns unit number 2 to the file ``{\tt model}'', unit number 10 to the
file {\tt /usr/jcd/agsm/gsm}, and so on
(note that the ``{\tt \ '\ }'' are required here
to ensure input of the entire string; the file name format is
peculiar to UNIX).
Notice the addition of ``{\tt @}'' to make these into input lines
passed to the programme.

The programme prompts for input of the file data,
giving information about the files required and the format.
The input data is echoed, with the addition to each line of the
character ``{\tt @}'', in accordance with the usage described
above in connection with list-directed input/output.

\subsect
{\it 7.2.3 Description of control parameters} 

Below is a complete list of the control variables and their meaning.
This is essentially copied from the list given in the programme.
To facilitate the use of the programme a discussion of some of the
options is given in Sections 7.3 -- 7.5.

Note that to separate the variables in the programme from the
other variables, the former have been written in {\tt typewriter type}.

\bigskip
\ref
{\tt cntrd}: String determining which control fields are read. See description
in Section 7.2.2 above.
\hfil\break \qquad (default: {\tt cntrd} = {\tt 'mod.osc.int.out.dgn'}.
  This reads all the control fields, except fundamental constants.)

\subsect
a) Equilibrium model controls (group {\tt mod}):

\param
{\tt ifind}, {\tt xmod}: determines which model is used.
\pparam
{\tt ifind} $<$ 0: do not read new model,
except if no model has been read so far.
\pparam
{\tt ifind} = --2: possibly reset model with subroutine {\tt modmod}
or set new truncation, even if no new model was read.
\pparam
{\tt ifind} = 0: rewind data set before reading model.
\pparam
{\tt ifind} = 1: read next model on data set.
\pparam
{\tt ifind} = 2: read model no. {\tt xmod} on dataset.
\pparam
{\tt xmod} is currently assumed be integer-valued.
Non-integer {\tt xmod} may be implemented later, 
with interpolation between models.
{\defaults ifind = -1
xmod = 0})
\param
{\tt imlds}: models are read from unit {\tt imlds}
{\defaults imlds = 2})
\param
{\tt in}: after reading model, take every {\tt in}-th point
{\defaults in = 1})
\param
{\tt nprmod}: if {\tt nprmod} $>$ 0 print the model at {\tt nprmod} points.
{\defaults nprmod = 0})
\param
{\tt xtrnct}: if {\tt xtrnct} $>$ 0 truncate the model 
fractional radius $r/R  = $ {\tt xtrnct}.
This is only implemented for radial
oscillations, or in the Cowling approximation. 
Thus if $l >  $ 0 and the model is truncated, {\tt icow} = 2 is
forced, and a diagnostics is printed.
{\defaults xtrnct = 0})
\param
{\tt ntrnsf}: if {\tt ntrnsf} $>$ 1 stop model at point no. {\tt
ntrnsf} from the
surface.
{\defaults ntrnsf = 0})
\param
{\tt imdmod}: when {\tt imdmod} $\not=$ 0 call 
{\tt s/r modmod}, which may be user-specified
to modify the model.
{\defaults imdmod = 0})

\subsect
b) Controls for $l$ and the trial frequency (group {\tt osc}):

\param
{\tt el}, {\tt nsel}, {\tt els1}, {\tt dels}:
controls for determining the value of the degree $l$.
\pparam
{\tt el}: when {\tt nsel} $\le$ 0 (and {\tt itrsig} $\not=$ 2;
see below) use $l$ = {\tt el} as input.
{\defaults el = 1})
\pparam
{\tt nsel}: when {\tt nsel} $\ge$ 1 step through 
$l = {\tt els1} + i*{\tt dels}$, 
$i = 0, \ldots, {\tt nsel}-1$. When {\tt iscan} $\le$ 1 (see below) 
{\tt sig} is given the initial value in {\tt sig1}.
The increment from one value of $l$ to the next may be controlled
by the parameters {\tt dfsig1, nsig1} (see below).
{\defaults nsel = 0
els1 = 10
dels = 1})
\param
{\tt dfsig1}, {\tt dfsig2}, {\tt nsig1}, {\tt nsig2}:
allows resetting of the {\tt sig1} and {\tt sig2}
during step in $l$ for {\tt nsel} $>$ 1.
For each new value of $l$,
{\tt sig1} and {\tt sig2} are incremented as determined by
({\tt nsig1, dfsig1}), ({\tt nsig2, dfsig2}),
in the same manner as in the definition of {\tt nsig} and {\tt dfsig} below.
{\defaults dfsig1 = 0
dfsig2 = 0
nsig1 = 1
nsig2 = 1})
\param
{\tt sig1}, {\tt sig2}, {\tt itrsig}, {\tt inomde}, {\tt istsig}, {\tt
itrds}, {\tt dfsig}, {\tt nsig}: determine trial frequency.  
\pparam
{\tt itrsig} = 0: trial frequency taken from {\tt sig1}.  
\pparam
{\tt itrsig} = 1: {\tt sig} is found initially from {\tt sig1}, and then, 
for {\tt jstsig} $= 2, \ldots$, {\tt istsig}, from previous value
{\tt sigp} and {\tt dfsig}. 
\pparam
Meaning of {\tt dfsig} depends on {\tt nsig}:
\ppparam
{\tt nsig} = 1: {\tt sig} = {\tt sigp} + {\tt dfsig}
\ppparam
{\tt nsig} = 2: {\tt dfsig} is increment in frequency
({\ie}, in $\sqrt {{\tt sig}}$),\nwl
{\tt sig} = $( \sqrt { {\tt sigp}} + {\tt dfsig} )^2 $.
\ppparam
{\tt nsig} = 3: {\tt dfsig} is increment in 1/(frequency)
({\ie}, in $1/ \sqrt {{\tt sig}}$),
\nwl
{\tt sig} = $(1/\sqrt {{\tt sigp}} +{\tt dfsig} )^{-2} $.
\pparam
(Note that {\tt itrsig} and {\tt nsig} have a special meaning when {\tt
iscan} $>$ 1, see below.)
\pparam
{\tt itrsig} = 2 or 3: find trial frequency from values on
grand summary residing on unit {\tt itrds}. 
If {\tt itrsig} = 2 mode no {\tt inomde} + {\tt jstsig} - 1 is taken, 
and {\tt el} is reset to the value for this mode.
If {\tt itrsig} = 3 the mode with the value of $l$ input as {\tt el}
and of order {\tt inomde} + {\tt jstsig} - 1 is used.
Here {\tt jstsig} is stepped through $1, \ldots,{ \tt istsig}$.
\pparam
{\tt itrsig} = 4 or 5: find trial frequency from values on short
summary residing on unit {\tt itrds}. Otherwise corresponds to
{\tt itrsig} = 2 or 3 above.
\pparam
{\tt itrsig} = 6 or 7: find trial frequency from 
cyclic frequencies (in $\mu$Hz) from file residing on d/s {\tt itrds};
the file is in ASCII, each record containing $l, n, \nu$,
where $\nu$ is the cyclic frequency in $\muHz$.
Otherwise {\tt itrsig} = 6 and 7 corresponds to itrsig = 2 and 3 above.
\pparam
{\tt itrsig} = --2 --- --5: Find trial frequencies from single-precision
files of grand or short summaries, as above for 
{\tt itrsig} = 2 --- 5.
\pparam
(Note that {\tt istsig} and {\tt inomde} have special meaning 
when {\tt iscan} $>$ 1; see below.)
{\defaults itrsig = 0
sig1 = 10
sig2 = 0
dfsig = 0.
nsig = 1
istsig = 1
inomde = 1
dfsig = 10})
\param
{\tt iscan}: flag for scan in {\tt sig}.
When {\tt iscan} $>$ 1 step in {\tt sig} with {\tt iscan} 
steps between {\tt sig1} and {\tt sig2}.
Step is uniform in squared frequency, frequency or period, for
{\tt nsig} = 1, 2, and 3, respectively.
When {\tt istsig} $\le$ 1, {\tt sig1} and {\tt sig2}
are limits in squared dimensionless frequency $\sigma^2$;
otherwise {\tt sig1} and {\tt sig2} are the limits in
cyclic frequency $\nu$, measured in mHz.
When {\tt itrsig} = 1 check for change of sign in the matching determinant
and iterate for eigenfrequency at each change of sign.
If {\tt inomde} $\ge 2$ stop scan after diagnostics from integration.
(Currently only for diagnostics from {\tt setbcs}, particularly when 
the frequency exceeds the acoustical cut-off frequency.)
{\defaults iscan = 1})
\param
{\tt eltrw1}, {\tt eltrw2}, {\tt sgtrw1}, {\tt sgtrw2}:
windows applied to trial $l$ or $\sigma^2$.
Only takes effect if {\tt eltrw1} $\le$ {\tt eltrw2}, 
respectively {\tt sgtrw1} $\le$ {\tt sgtrw2}.
{\defaults eltrw1 = 0
eltrw2 = -1
sgtrw1 = 0
sgtrw2 = -1})

\noindent
See also note (a) below.

\subsect
c) Controls for the inclusion of rotation (group {\tt rot}):

\param
{\tt irotsl}: if {\tt irotsl = 1}, calculate solution including first-order
rotational effects as in Soufi {\etal} (1998) for azimuthal order {\tt em}.
{\defaults irotsl = 0
em = 0})
\param
{\tt nsem}, {\tt ems1}, {\tt dems}: Controls step in azimuthal order, for
{\tt irotsl} = 1.
\pparam
{\tt nsem = -1}: step with step {\tt dems}, from {\tt -el} to {\tt el}
\pparam
{\tt nsem .gt. 0}: take {\tt nsem} steps, starting from {\tt ems1}, 
with step {\tt dems}
{\defaults
nsem = 0
ems1 = 0
dems = 0})

\subsect
d) Fundamental constants (group {\tt cst}):

\param
{\tt cgrav}: Value of gravitational constant (the default is value hardcoded
in the programme before 23/3/88)
{\defaults cgrav = $6.6732\times10^{-8}$})

\subsect
e) Controls for equations, boundary conditions and integration method
(group {\tt int}):

\param
{\tt iplneq}: when {\tt iplneq} = 1 use equations for plane-parallel layer
(note that model coefficients should then also correspond
to plane-parallel case). Only implemented for non-radial
oscillations; attempted calculations with $l$ = 0 are
skipped, and a diagnostics is printed.
{\defaults iplneq = 0})
\param
{\tt iturpr}: flag for turbulent pressure in equilibrium model
(see Section A.3; {\bf probably requires further testing})
{\defaults iturpr = 0})
\param
{\tt icow}: flag for Cowling approximation.
\pparam
{\tt icow} = 0: solve full equations.
\pparam
{\tt icow} = 1: solve equations in Cowling approximation and go back
and solve full equations with Cowling result as trial.
\pparam
{\tt icow} = 2: solve equations in Cowling approximation only. Also sets
frequencies corrected by perturbation technique.
\pparam
{\tt icow} = 3: as {\tt icow} = 2, except that Richardson extrapolation of
cyclic frequency is based on uncorrected eigenfrequency.
{\defaults icow = 0})
\param
{\tt alb}: fudge factor in Poisson's equation (the $\lambda$ in equation
2.3). Should be 1, except
when studying gradual transition between Cowling approximation
and full case.
{\defaults alb = 1.})
\param
{\tt istsbc}, {\tt fctsbc}: determines surface pressure boundary condition.
\pparam
{\tt istsbc} = 1: find condition by matching to exponentially
decaying solution in an isothermal atmosphere matched to
the outermost mesh point [{\ie}, use conditions (2.17) - (2.20)].
This assumes that the frequency is
below the acoustical cut-off frequency at that point.
Otherwise a message is printed and the condition for 
{\tt istsbc} = 0 is used.
\pparam
{\tt istsbc} = 0: use simple surface condition as given in
equations (2.16c) and (2.16d).
When {\tt fctsbc} = 0 (the default) this corresponds to $\delta p = 0$.
\pparam
{\tt istsbc} = 9: use $\delta r = 0$ on surface.
\pparam
{\tt fctsbc}: determines condition when {\tt istsbc} = 0:
the conditions (2.16c) or (2.16d) with $F_{\rm SBC}$ = {\tt fctsbc}.
Note that {\tt fctsbc} = 0 corresponds to using $\delta p$ = 0,
and {\tt fctsbc} = 1 corresponds to using $p '$ = 0. 
However, all intermediate cases are allowed.
{\defaults istsbc = 1
fctsbc = 0})
\param
{\tt ibotbc}, {\tt fcttbc}: determines bottom boundary condition in truncated
model.
\pparam
{\tt ibotbc} = 0: set relation between $y_1$ and $y_2$ 
at bottom to isolate
solution that decreases exponentially towards the interior ({\ie},
use condition 2.11).
This assumes that the bottom is in an evanescent region for
the $l$-value and frequency used. Otherwise the condition corresponding
to {\tt ibotbc} = 1 is used.
\pparam
{\tt ibotbc} = 1: use condition (2.12), with $F_{\rm BC}$ = {\tt fcttbc}.
\pparam
{\tt ibotbc} = 2: use condition (2.13), {\ie},
vanishing gradient of displacement.
\pparam
{\tt ibotbc} = 3: use condition (2.14), {\ie},
vanishing gradient of relative displacement.
{\defaults ibotbc = 0
fcttbc = 0})
\param
{\tt mdintg}: determines type of integration used (see Sections 3.1 and 3.2).
%{\tt mdintg} = 1 or 2 uses the
%shooting technique (Section 3.1), {\tt mdintg} = 3 the relaxation technique
%(Section 3.2).
\pparam
{\tt mdintg} = 1: use shooting method with centred
differences equation (3.11).
\pparam
{\tt mdintg} = 2: use shooting method with constant coefficient integration on 
each mesh interval (only implemented for second-order systems, {\ie},
radial or Cowling approximation; if {\tt mdintg} = 2
and $l > 0$ {\tt icow} = 2 is forced, and a diagnostics is printed).
\pparam
{\tt mdintg} = 3: use relaxation method.
Find frequency by iterating on interior
boundary condition for {\tt xfit} = 0, on outer boundary
condition for {\tt xfit} = 1, or on matching at an internal point for
$0 < {\tt xfit} < 1$.
\pparam
{\tt mdintg} = 5: use shooting method with fourth-order scheme
of Cash \& More (1980).
{\defaults mdintg = 1})
\param
{\tt iriche}: if {\tt iriche} = 1, Richardson extrapolation is used to
improve the computed frequency (but not the other quantities),
by solving initially the equations on every second meshpoint.
Note: trial frequency is decreased by {\tt dsigre}.
This option is currently only implemented for {\tt mdintg} = 1 or 3.
{\defaults iriche = 0}).
\param
{\tt xfit}: match solution at mesh point $\xf = x^{n_{\rm f}}$ = {\tt xfit}.
(Note that fitting point may be reset automatically by programme;
see Section 3.5).
For {\tt xfit} $= -1$ and $l > 0$ match solution at
edge of inner evanescent region.
{\defaults xfit = 0.5})
\param
{\tt fcnorm}: for {\tt mdintg} = 3 and {\tt xfit} = 0 or 1,
normalize boundary condition 
by solution at point $x^{\rm (norm)} = x^{n_{\rm norm}}$,
where $n_{\rm norm}$ = {\tt fcnorm}*{\tt nn}.
{\defaults fcnorm = 0.5})
\param
{\tt eps}: convergence of {\tt sig} assumed when relative change in 
{\tt sig} between two iterations is less than {\tt eps}.
\ppparam
(default: depends on internal precision, as set in variable {\tt epsprc}
in {\tt common/cprcns/}).
\param
{\tt epssol}: when {\tt mdintg} = 1, 2 or 5
convergence criterion for eigenfunction is
assumed to be that relative discontinuity at matching
point is less than {\tt epsol}.
When {\tt mdintg} = 3 convergence criterion
is assumed to be that the mean relative
change in the eigenfunction between two iterations is less than {\tt epssol}.
{\defaults epssol = $10^{-5}$})
\param
{\tt itmax}: maximum number of iterations.
\pparam
When {\tt itmax} = 0 just integrates once and, when {\tt mdintg} = 1, 2 or 5,
tests the continuity of the eigenfunction. This is useful
for re-computing eigenfunctions when the eigenfrequency is known.
This may also be used to output solution for given frequency, regardless
of whether it has converged, for {\tt nfmod1} = 1 (see {\tt nfmode} below).
{\defaults itmax = 8})
\param
{\tt dsigre}: 
%\revision{8/8/91}
when using Richardson extrapolation, decrease trial
{\tt sig} by {\tt dsigre} before iteration on thinned-out mesh
{\defaults dsigre = 0})
\param
{\tt fsig}: in the secant iteration for {\tt sig}, the second value is
 (1+{\tt fsig})*(the first trial value)
{\defaults fsig = 0.001})
\param
{\tt dsigmx}: the relative change in {\tt sig} during the iteration is limited
to be less than {\tt dsigmx} in absolute value.
{\defaults dsigmx = 0.1})
\param
{\tt irsevn}: Controls scaling of solution in evanescent region
({\cf} Section 3.3).
\pparam
{\tt irsevn} $= -1$: do not use scaling in evanescent region.
\pparam
{\tt irsevn} $\ge 1$: reset transition
point $x_{\rm ev}$ between modified and standard equations (at boundary
of evanescent inner region for the trial {\tt sig}) before iterating 
for each eigenvalue,
when iterating for eigenfrequency during
scan ({\ie}, for {\tt iscan} $>$ 1, {\tt itrsig} = 1).
\pparam
{\tt irsevn} = 2:
%\revision{22/7/94}
reset fitting point if it is deeper
than the evanescent transition (the default is to shift
instead the evanescent transition to fitting point).
{\defaults irsevn = 2})
\param
{\tt xmnevn}: Search for evanescent transition is restricted to
$x \ge$ {\tt xmnevn}.
{\defaults xmnevn = 0})
\param
{\tt nftmax}, {\tt itsord}: 
%\revision{22/7/94} 
When ${\tt nftmax > 1}$, attempt changing fitting
point up to {\tt nftmax} times, if iteration does not converge,
or, if {\tt itsord = 1}, if the correct order is not obtained.
Note that this is most likely to be effective for {\tt mdintg = 3}.
\pparam
{\tt itsord}: Test on order obtained from trial solution.
This is only usable for {\tt itrsig} = 2 -- 7.
{\defaults nftmax = 3
itsord = 0})

\subsect
f) Controls for output (group {\tt out}):

\param
{\tt istdpr}: unit number for printed output. Default set in block data
subprogram {\tt blstio}.
\param
{\tt nout}: when {\tt nout} $>$ 0 print solution at {\tt nout} points
{\defaults nout = 50})
\param
{\tt nprcen}: if {\tt nprcen} $>$ 1 print solution at all the {\tt
nprcen} points
closest to the centre.
{\defaults nprcen = 0})
\param
{\tt irsord}: Controls setting of order for modes of degree $l = 1$,
when the Cowling approximation is not used
({\cf} Section 4.2).
\pparam
{\tt irsord} between 1 and 10:
Increment Scuflaire order between 0 and {\tt irsord} by 1.
\pparam
{\tt irsord} = -- 11 or 11: Set order according to Lee's scheme.
When {\tt irsord} = --11, evaluate both values of the order,
and write diagnostics if they differ.
\pparam
{\tt irsord} = 20: Use the Takata (2006b) scheme for dipolar modes.
{\defaults irsord = 0}).
\param
{\tt iekinr}: Determines normalization of energy ({\cf} Section 4.3).
\pparam
{\tt iekinr} = 0: normalize energy with surface vertical displacement.
\pparam
{\tt iekinr} = 1: normalize with total photospheric displacement.
{\defaults iekinr = 0}).
\param
{\tt iper}, {\tt ivarf}, {\tt kvarf}, {\tt npvarf}:
controls for calculation of the variational frequency.
\pparam
{\tt iper}: when {\tt iper} = 1 calculate variational frequency.
{\defaults iper = 0})
\pparam
{\tt ivarf} = 1: use p-mode formulation (equations (D2) and (D3) of CD82).
\pparam
{\tt ivarf} = 2: use g-mode formulation (equations (D7) and (D8) of CD82).
\pparam
{\tt ivarf} = 3: use nonradial p-mode formulation also for radial modes.
(see note in Section 4.4).
{\defaults ivarf = 1})
\pparam
{\tt kvarf}: numerical differentiation and integration in calculation
of variational frequency uses polynomials of degree 2*{\tt kvarf}
{\defaults kvarf = 2})
\pparam
{\tt npvarf}: for {\tt npvarf} $>$ 0 print integrands and integrals in
variational calculation at {\tt npvarf} points.
{\defaults npvarf = 0})
\param
{\tt nfmode}: {\tt nfmode = nfmod0 + 10*nfmod1} controls output of solutions
to file. {\tt nfmod1 = 1} is used to output possibly unconverged solution
for a given frequency, computed with {\tt itmax = 0}, to file.
\pparam
For {\tt nfmod0} = 1, 2 or 3 write eigenfunctions to file on unit 4.
The output format depends on the value of {\tt nfmod0}
(see also Section 8.4):
\pparam
{\tt nfmod0 = 1}: full set of variables.
\pparam
{\tt nfmod0 = 2}: displacement eigenfunctions $y_1(x^n), y_2(x^n)$.
\pparam
{\tt nfmod0 = 3}: density-weighted displacement eigenfunctions
$\hat z_1(x^n), \hat z_2(x^n)$ ({\cf} equation 4.4).
{\defaults nfmode = 0})
\pparam
Note that {\tt nfmod0} $\ge 1$ can be used to output unconverged solutions to
file for {\tt iscan} $> 1$.
\param
{\tt irotkr}, {\tt nprtkr}: for {\tt irotkr} = 1 calculate rotational kernels.
If in addition {\tt nprtkr} $> 1$ rotational kernel is printed
at {\tt nprtkr} points.
{\defaults irotkr = 0
nprtkr = 50})
\param
{\tt igm1kr}, {\tt npgmkr}: for {\tt igm1kr} = 1 calculate $\Gamma_1$ kernels
({\cf} equation 4.8).
If in addition {\tt npgmkr} $> 1$, $\Gamma_1$ kernel is printed
at {\tt npgmkr} points.
{\defaults igm1kr = 0
npgmkr = 50})
\param
{\tt ispcpr}: if {\tt ispcpr} $\not=$ 0 special output may be produced by
call of user-supplied routine {\tt spcout}.
{\defaults ispcpr = 0})
\param
{\tt icaswn}: If {\tt icaswn} $\ge$ 0, only output modes to file for which 
{\tt icase = icaswn}
{\defaults icaswn $= -1$)}
\param 
{\tt sigwn1}, {\tt sigwn2}, {\tt frqwn1}, {\tt frqwn2}, {\tt iorwn1}, 
{\tt iorwn2}, {\tt frlwn1}, {\tt frlwn2}:
Windows for output of modes to file.
Windowing is applied only if first parameter is $\le$ second parameter
({\eg} {\tt sigwn1} $\le$ {\tt sigwn2}).
Defaults are no windowing.
\pparam
{\tt sigwn1, sigwn2}: Window in $\sigma^2$
\pparam
{\tt frqwn1, frqwn2}: Window in cyclic frequency $\nu$ (in $\muHz$) 
\pparam
{\tt iorwn1, iorwn2}: Window in mode order
\pparam
{\tt frlwn1, frlwn2}: Window in $\nu/(l + 1/2)$ ($\nu$ in $\muHz$)
{\defaults sigwn1 = 0
sigwn2 = $-1$
frqwn1 = 0
frqwn2 = $-1$
iorwn1 = 0
iorwn2 = $-1$
frlwn1 = 0
frlwn2 = $-1$})

\subsect
g) Controls for diagnostics (group {\tt dgn}):

\param
{\tt itssol}: when {\tt itssol} = 1 test solution with 
{\tt nrkm} right hand side routine.
For {\tt idgtss} = 1 additional details about solution etc. are printed.
{\defaults itssol = 0
idgtss = 0})
\param
{\tt moddet}, {\tt iprdet}: flags for modifying and printing 
matching determinant, for {\tt mdintg} = 1 or 2.
{\defaults moddet = 0
iprdet = 0})
\param
{\tt npout}: when finding eigenfrequency by matching and {\tt npout} $>$ 0 
print the separate solutions at {\tt npout} points.
{\defaults npout = 0})
\param
{\tt imstsl}, {\tt imissl}, {\tt imjssl}:
Parameters controlling the determination of the matching
coefficients in the full case.
The equation no. {\tt imissl} is ignored, and the coefficient no.
{\tt imjssl} is given a fixed value.
If {\tt imstsl} $\not=$ 1, {\tt imissl} and {\tt imjssl} are taken as input.
Otherwise {\tt imissl} and {\tt imjssl} are determined to minimize the error.
{\defaults imstsl = 1
imissl = 2
imjssl = 2})
\param
{\tt idgnrk}: determines level of diagnostics in nrkint solution,
for {\tt mdintg} = 3.
{\defaults idgnrk = 0}).

\noindent
{\bf Notes:} 

\item{(a)} Notes on preference in determining trial el and sig:
\itemitem{--}
{\tt nsel} $\ge 1$ may be combined with {\tt iscan} $> 1$ to produce
a scan between {\tt sig1} and {\tt sig2} in sig for each
{\tt el = els1, els1+dels, ..., els1+(nsel-1)*dels}.
\itemitem{--}
Test on {\tt iscan} takes place before test on {\tt itrsig}.
\itemitem{--}
{\tt nsel} $> 1$ cannot currently be combined with {\tt itrsig} $> 1$.
when {\tt itrsig} $> 1$ (or {\tt itrsig = 1} for {\tt iscan} $\le 1$) the
value of the degree in i{\tt el} is always used.

\subsect
{\bf 7.3 Comments on the input parameters} 

The number of parameters may appear overwhelming.
They have been introduced to provide a great deal of flexibility in
the handling of the input model, the choice of setting up trial
frequencies, the execution of the calculation,
and the control of the output to file.
In practice, standard settings of most parameters are quite adequate;
furthermore, template input files can be provided which illustrate
the various usages.
As a further aid, this section discusses some of the possibilities.

Certain settings of the parameters are inconsistent.
I have attempted to catch these cases.
When the intended usage is fairly clear,
the offending parameters is reset, with a warning message;
otherwise an error message is written and the programme proceeds
to the next set of input.
There is no guarantee that all possible combinations are checked
and caught, however.

To select modes on input or output, `windowing parameters'
are typically used.
For example, on output modes in a selected frequency
range $\nu_1 \le \nu \le \nu_2$ can be selected by setting
{\tt frqwn1 =} $\nu_1$, {\tt frqwn2 =} $\nu_2$.
Windowing is not invoked (fairly obviously) when the lower limit 
exceeds the upper one, in the present case when
{\tt frqwn1} $>$ {\tt frqwn2}.

\subsect
{\it 7.3.1 Notes on model} 

The default {\tt ifind} =$ -1$ ensures reading the first model 
on unit {\tt imlds} and using it in all the calculations. 
If the unit number {\tt imlds} is changed in
a subsequent read of control input, however,
{\tt ifind} must be set $\ge$ 0 to force reading a new model.

Modifications of the model are controlled by the parameters
{\tt xtrnct}, {\tt ntrnsf} and {\tt imdmod}.
These modifications may be applied to a model previously
read in by setting {\tt ifind = $-2$}.

Choosing ${\tt in} > 1$ allows using a coarser model than 
provided on the file. 
This may be useful, in saving computing time,
for computing low-order modes where high resolution is not required,
or for an initial survey of the modes in a model;
it is also useful for estimating the effect of truncation
errors on the eigenvalues and eigenfunctions.
Note that the truncation parameter {\tt ntrnsf} refers to the mesh
{\it after taking every {\tt in}-th point.}

\subsect
{\it 7.3.2 Notes on $l$ and trial eigenfrequency}

The large number of parameters may make this somewhat complex. However
the default values allow simple use, while on the other hand the parameters
permit very efficient computation of large sets of modes.
The two chief options are 
\medskip
\item {a)} To search for modes without any prior information about
their location.
This would typically be used for general stellar models, where
no previously computed set of frequencies is likely to be available.

\item{b)} To use a previously computed set of frequencies for trial.
This would often be used for solar models which are likely to be quite
similar to existing models for which extensive mode sets are 
available.

\subsect
{a) Searching for modes}

To find a mode with a given degree $l$ and near a given trial frequency
$\sigma_{\rm tr}$,
user only has to provide $l$ in {\tt el} $\sigma_{\rm tr}^2$ in
in {\tt sig1},
leaving all other parameters in the group {\tt osc}
at their default values.
To obtain a sequence of modes with the same $l$, for which the separation
between consecutive modes is approximately known,
one uses {\tt itrsig} = 1, and sets {\tt istsig} $>$ 1.
Then the programme attempts to find {\tt istsig} modes, the first with
$\sigma^2$ at {\tt sig1}, and the remaining determined from the last
obtained eigenvalue and the increment {\tt dfsig}.
The three possible 
ways of specifying the increment, as
determined by {\tt nsig}, reflects the
possible asymptotic behaviour of the eigenfrequencies
\medskip
\item{--} {\tt nsig} $ = 1$:
For low-order
p modes the frequencies are approximately uniformly spaced in $\sigma^2$;
here {\tt dfsig} provides the increment between modes in $\sigma^2$.
\item{--} {\tt nsig} $ = 2$:
For high-order p modes the spacing is asymptotically uniform in $\sigma$;
here {\tt dfsig} provides the increment between modes in $\sigma$.
\item{--} {\tt nsig} $ = 3$:
For high-order g modes
the spacing is asymptotically uniform in 1/$\sigma$;
here {\tt dfsig} provides the increment between modes in $\sigma^{-1}$.
Note that in this case the modes are obtained in the order of decreasing
frequency and order.
\medskip

Using {\tt iscan} $>$ 1 permits searching a given region of the spectrum for
eigenmodes,
between $\sigma^2 =$ {\tt sig1} and $\sigma^2 =$ {\tt sig2}.
The step is uniform in $\sigma^2$, $\sigma$ or
$\sigma^{-1}$ depending on {\tt nsig}, as discussed above.
When {\tt itrsig} $\not=$ 1 the programme only prints
a table of the matching determinant $\Delta$ (defined by
equation (3.5), (3.9), (3.13) or (3.15), depending on the 
integration parameters), as a function of $\sigma^2$.
This gives an initial idea about the spectrum. 
For {\tt itrsig} = 1
the programme in addition looks for changes of sign in $\Delta$,
and at each change of sign (except those associated with singularities
for {\tt mdintg} = 3; {\cf} Section 3.2) 
it attempts to iterate for the eigenvalue.
Thus automatic determination of all modes in a given region of the
spectrum is possible. To ensure that all modes are found with
reasonable certainty the number {\tt iscan} of steps should probably
be at least about four times the number of modes expected in the
region considered, and {\tt nsig} should of course be chosen appropriately
for the kind of modes expected. It might also be pointed out
that two very close eigenvalues (near an avoided crossing, say)
may manifest themselves as a minimum in $| \Delta ( \sigma^2 ) |$,
without changes of sign; in this case the search may be repeated
with smaller step near the minimum (this procedure could clearly be
automated, but that has so far not been done).

Instead of considering a single value of $l$ given in {\tt el},
the programme may be asked to step in $l$.
This is accomplished by setting {\tt nsel} $>$ 1;
{\tt nsel} is the number of $l$-values considered, {\tt els1}
is the initial value of $l$ and {\tt dels} is the step in $l$.
This may be used in connection with all the options mentioned above:
\medskip
\item{--}
To follow a single mode, or step through $\sigma^2$ with
{\tt istsig} $> 1$, {\tt itrsig} $ = 1$, the initial
trial $\sigma^2$ must be provided in {\tt sig1}.
For each step in $l$, the trial $\sigma^2$ is incremented
as determined by {\tt dfsig1} and {\tt nsig1}, in the same
way as described for {\tt dfsig}, {\tt nsig} above.
\item{--}
To scan in $\sigma^2$ with {\tt iscan } $> 1$, 
the initial range in $\sigma^2$ must be provided in {\tt sig1}, {\tt sig2}.
For each step in $l$, the end points of the range are incremented
as determined by {\tt dfsig1}, {\tt nsig1} and {\tt dfsig2}, {\tt nsig2},
as above.

\fig{0.1truecm}

\smallskip
\hrule
\medskip
{
\obeyspaces
\source
osc:
  el, nsel, els1, dels, dfsig1, dfsig2, nsig1, nsig2
    ,8,  30  ,10,   10.,    10.,   1,    1     @
  itrsig, sig1, istsig, inomde, itrds,
    1,    30. ,     ,      ,     ,       @
  dfsig, nsig, iscan, sig2,
      ,   1   ,10,  34     @
eltrw1, eltrw2, sgtrw1, sgtrw2
    ,        ,      ,        ,    @
}
\msni
{\bf Figure 1}. Input block to scan for f mode.
\medskip
\hrule
\smallskip
\gif

\medskip\noindent
Example:
To compute the f mode for degrees $l = 30, 40, \ldots, 100$
by scanning in $\sigma^2$, 
the {\tt osc} group of input shown in Figure 1 may be used.
This will scan in $\sigma^2$ between $\sigma^2 = l$ and $\sigma^2 = l+4$
for each value of $l$, and hence is likely to find the f mode,
with $\sigma^2$ slightly exceeding $l$.

\subsect
{b) Using previously computed frequencies as trials}

The options of having {\tt itrsig} $\ge 2$
and {\tt istsig} $\ge 1$ allow easy computation of modes
of a model, given results on the same modes in a slightly different model. 
The results on the modes can be in the form of binary files 
containing a grand or a short summary 
(see Sections 8.2 and 8.3 below),
or an ASCII file containing degree, order and cyclic frequency (in $\muHz$).
(The possibility of specifying {\tt itrsig} $\le -2$ was
introduced to allow use of older files of summaries in
single precision; it is unlikely to be of general usefulness.)

There are two different options:
\medskip
\item{--} When {\tt itrsig} is even,
the modes are located according to their position on the file,
by specifying the range 
${\tt inomde}, {\tt inomde} + 1, \ldots, {\tt inomde} + {\tt istsig} - 1$ 
of mode numbers in the file;
both $\sigma^2$ and $l$ are taken from the values on the file.

\item{--}
When {\tt itrsig} is odd the
programme searches for modes of a specified degree and with orders
${\tt inomde}, {\tt inomde} + 1, \ldots, {\tt inomde} + {\tt istsig} - 1$.
The degree is either provided in {\tt el} or, by setting {\tt nsel} $> 1$,
by stepping through a sequence of degrees.

\medskip
In both cases the search can be restricted to given ranges
{\tt eltrw1}, {\tt eltrw2} in degree and/or
({\tt sgtrw1}, {\tt sgtrw2}) in $\sigma^2$ by having
{\tt eltrw1} $\le$ {\tt eltrw2} and/or {\tt sgtrw1} $\le$ {\tt sgtrw2}.

\fig{0.1truecm}
\smallskip
\hrule
\medskip
{
\obeyspaces
\source
osc:
  el, nsel, els1, dels, dfsig1, dfsig2, nsig1, nsig2
    ,21,  5  ,5,   0.,    0.,   1,    1     @
  itrsig, sig1, istsig, inomde, itrds,
    3,       ,   9 ,    2 ,     ,       @
  dfsig, nsig, iscan, sig2,
      ,       ,    ,    ,    @
eltrw1, eltrw2, sgtrw1, sgtrw2
    ,        ,   100 ,   1.e6    ,    @
}
\msni
{\bf Figure 2}. Input block to computes modes of given order,
based on trial frequencies on file.
\medskip
\hrule
\smallskip
\gif

The possibility of combining {\tt itrsig} odd with {\tt nsel} $> 1$
allows the computation of modes of a given order for several values of $l$. 
Figure 2 illustrates the {\tt osc} input group for computing
modes of order $2, 4, \ldots 10$, for $l = 0, 5, \ldots , 100$,
and restricting $\sigma^2$ to exceed 100.

The choice of {\tt itrsig} $= 6$ or 7 allows convenient computation
of frequencies on the basis of a set of observed modes.
A possible input format is illustrated in Figure 3.
Note that the file may contain a header, indicated by
the character {\tt \#} in the first column.
The data are read in using free format.
Also, the file may contain additional columns beyond the
degree, order and frequency, in the case illustrated the standard error.
These columns are ignored.

\fig{0.1truecm}
\smallskip
\hrule
\medskip
{
\obeyspaces
\source
\# Data from Libbrecht, K.G., Woodard, M.F., and Kaufman, J.M.,
\# ApJS, vol. 74, p. 1129 (1990).
\#
\#
\# Mode Frequency Table (+/- gives 1-sigma random errors)
\#    l     n   nu(uHz)    +/-  source  
\#    ---   --- --------- ------- ------
     0    12  1823.600   0.600    2
     0    13  1957.300   0.400    2
     0    14  2093.500   0.200    2
     0    15  2228.600   0.100    2
     0    16  2362.500   0.100    2
     0    17  2496.600   0.300    2
     0    18  2629.600   0.300    2
                  .
                  .
                  .
                  .
}
\msni
{\bf Figure 3}. Possible structure of input file for the
cases {\tt itrsig} $= 6$ and 7.
\medskip
\hrule
\smallskip
\gif


\subsect
{\it 7.3.3 Remaining parameters} 

The meaning of most of the parameters controlling the equations and
boundary conditions should be reasonably clear on the basis of the
discussion in Section 2.
The case of a plane-parallel layer ({\tt iplneq} = 1)
is discussed in Section A.2; {\tt iturpr} is used to control 
the treatment of turbulent pressure in a realistic model of the solar
atmosphere, as discussed in Section A.3.

The parameters controlling the integration are essentially discussed
in Section 3, with the exception of the test on the discontinuity
of the eigenfunction. This is based on
the absolute value of the determinant in 
equation (3.5) or (3.9), normalized by the product
of the norms of the columns; thus in the case of equation (3.5) we
introduce 
$$ 
\tilde \Delta \equiv 
{| \Delta | \over 
[ ( y_1^{\rm (i)} (\xf )^2 + y_2^{\rm (i)} (\xf )^2 )
( y_1^{\rm (o)} (\xf )^2 + y_2^{\rm (o)} (\xf )^2 ) ]^{1/2} } \; ,
\eqno(7.1)
$$
and the condition for continuity of the eigenfunction is that
$\tilde \Delta \le {\tt epssol}$.

Similarly the meaning of the parameters determining the output should
be clear or may be inferred from Section 4 (see also Section 8
below describing the results of the computation). Finally, the parameters
controlling the diagnostics are unlikely to be of interest for
general users.

\subsect
{\bf 7.4 The user-supplied routines {\tt modmod} and {\tt spcout}}

If {\tt imdmod} $\not=$ 0 {\tt modmod} is called after the equilibrium model
has been read (and, if applicable, after taking every {\tt in}-th point), as
\ms
{
\source
call modmod(x, aa, data, nn, iaa, imdmod)
}
\msni
where {\tt iaa} is the first dimension of {\tt aa}, and is set by the programme
calling {\tt modmod}; the remaining quantities are defined above. 
It has no further effect on the calculations, but may be used by the
user to carry out modifications to the model. Similarly,
if {\tt ispcpr} $\not=$ 0 the routine {\tt spcout} is called at the end of the
calculation for each mode, as
\ms
{
\source
call spcout(x, y, aa, data, nn, iy, iaa, ispcpr)
}
\msni
to allow the user to produce output in addition to that
otherwise produced by the programme;
here {\tt iy} is the first dimension of {\tt y} and is set by the programme
calling {\tt spcout}. When {\tt imdmod} = 0 (or {\tt ispcpr} = 0), as is the
default, the routines are not called.

The standard set-up of the code 
supplies dummy subroutines, in the file containing the main programme
{\tt main}.
These must obviously be replaced by the user for these options to
take effect.

\mainsect
\centerline{\twelvebf 8 Output from the programme} 

This section contains a description of the principal output
produced by the programme. 
Section 8.1 describes 
what may somewhat archaically be referred to as printed output, for each mode.
The remaining sections gives the format of
the output, likely to be of most use, produced on disk files.
The unit numbers of the the output files are currently
hardcoded into variables in the programme.
The relevant unit numbers, with the (hardcoded) defaults
and description of the output, are:
\medskip
\ref
{\tt idsgsm} (default 11): Grand summary file 
\nwl
(Section 8.2; conventional name {\tt agsm.<model descriptor>}).
\ref
{\tt idsssm} (default 15): Short summary file 
\nwl
(Section 8.3; conventional name {\tt assm.<model descriptor>}).
\ref
{\tt idsefn} (default 4): Eigenfunction file 
\nwl
(Section 8.4; conventional name {\tt amde.<model descriptor>[.z]}).
\ref
{\tt idsrkr} (default 12): Rotational splitting kernel file (Section 8.5)
\ref
{\tt idsgkr} (default 13): $\Gamma_1$ kernel file (Section 8.6)
\ref
{\tt idslog} (default 20): 
%\revision{22/7/94}
Log file of error and warning messages from eigenfrequency iteration.
If {\tt idslog} is not defined in input file,
the default name {\tt adipls-status.log} is used (see Section 8.7).


\subsect
{\bf 8.1 Printed output} 

This is output on unit {\tt istdpr}, which may be modified
in the input file (group {\tt out}).
If {\tt istdpr} $\neq$ {\tt istdou}, where {\tt istdou} is the unit number
for the standard output, a summary of the calculation,
including error and warning messages, is also output to {\tt istdou}.
Note that {\tt istdou} is normally 6.
It is defined in {\tt block data} subprogramme {\tt blstio}.

After some diagnostics relating to the integration follows output
related to the iteration for (or scan in) $\sigma^2$. For each
iteration is given the iteration number, the current value of $\sigma^2$,
and the corresponding values {\tt ddsig} and {\tt ddsol}
of the matching determinant $\Delta$ and
the normalized determinant $\tilde \Delta$ (or, when iterating with
{\tt mdintg} = 3, the
mean change in the solution since the previous value of $\sigma^2$).
For {\tt mdintg} = 2 an estimate of the mode order, defined by equation (4.1),
is also given. After the iteration has converged the final value of
$\sigma^2$ is printed, as well as $\tilde \sigma^2$
({\cf} equation 4.1) and the period calculated from $\sigma^2$. If the
equations are solved in the Cowling approximation the correction 
$\delta \sigma^2$ and the corrected value $\sigma_{\rm c}^2$
obtained from Cowling's perturbation technique are also printed.

Then follows the maximum absolute value $z_{1,\rm max}$ 
of the energy-related eigenfunction $z_1 (x)$ 
({\cf} equation 4.4) and the location $\hat x_{max}$ of the maximum. 
After this comes 
the printout of the eigenfunction, in the form of
$$
n, \  x^n , \  
y_1 ( x^n ) , \ 
y_2 ( x^n ) , \ 
y_3 ( x^n ) , \ 
y_4 ( x^n ) , \ \ \ 
\hat z_1 ( x^n ) , \ 
\hat z_2 ( x^n ) , \ 
$$
for non-radial oscillations, or
$$
n, \  x^n , \  
y_1 ( x^n ) , \ 
y_2 ( x^n ) , \ 
\hat z_1 ( x^n ) , \ 
$$
for radial oscillations. Here $\hat z_1$ and $\hat z_2$
are as defined in equations (4.4) and (4.6).

The labelling of the mode is given next, in the form 
\ms
{\source
This is a p$n$(l = $l$) mode
This is a f(l = $l$) mode
This is a g$|n|$(l = $l$) mode
}
\msni
for the order $n > 0$, $n = 0$ or $n < 0$ respectively.
After this follows the maximum absolute value $y_{1,\rm max}$ of
$y_1 (x)$ and the location $x_{\rm max}$ of the maximum, and the
dimensionless energy $E$ of pulsation ({\cf} equation 4.3).

If {\tt iper} has been set to 1 there next follows output from the
calculation of the variational frequency and the corresponding
frequency. Here {\tt sigv} gives the value of $\sigma^2$ obtained
from the variational expression; in addition the period $\Pi_E$ found
from the value of $\sigma^2$ obtained as an eigenvalue is repeated,
and the period $\Pi_V$ and cyclic frequency $\nu_V$ 
found from the variational $\sigma^2$ are printed.

If the rotational kernel is calculated 
({\cf} equation 4.7), the value of $\beta_{nl}$ is printed and, if
{\tt nprtkr} $>$ 0, the kernel is printed, as
$$
n , \  x^n , \  K_{nl} ( x^n ) \; .
$$

\subsect
{\bf 8.2 The grand summary} 

A fairly complete summary of
the calculation is written to unit {\tt idsgsm},
without format; this only takes
place if the eigenvalue iteration has converged. 
The summary consists of 38 real variables {\tt cs(1:38)}
and 8 integer variables {\tt ics(1:8)}.
These are defined as follows
({\tt typewrite-type names} refer to the list of input parameters,
{\cf} Section 7.2.3):
\medskip
{\obeylines\smallskip
{\tt cs(1)}: {\tt xmod}
{\tt cs(2:8)}: $D_1$ - $D_7$:
\qquad {\tt cs(2)}: $M$
\qquad {\tt cs(3)}: $R$
\qquad {\tt cs(4)}: $p_{\rm c}$
\qquad {\tt cs(5)}: $\rho_{\rm c}$
\qquad {\tt cs(6)}: $- (1 /\Gamma_1 p) (\dd^2 p / \dd x^2) $ at centre
\qquad {\tt cs(7)}: $- (1 /\rho) (\dd^2 \rho / \dd x^2) $ at centre
\qquad {\tt cs(8)}: $\mu$
\qquad {\tt cs(9)}: $D_8$ (flag for version of {\tt amdl file}; see Section 5).
{\tt cs(10)}: $A_2 (\xs )$
{\tt cs(11)}: $A_5 (\xs )$
{\tt cs(12)}: $x_1$
{\tt cs(13)}: $\sigma_\Omega^2$ ({\cf} Eq. A.60)
{\tt cs(14)}: $\xf$
{\tt cs(15:16)}: {\tt fctsbc}, {\tt fcttbc}
{\tt cs(17)}: $\lambda$ (fudge factor in Poisson's equation)
{\tt cs(18)}: $l$ (degree)
{\tt cs(19)}: $n$ (order)
{\tt cs(20)}: $\sigma^2$ 
{\tt cs(21)}: $\sigma_{\rm c}^2$
{\tt cs(22:23)}: $y_{1,\rm max}$, $x_{\rm max}$
{\tt cs(24)}: $E$ (dimensionless energy)
{\tt cs(25:27)}: $\Pi_E$, $\Pi_V$, $\nu_V$ (periods in minutes, $\nu_V$ in mHz).
{\tt cs(28:29)}: {\tt ddsig}, {\tt ddsol}
{\tt cs(30:33)}: $y_1 (\xs ) - y_4 (\xs )$
{\tt cs(34:35)}: $z_{1,\rm max}$, $\hat x_{\rm max}$
{\tt cs(36)}: $\beta_{nl}$
{\tt cs(37)}: $\nu_{\rm Ri}$, {\ie}, cyclic frequency from %
Richardson extrapolation, (in mHz).
{\tt cs(38)}: $m$ (azimuthal order) for {\tt irotsl = 1} (added 3/11/02).
}
\bigskip
{\obeylines
{\tt ics(1)}: {\tt in}
{\tt ics(2)}: {\tt nn}
{\tt ics(3)}: {\tt mdintg + 10*iriche}
{\tt ics(4)}: {\tt ivarf}
{\tt ics(5)}: {\tt icase}
{\tt ics(6)}: {\tt iorign}
{\tt ics(7)}: {\tt iekinr}
{\tt ics(8)}: unused.
}
\medskip\noindent
Notes:

\medskip
\item{--}
Depending on the parameters of the calculation, some variables may not be set.
All unset variables are initialized to 0. 
In particular {\tt cs(9)}, {\tt cs(13)},
{\tt cs(38)} and {\tt ics(8)} are currently unused.

\item{--}
{\tt ddsig} and {\tt ddsol} [{\ie}, {\tt cs(28)} and {\tt cs(29)}]
are set to the values of $\Delta$ and $\tilde \Delta$ in
the last iteration, and thus provide a measure of the extent to which
the iteration has converged.

\item{--}
When Cowling approximation is used $\Pi_E$ (in {\tt cs(25)}
is set to the period corresponding to the {\it uncorrected}
eigenfrequency.

\item{--}
{\tt icase} is a compressed case number for the calculation. It is defined as
\ms
{\source
icase = icow1 + 10 iper + 100 irotsl + 1000 ispec + 10 000 istsbc 
\qquad + 100 000 iplneq + 1 000 000 iturpr
}
\msni
\item{}
Here {\tt icow1} is defined as follows:
\itemitem{-}
{\tt icow1} = is 0 when the full set of equations is used.
\itemitem{-}
{\tt icow1} = 1 when the Cowling approximation is used, and the Richardson
extrapolated frequency is based on the corrected eigenfrequency
$\sigma_{\rm c}$.
\itemitem{-}
{\tt icow1} = 2 when the Cowling approximation is used, and the Richardson
extrapolated frequency is based on the uncorrected eigenfrequency
({\ie}, computed with the input parameter {\tt icow} = 3).

\ms
\item{}
Furthermore
{\tt ispec} = 1 if $\lambda \not= 1$ 
or {\tt fctsbc} $\not=$ 1, and the remaining parameters are as defined in
Section 7.2.3. Note that {\tt irotsl} to flag for rotational effects
was added 3/11/02.

\item{--}
{\tt iorign} is always set to 3 in the pulsation programme (it may
have other values for grand summaries set on the basis of other, more 
restricted sources of information). 

\medskip
{\tt ics} is stored in the array {\tt cs(1:50)},
by the following equivalencing
\ms
{\source
equivalence(ics(1), cs(39))
}
\msni
Thus if reals and integers have the same length only the
first 46 positions in {\tt cs} are used,
whereas (as is typically the case) if 8-byte real variables and
4-byte integers are used, only the first 42 positions of
{\tt cs} are used%
\footnote*{This is a historical consequence of a previous possibility
of storing a model name in the last part of {\tt cs};
in future, an extension or generalization of the storage might
be contemplated.}.
In the programme {\tt cs} is stored in {\tt common/csumma/}.

The summary is output in a single record for each mode, by the statement
\ms
{\source
write(idsgsm) (cs(i), i=1,50)
}
\msni

Various programmes are available to scan or manipulate the
contents of the grand summary
(see {\it Notes on using the solar models and adiabatic pulsations package}).
In particular, the programme {\tt scan-agsm.d} scans the grand summary,
whereas {\tt set-obs.d} outputs a file giving degree, order,
cyclic frequency and possibly mode energy.

\subsect
{\bf 8.3 The short summary} 

To save disk space, a condensed summary has been introduced.
This contains the most essential information about the modes,
without the details that are rarely used.
Thus it is adequate for most purposes.
In addition data on the model are given as special records,
rather than being repeated in each record.

The data are stored in the real array {\tt ss(1:5)} and the integer
array {\tt iss(1:2)}.

The type of the record is flagged by {\tt ss(1)}:
\medskip\ref
{\it Model record} :
This is flagged by {\tt ss(1)} $<$ 0. {\tt ss(2)} is set to {\tt xmod}.
The remaining variables are defined by
\medskip
{\obeylines\smallskip
{\tt ss(3)}: $M$
{\tt ss(4)}: $R$
{\tt ss(5)}: $p_{\rm c}$
{\tt ss(6)}: $\rho_{\rm c}$}

\medskip\ref
{\it Oscillation record}:
This is flagged by {\tt ss(1)} $\ge$ 0. Here
\medskip
{\obeylines
{\tt ss(1)}: $l$ (degree)
{\tt ss(2)}: $n$ (order)
{\tt ss(3)}: $\sigma^2$ 
{\tt ss(4)}: $E$ (dimensionless energy)
{\tt ss(5)}: $\nu_V$ (in mHz)
\medskip
{\tt iss(1)}: {\tt icase}
{\tt iss(2)}: {\tt iorign}}
\medskip


Note that in the Cowling approximation
$\sigma^2$ is taken to be the corrected value $\sigma_{\rm c}^2$
obtained with the Cowling perturbation expression.
Also {\tt ss(5)} is the {\it variational} frequency if this has
been calculated; otherwise the frequency obtained as an eigenvalue
is used.
{\tt iss(1:2)} are stored in {\tt ss(1:7)},
by the following equivalence statement
\ms
{\source
equivalence (ss(6), iss(1))
}
\msni
The summary is written in a single record, with the statement
\ms
{\source
write(idsssm) (ss(i), i=1,7)
}


\subsect
{\bf 8.4 Output of eigenfunction to file} 

%\revision{(ca. 1991; prior to this only {\tt nfmode = 1} was implemented)}
If {\tt nfmode} = 1, 2 or 3 and the eigenfunction iteration has converged
the eigenfunctions are written on unit 4, as a single
record, in binary form.
The format depends on the value of {\tt nfmode}:

\subsect
a) {\tt nfmode = 1}, full set of eigenfunctions.

Here the output is done with the statement 

      {\tt write(idsefn) (cs(i), i=1,50), nnw, (x(n), (y(i,n), i=1,6), n=nw1,nn)}
\smallskip\noindent
Here {\tt nnw = nn - nw1 + 1},
{\tt y(1:4,n)} contains the eigenfunctions $y_1 (x^n ) \ -\ y_4 (x^n )$
({\cf} Section 6), and 
$$
{\tt y(5,n)} = \hat z_1 (x^n ) , \quad
{\tt y(6,n)} = \hat z_2 (x^n ) \; 
$$
[{\cf} equations (4.4) and (4.6)].
An eigenfunction file of this format conventionally has
name {\tt amde.<model descriptor>}.

\subsect
b) {\tt nfmode = 2}, displacement eigenfunctions.

Here the first record contains the number of mesh points and the 
mesh, written as

      {\tt write(idsefn) nnw, (x(n), n=nw1,nn)}
\smallskip\noindent
and the subsequent records contain the mode data, in the form

      {\tt write(idsefn) (cs(i), i=1,50), ((y(i,n), i=1,2), n=nw1,nn)}

\subsect
c) {\tt nfmode = 3}, density-weighted displacement eigenfunctions.

Here the first record contains the number of mesh points and the 
mesh, written as

      {\tt write(idsefn) nnw, (x(n), n=nw1,nn)}
\smallskip\noindent
and the subsequent records contain the mode data, in the form

      {\tt write(idsefn) (cs(i), i=1,50),((z(i,n), i=1,2), n=nw1,nn)}
\smallskip\noindent
where
$$
{\tt z(1,n)} = \hat z_1 (x^n ) , \quad
{\tt z(2,n)} = \hat z_2 (x^n ) 
$$
[{\cf} equations (4.4) and (4.6)].
An eigenfunction file of this format conventionally has
name {\tt amde.<model descriptor>.z}.
\medskip

Thus in all cases the eigenfunction record contains 
the grand summary ({\cf} Section 8.3).
Format a) is clearly the most space-consuming.
Format c) produces eigenfunctions in a form suitable for
setting up kernels.
In particular, to compute rotational kernels no further information
about the equilibrium model is required.
This is therefore most likely the format of choice, unless
the full set is explicitly needed.

\subsect
{\bf 8.5 Output of rotational kernel to file ({\cf} equation 4.7)}

If {\tt irotkr} = 1 
the rotational kernel is written on unit 12, as a single
record, without format. This is done with the statement 

      {\tt write(idsrkr) (cs(i), i=1,50),nnw, (x(n), akr(n), n=nw1,nn)}
\medskip\noindent
Here 
$$
{\tt akr(n)} = K_{nl} (x^n ) , 
$$
Thus the record contains the grand summary ({\cf} Section 8.3).

\subsect
{\bf 8.6 Output of $\Gamma_1$ kernel to file ({\cf} equation 4.8)}

If {\tt igm1kr = 1} 
the $\Gamma_1$ kernel is written on unit 13, as a single
record, without format. This is done with the statement 

      {\tt write(idsgkr) (cs(i), i=1,50), nnw, (x(n), akrgm1(n), n=nw1,nn)}
\medskip\noindent
Here 
$$
{\tt akrgm1(n)} = K_{nl}^{(\Gamma_1 )} (x^n ) , 
$$
Thus the record contains the grand summary ({\cf} Section 8.3).

\subsect
{\bf 8.7 Iteration status log}

%\revision{22/7/94}
A log of problems with the eigenfrequency calculation is
output to unit {\tt idslog};
if no file name is provided, the default is {\tt adipls-status.log}.
This may contain both error and warning messages:
\medskip
\item{---} An error message is printed if
\itemitem{-} 
The iteration failed to converge, even after possible
attempts of adjusting $\xf$ ({\cf} Section 3.5).
\itemitem{-} 
Significantly different eigenfunctions were found in Richardson-extrapolation 
calculation ({\cf} Section~3.4).
\item{}
Note that in cases of errors, no output is made to 
the files with grand summaries, eigenfunctions, {\etc}
\medskip
\item{---} A warning message is printed if
\itemitem{-} 
The location of $\xf$ had to be changed to obtain convergence or the
correct order.
\itemitem{-} 
Different mode orders were found in Richardson-extrapolation calculation,
but the eigenfunctions were deemed to be sufficiently similar.
\item{}
In cases of warnings, normal output is made to 
the files with grand summaries, eigenfunctions, {\etc}
Indeed, in these cases one can generally assume that the
calculation has been successful.

\mainsect
\centerline{\twelvebf 9 The main programme} 

The calculation is controlled by the call of the subroutine {\tt adipls}. 
A small main programme is needed to set up storage for the calculation.
This has the form
\ms
{\source
\obeyspaces
\     program runadi
c
c  main programme for adiabatic pulsations
c
c  quantities set in parameter statement:
c  nnmax: maximum number of mesh points in integration
c
c  Note: as presently set up, nnmax is set also in s/r nrkint and
c  eigin4
c
c  Double precision version.
c  +++++++++++++++++++++++++
c
c  Dated: 10/3/90
c
\      implicit double precision (a-h, o-z)
\      parameter (nnmax = 10000)
\      parameter (nnmax1 = nnmax+1, nnmax2 = nnmax+10)
\      common/rhsdat/ dt1(20), aa(6,nnmax) /xarr/ x(nnmax)
\     *  /xarr1/ x1(nnmax1) /xarr2/ x2(nnmax1)
\     *  /worksp/ aa1(9,nnmax)   /yyyyyy/ y(8,nnmax)  
\     *  /yyyyri/ yri(4,nnmax)
\     *  /sysord/ sysyso(4)
\     *  /work/ wwnrk(20,nnmax2)
\      common/wrkleq/ wwwlll(1500)
\      common/cderst/ derc(6,nnmax)  /cintst/ aintc(6,nnmax)
c
c  common defining standard input and output
c
\      common/cstdio/ istdin, istdou, istdpr
c
\      write(istdou,100) nnmax
c
\      call adipls
\      stop
\        100 format(//61('*')//
\     * ' In this version, the maximum number of mesh points is', i5//
\     * 61('*'))
\      end
}
\msni
Here {\tt nnmax} is the maximum number of mesh points 
{\it used in the calculation},
and {\tt ii} is the order of the system. Furthermore {\tt common /worksp/}
should be large enough to contain the array {\tt aain(6,nnmodl)},
where {\tt nnmodl} is the total number of mesh points in the 
equilibrium model read in (if {\tt in} is not equal to 1, {\tt nnmodl} is
larger than {\tt nn}).
%Note that I have attempted to assign all storage
%that depends on the number of mesh points in the common blocks 
%in the main programme. However, it is still possible that there may
%additional storage assigned elsewhere. This needs further checking.

The work space set in {\tt common/work/} is only needed
for integration with the relaxation technique, {\ie},
for {\tt mdintg} = 3.
If {\tt mdintg} = 1 or 2 the line setting up this work area may be removed.

\mainsect
\centerline{\twelvebf References} 

\ref
Aerts, C., Christensen-Dalsgaard, J. \& Kurtz, D. W., 2010.
{\it Asteroseismology},
Springer, Heidelberg.
\ref
Baker, N. H., Moore, D. W. \& Spiegel, E. A., 1971.
[Aperiodic behaviour of a non-linear oscillator].
{\it Q. J. Mech. Appl. Math.,} 
{\bf 24}, 391 -- 422.
\ref
Cash, J. R. \& Moore, D. R., 1980.
[A high order method for the numerical solution of two-point boundary
value problems].
{\it BIT}, {\bf 20}, 44 -- 52.
\ref
Christensen-Dalsgaard, J., 1980.
[On adiabatic non-radial oscillations with moderate or large $l$].
{\it Mon. Not. R. astr. Soc.},
{\bf 190}, 765 -- 791.
\ref
Christensen-Dalsgaard, J., 1981.
[The effect of non-adiabaticity on avoided
crossings of non-radial stellar oscillations].
{\it Mon. Not. R. astr. Soc.}, 
{\bf 194}, 229 -- 250 (CD81).
\ref
Christensen-Dalsgaard, J., 1982.
[On solar models and their periods of oscillation].
{\it Mon. Not. R. astr. Soc.},
{\bf 199}, 735 -- 761 (CD82).
\ref
Christensen-Dalsgaard, J., 2008.
[ADIPLS -- the Aarhus adiabatic pulsation package].
{\it Astrophys. Space Sci.},  {\bf 316}, 113 -- 120 (CD08).
\ref
Christensen-Dalsgaard, J. \& Frandsen, S., 1983.
[Radiative transfer and solar oscillations].
{\it Solar Phys.},
{\bf 82}, 165 -- 204.
\ref
Christensen-Dalsgaard, J. \& Mullan, D. J., 1994.
[Accurate frequencies of polytropic models].
{\it Mon. Not. R. astr. Soc.}, {\bf 270}, 921 -- 935.
\ref
Christensen-Dalsgaard, J., Dilke, F. W. W. \& Gough, D. O., 1974.
[The stability of a solar model to non-radial oscillations].
{\it Mon. Not. R. astr. Soc.},
{\bf 169}, 429 -- 445.
\ref
Cowling, T. G., 1941.
[The non-radial oscillations of polytropic stars].
{\it Mon. Not. R. astr. Soc.}, 
{\bf 101}, 367 -- 375.
\ref
Gabriel, M. \& Noels, A., 1976. 
[Stability of a $30\,M_{\odot}$ star towards $g^{+}$ modes of
high spherical harmonic values].
{\it Astr. Astrophys.}, 
{\bf 53}, 149 -- 157.
\ref
Gabriel, M. \& Scuflaire, R., 1979.
[Properties of non-radial stellar oscillations].
{\it Acta Astron.}, 
{\bf 29}, 135 -- 149.
\ref
Lee, U., 1985.
[Stability of the Delta Scuti stars against nonradial oscillations
with low degree $l$].
{\it Publ. Astron. Soc. Japan}, {\bf 37}, 279 -- 291.
\ref
Mihalas, B. W. \& Toomre, J., 1981.
[Internal gravity waves in the solar atmosphere. I.
Adiabatic waves in the chromosphere].
{\it Astrophys. J.}, {\bf 249}, 349 -- 371.
\ref
Scuflaire, R., 1974.
[The non radial oscillations of condensed polytropes].
{\it Astr. Astrophys.}, 
{\bf 36}, 107 -- 111.
\ref
Rosenthal, C. S., Christensen-Dalsgaard, J., Houdek, G, Monteiro, M.J.P.F.G.,
Nordlund, {\AA}. \& Trampedach, R., 1995.
[Seismology of the solar surface regions].
In: {\it Proc. Fourth SOHO Workshop: Helioseismology},
eds Hoeksema, J. T., Domingo, V., Fleck, B \& Battrick, B., 
ESA SP-376, vol. 2, ESTEC, Noordwijk, p. 459 -- 464.
\ref
Rosenthal, C. S., Christensen-Dalsgaard, J.  Nordlund, {\AA}.,
Stein, R. F. \& Trampedach, R., 1999.
[Convective contributions to the frequencies of solar oscillations].
{\it Astron. Astrophys.}, {\bf 351}, 689 -- 700.
\ref
Shibahashi, H. \& Osaki, Y., 1981.
[Theoretical eigenfrequencies of solar oscillations of low harmonic
degree $\ell$ in five-minute range].
{\it Publ. Astron. Soc. Japan}, 
{\bf 33}, 713 -- 719.
\ref
Soufi, F., Goupil, M. J. \& Dziembowski, W. A., 1998.
[Effects of moderate rotation on stellar pulsation.
I. Third order perturbation formalism].
{\it Astron. Astrophys.}, {\bf 334}, 911 -- 924.
\ref
Takata, M., 2005.
[Momentum conservation and model classification of the dipolar
oscillations in stars].
{\it Publ. Astron. Soc. Japan}, {\bf 57}, 375 -- 389.
\ref
Takata, M., 2006a.
[Analysis of adiabatic dipolar oscillations of stars].
{\it Publ. Astron. Soc. Japan}, {\bf 58}, 893 -- 908.
\ref
Takata, M., 2006b.
[Rigorous analysis of dipolar oscillations of stars].
In {\it Proc. SOHO 18 / GONG 2006 / HELAS I Conf.
Beyond the spherical Sun},
ed. K. Fletcher, ESA SP-624, ESA Publications Division,
Noordwijk, The Netherlands.
\ref
Unno, W., Osaki, Y., Ando, H. \& Shibahashi, H., 1989.
{\it Nonradial Oscillations of Stars} (2. ed.), University of Tokyo Press.

\newpage
\mainsect
\centerline{\twelvebf Appendix A. Equations} 

In this appendix the equations of non-radial oscillations are given in
the form they are solved in the programme. Section~A.1 presents the equations
in the standard case of oscillations of a spherical star with no turbulent
pressure,
Section~A.2 discusses the implementation used for
oscillations of a plane-parallel layer,
and Section~A.3 discusses the ways turbulent pressure may be treated 
(or neglected).

\subsect
{\bf A.1 Equations in the standard case} 

For non-radial oscillations the equations satisfied by the $y_i$ 
are
$$
x  {\dd y_1  \over \dd x } = (V_g - 2 )  y_1 +
\left( 1 - {V_g  \over \eta }\right)  y_2 - V_g  y_3 \; ,
\eqno (A.1)$$
$$
x {\dd y_2  \over \dd x } = [ l ( l + 1) - \eta A ]  y_1 
+ (A - 1 )  y_2 + \eta A  y_3 \; ,
\eqno (A.2)$$
$$
x  {\dd y_3  \over \dd x } = y_3 + y_4 \; ,
\eqno (A.3)$$
and
$$
x  {\dd y_4  \over \dd x } = - \lambda A U  y_1  - 
\lambda U {V_g   \over \eta } y_2 
\eqno (A.4)$$
$$
+ [ l ( l + 1) + U(A - 2 )
+ (1 - \lambda ) U V_g ]  y_3 + 2(1 - U )  y_4 \; .
$$
Here $\eta = l ( l + 1) g / ( \omega^2 r )$, and the notation is otherwise
as defined in equation (5.1). Note that the modified form of Poisson's
equation, equation (2.3), has been used. In the Cowling approximation the
terms in $y_3$ are neglected in equations (A.1) and (A.2), and 
equations (A.3) and (A.4) are not used.

For radial oscillations the equations are
$$
x  {\dd y_1  \over \dd x } = (V_g - 2 )  y_1 -
V_g {\sigma^2 x^2   \over q } y_2 \; ,
\eqno (A.5)$$
and
$$
x {\dd y_2  \over \dd x } = \left[ x - {q \over \sigma^2 x^2 }
(A - \lambda U) \right] y_1 + A  y_2 \; .
\eqno (A.6)$$

\subsect
{\bf A.2 Equilibrium variables and oscillation equations in 
plane-parallel case }

The option has been built into the programme to study 
{\it nonradial} oscillations of a plane-parallel envelope model
with constant gravity.
This option is invoked by setting the parameter {\tt iplneq} to 1.
The oscillations are treated in the Cowling approximation.
For simplicity the model and the oscillations are still put on dimensionless
form in terms of a total mass $M$ and a radius $R$, which, however, clearly
has no specific physical meaning (except that the results may be
taken as approximations to envelope oscillations of a star with the
given mass and radius). The gravity is then given by $g = G M/R^2 $.
The horizontal wave number $k_h$ is parameterized in terms of the
``degree'' $l$, defined such that 
$$
k_h = {l \over R } \; .
\eqno (A.7)$$

The equilibrium model is still given by $x$ and $A_1 - A_5$.
Now, however, $x$ is a measure of position, in units of $R$, decreasing
towards the interior of the model; the point $x = 1$ may still conveniently
be chosen at or close to the surface of the model.
The variables $A_1 \ -\ A_5$ are defined as
$$
\eqalign{
& A_1 \equiv 1 \; , \cr
& A_2 = V_g  \equiv {g \rho R   \over \Gamma_1 p} \; , \cr
& A_3 \equiv \Gamma_1 \; , \cr
& A_4 = A \equiv - V_g - {\dd \ln \rho  \over \dd x}  \; , \cr
& A_5 = U \equiv {4 \pi \rho R^3  \over M } \; . \cr
}
\eqno (A.8)$$

The oscillation equations are now
$$
{\dd y_1  \over \dd x } 
= V_g y_1 + \left( 1 - {V_g  \over \eta }\right) y_2 \; ,
\eqno (A.9)$$
and
$$
{\dd y_2  \over \dd x } = ( l^2 - \eta A ) y_1 + A y_2 \; ,
\eqno (A.10)$$
where $\eta = l^2 / \sigma^2$. Notice that $A_1$ and $A_5$
are not needed in the oscillation equations. However, they should still
be defined as in equations (A.8), as they are used elsewhere in the
calculations in the programme,  {\eg}
in the evaluation of the energy in equation (4.3).

The calculation of radial oscillations in a plane-parallel envelope
has not been implemented.

\subsect
{\bf A.3 The treatment of turbulent pressure} 

There is no doubt that turbulence contributes significantly
to the hydrostatic balance in the outermost layers of
stars with convective envelopes.
This effect is often described in terms of a turbulent pressure $\pturb$;
in solar models estimates of $\pturb$ indicate that it may
be as high as 10 -- 15 per cent of the total pressure
near the top of the convection zone.
Even so, turbulent pressure is often neglected in the
calculation of equilibrium models.

If turbulent pressure is included in the equilibrium model,
the oscillation calculations require an assumption about
the response of $\pturb$ to the pulsations.
This is sometimes stated by saying that ``the perturbation
to turbulent pressure is neglected''.
However, the effects on the frequencies might depend
sensitively on the way in which the $\pturb$-perturbation
is neglected, often determined by the precise form of
the oscillation equations used for the numerical solution.

This programme allows several options for the treatment
of turbulent pressure:
\medskip
\item{i)}
Neglect of the Eulerian perturbation in the turbulent body force.
\item{ii)} 
Neglect of the Eulerian perturbation in the turbulent pressure.
\item{iii)} 
Neglect of the Lagrangian perturbation in the turbulent pressure.
\medskip
In all three cases, the equilibrium model, defined by the
$\{A_i\}$, must be set up appropriately.
In addition, the first two cases must also be flagged by
the input parameter {\tt iturpr}.
The detailed use of these options is discussed below.

The first option, i), is the original option in the code;
it was introduced largely because of computational
convenience, but follows the treatment used by Mihalas \& Toomre (1981).
It was argued by Christensen-Dalsgaard \& Frandsen (1983) that neglect
of the {\it Lagrangian} perturbation of the turbulent body force
might be somewhat more reasonable, although in no ways satisfactory.
This has not been implemented; however, option iii) shares some of the
same advantages.
Option ii) is retained in the code to allow tests of
the sensitivity of the computed frequencies to different assumptions.

The frequency effects of turbulent pressure were discussed in more detail by
Rosenthal {\etal} (1995) who found that the effect on the
equilibrium structure of turbulent pressure,
and the perturbation in turbulent pressure,
is of the same order of magnitude as other uncertain aspects
of the mode physics, including nonadiabaticity, and 
also of roughly the magnitude and shape of the current dominant
component of the difference between observed frequencies
and frequencies of adiabatic oscillation for normal solar models.
Thus, although turbulent pressure affects a region of the
Sun suffering from other uncertainties in the treatment of
the oscillations, the way it is included or ignored is
of considerable interest.

The implementation of the different options has not been tested
with great care yet, and hence some caution is required when using them.
Also, it should be noticed that computation of variational
frequencies has not been consistently corrected for turbulent pressure.
As discussed below, this should have no effect for option iii)
but is likely to influence options i) and ii).

\subsect
{\it A.3.1 Neglect of the Eulerian perturbation in the turbulent body force
({\tt iturpr = 1})}

We write the equations of motion as
$$
\rho {\DD  {\bf u}  \over \DD t } = - \nabla p + \rho {\bf g} +
\rho {\bf f}_{\rm turb} \; ,
\eqno (A.11)$$
where ${\bf g} = \nabla \Phi $ is the gravitational acceleration, 
$\Phi$ being the gravitational potential, and ${\bf f}_{\rm turb}$
is the turbulent body force; furthermore $p$ is the combined gas and
radiation pressure.
Thus the equation of hydrostatic support, when expressed in
terms of $p$, is
$$
{\dd p \over \dd r } = - \tilde g \rho \; .
\eqno (A.12)$$
where $\tilde g = g - f_{\rm turb}$, where $-  g$ and $f_{\rm turb}$ 
are the radial components of the equilibrium gravity and turbulent force.
Thus here and in the following
$$
g = {G m \over r^2} \; .
\eqno(A.13)
$$
In the oscillation calculation the assumption is now that
the Eulerian perturbation of ${\bf f}_{\rm turb}$ can be neglected.
Hence the Eulerian perturbation of equation (A.11) may be written
$$
\rho {\partial^2 {\bf \delta r}  \over \partial t^2 } = - \nabla p^{\prime}
- \rho^{\prime} \tilde g {\bf a}_r + \rho \nabla \Phi ' \; .
\eqno (A.14)$$
Evidently, the condition of adiabaticity is expressed as
a relation between the Lagrangian perturbations in $p$ and $\rho$:
$$
{\delta p \over p} = \Gamma_1 {\delta \rho \over \rho} \; ,
\eqno(A.15)
$$
where $\Gamma_1$ is the (usual) thermodynamic adiabatic exponent.

To take the distinction between $g$ and $\tilde g$ into account, we
modify the definitions of the equilibrium variables $A_1 , A_2 $ 
and $A_4$. Thus, instead of equations (5.1), we use
$$
\eqalign{
& A_1 \equiv {R^3  \over M } {\tilde g  \over r }
\equiv {\tilde q  \over x^3} \; ,
\qquad \hbox{\rm which defines} \quad  \tilde q 
= {\tilde g  \over g } q \; , \cr
& A_2 = \tilde V_g \equiv - {1 \over \Gamma_1 } {\dd \ln p \over \dd \ln r} 
 \equiv {r \rho \tilde g  \over \Gamma_1 p} \; , \cr
& A_4 = {1 \over \Gamma_1} {\dd \ln p  \over \dd \ln r} -
{\dd \ln \rho  \over \dd \ln r} \; ,
}
\eqno (A.16)
$$
together with $A_3$ and $A_5$ which are unchanged.
We also introduce
$$
V_g = {r \rho g  \over \Gamma_1 p} \; ,\qquad  
A = - V_g - {\dd \ln \rho  \over \dd \ln r} \; .
\eqno (A.17)$$
Clearly in the absence of turbulent pressure these variables reduce
to the variables defined in Section~5.

Note that with these definitions the {\it structure} of the
model file is unchanged.
In the programme it is assumed that $\tilde g/g$ differs from
unity only very near the surface.
Since in this region $q = m/M$ is one with very high precision,
$\tilde q = x^3 \tilde A_1$ can be taken as a very good approximation
to $\tilde g/g$, allowing, {\eg}, the evaluation of $V_g$ and $A$.

The dimensionless perturbation variables $y_1 - y_4$ are defined
as in Section~2.1; in particular the actual gravitational acceleration
$g$ is used in the definition of $y_3$ ({\cf} equation 2.5a).

With these definitions,
the equations satisfied by the $y_i$ are, for non-radial oscillations 
$$
x {\dd y_1  \over \dd x } = (\tilde V_g - 2 )  y_1 +
\left( 1 - {\tilde V_g  \over \tilde \eta} \right) y_2 
- V_g y_3 \; ,
\eqno (A.18)$$
$$
x {\dd y_2  \over \dd x } = [ l ( l + 1) - \tilde \eta \tilde A ] y_1 
+ ( \tilde A - 1 ) y_2 + \eta \tilde A y_3 \; ,
\eqno (A.19)$$
$$
x {\dd y_3  \over \dd x } = y_3 + y_4 \; ,
\eqno (A.20)$$
and
$$
\eqalign{
x {\dd y_4  \over \dd x } = & - \lambda \tilde A U y_1 -
\lambda U {\tilde V_g   \over \tilde \eta} y_2  \cr
& + [ l ( l + 1) + U(A - 2 )
+ (1 - \lambda ) U V_g ] y_3 + 2(1 - U )  y_4 \; . \cr
}
\eqno (A.21)
$$
Here $\tilde \eta = l ( l + 1) \tilde g / ( \omega^2 r )$.

For radial oscillations the equations are
$$
x  {\dd y_1  \over \dd x } = (\tilde V_g - 2 ) y_1 -
\tilde V_g {\sigma^2 x^2   \over \tilde q} y_2 \; ,
\eqno (A.22)$$
and
$$
x  {\dd y_2  \over \dd x } = \left[ x - {1 \over \sigma^2 x^2 }
( \tilde q \tilde A - \lambda q U) \right] y_1 + \tilde A y_2 \; .
\eqno (A.23)$$

To use this option, the equilibrium model must obviously
be set up correctly, as defined in equations (A.16).
In addition, the parameter {\tt iturpr} must be set to 1.

\subsect
{\it A.3.2 Neglect of the Eulerian perturbation in the turbulent pressure
({\tt iturpr = 2})}

Here we assume that the pressure can be written as
$$
p = \pg + \pturb \; ,
\eqno(A.24)
$$
where $\pg$ is the thermodynamic pressure and $\pturb$
is the turbulent pressure.
Hence,
the equations of motion are written as
$$
\rho {\DD {\bf u}  \over \DD t } = - \nabla p + \rho {\bf g} \; ,
\eqno (A.25)$$
with no explicit turbulent body force.
In the equilibrium model, $p$ is assumed to be in hydrostatic
equilibrium, so that the equation of hydrostatic support is
$$
{\dd p \over \dd r } = - g \rho \; .
\eqno (A.26)$$

In the oscillation calculation the assumption is now that
the Eulerian perturbation of $\pturb$ can be neglected.
Hence the Eulerian perturbation of equation (A.25) may be written
$$
\rho {\partial^2 {\bf \delta r}  \over \partial t^2 } = - \nabla \pg^{\prime}
- \rho^{\prime} g {\bf a}_r + \rho \nabla \Phi ' \; .
\eqno (A.27)$$
The condition of adiabaticity is assumed to relate the Lagrangian
perturbation of the gas pressure to the density perturbation,
$$
{\delta \pg \over \pg} = \Gamma_1 {\delta \rho \over \rho} \; ,
\eqno(A.28)
$$
where again $\Gamma_1$ is the thermodynamic adiabatic exponent.

In the oscillation equations, derived under these assumptions,
we need to distinguish between the gradients in $\pg$ and $p$.
Thus we introduce
$$
V_g^{(0)} = - {1 \over \Gamma_1} {\dd \ln \pg \over \dd \ln r} \; , 
\qquad
V_g^{(1)} = - {1 \over \Gamma_1} {\dd \ln p \over \dd \ln r} 
= {\rho g r \over \Gamma_1 p} \; ,  
\eqno(A.29) 
$$
and
$$
A^{(0)} = - V_g^{(0)} - {\dd \ln \rho \over \dd \ln r} \; , \qquad
A^{(1)} = - V_g^{(1)} - {\dd \ln \rho \over \dd \ln r} \; .
\eqno(A.30)
$$
This requires an extension of the model format defined in
Section~5.
Specifically, we replace the definitions of $A_2$ and $A_4$,
and add a new variable $A_6$, in equation (5.1):
$$
\eqalign{
& A_2 = V_g^{(1)} \; , \cr
& A_4 = A^{(1)} \; , \cr
& A_6 = V_g^{(0)} - V_g^{(1)}  = - (A^{(0)} - A^{(1)}) \; , \cr
}
\eqno (A.31)
$$
the remaining definitions being unchanged.
To flag for a model file of this format,
the otherwise unused variable {\tt data(8)} should be set to 10.

The dimensionless perturbation variables $y_1 - y_4$ are defined
as in Section~2.1.

With these definitions,
the equations satisfied by the $y_i$ are, for non-radial oscillations 
(note that for the time being I have not considered the option
of $\lambda \neq 1$ in this case)
$$
x {\dd y_1  \over \dd x } = (V_g^{(0)} - 2 )  y_1 +
\left( 1 - {V_g^{(1)}  \over \eta} \right) y_2 
- V_g^{(1)} y_3 \; ,
\eqno (A.32)$$
$$
x {\dd y_2  \over \dd x } = [ l ( l + 1) - \eta A^{(0)} ] y_1 
+ ( A^{(1)} - 1 ) y_2 + \eta A^{(1)} y_3 \; ,
\eqno (A.33)$$
$$
x {\dd y_3  \over \dd x } = y_3 + y_4 \; ,
\eqno (A.34)$$
and
$$
\eqalign{
x {\dd y_4  \over \dd x } = & - A^{(0)} U y_1 -
U {V_g^{(1)} \over \eta} y_2  \cr
& + [ l ( l + 1) + U(A^{(1)} - 2 )] y_3 + 2(1 - U )  y_4 \; . \cr
}
\eqno (A.35)
$$

For radial oscillations the equations are
$$
x  {\dd y_1  \over \dd x } = (V_g^{(0)} - 2 ) y_1 -
V_g^{(1)} {\sigma^2 x^2   \over q} y_2 \; ,
\eqno (A.36)$$
and
$$
x  {\dd y_2  \over \dd x } = \left[ x - {q \over \sigma^2 x^2 }
( A^{(0)} - U) \right] y_1 + A^{(1)} y_2 \; .
\eqno (A.37)$$

To use this option, the equilibrium model must obviously
be set up correctly, as defined in equations (A.31).
In addition, the parameter {\tt iturpr} must be set to 2.

\subsect
{\it A.3.3 Neglect of the Lagrangian perturbation in the turbulent pressure}

We still assume the separation in pressure given in equation (A.24),
the equations of motion (A.25) and that the equilibrium
total pressure satisfies the equation of hydrostatic support (A.26).
However, we now assume that the Lagrangian perturbation
$\delta \pturb = 0$
(see also Rosenthal {\etal} 1995).
The equations of motion are now
$$
\rho {\partial^2 {\bf \delta r}  \over \partial t^2 } = - \nabla p^{\prime}
- \rho^{\prime} g {\bf a}_r + \rho \nabla \Phi ' \; .
\eqno (A.38)$$
Furthermore, assuming still equation (A.28) for the adiabatic
change in the {\it thermodynamic} pressure, we obtain for the
relation between the Lagrangian perturbations in the total pressure and
density:
$$
{\delta p \over p} = 
{\delta \pg \over p} = 
{\delta \pg \over p} = 
{\pg \over p} \Gamma_1 {\delta \rho \over \rho} \; ,
\eqno(A.39)
$$
or
$$
{\delta p \over p} = \Gammar {\delta \rho \over \rho} \; ,
\eqno(A.40)
$$
defining {\it the reduced $\Gammar \equiv (\pg/p) \Gamma_1$}
(Rosenthal {\etal} 1995).
Here, obviously, $\Gamma_1$ is the thermodynamic adiabatic exponent.

With this definition, the oscillation equations are precisely
as in the case without turbulent pressure, when expressed in
terms of the Eulerian perturbation $p^{\prime}$ in total pressure,
provided $\Gamma_1$ is replaced throughout by $\Gammar$.
It should be noticed also that this is the case for {\it any}
relation of the form (A.40), for a function $\Gammar$ defined
in the equilibrium model.
Indeed, it was argued by Rosenthal {\etal} (1999) that
a more appropriate form of $\Gammar$ could be obtained by
averaging the equations for turbulent convection.

Implementation of this approximation therefore only requires
modification of the equilibrium model.
Specifically, the definitions of $A_2 - A_4$ are changed to
$$
\eqalign{
& A_2 = - {1 \over \Gammar} {\dd \ln p \over \dd \ln r} 
= {G m \rho \over \Gammar p r} \; , \cr
& A_3 =  \Gammar \; , \cr
& A_4 = {1 \over \Gammar} {\dd \ln p \over \dd \ln r} 
- {\dd \ln \rho \over \dd \ln r} \; , \cr
}
\eqno(A.41)
$$
with the remaining definitions being unchanged.

The definition of the sound speed in this case deserves a little comment.
It is evident from the unchanged form of the oscillation equations
that the relevant expression for the sound speed is 
$$
c^2 = {\Gammar p \over \rho} \; ,
\eqno(A.42)
$$
expressed in terms of the modified $\Gamma_1$ and the total pressure.
It is interesting that with the simple definition of $\Gammar$
implied by equation (A.39) this is identical to $\Gamma_1 \pg / \rho$,
{\ie}, the ``thermodynamic'' squared sound speed.
However, with a more general definition of $\Gammar$ this is not 
generally the case.
On the other hand, it is obvious that equation (5.7) still holds.

\subsect
{\bf A.4 Inclusion of rotational effects}

In this case it is assumed that the equilibrium model has been computed
including the spherically symmetric part of the centrifugal acceleration,
and that the model as been defined with $D_8 = 20$, to flag for the
inclusion of $A_6$.
This case allows calculation of the structure effects on the frequencies,
to second order in the angular velocity $\Omega$, in preparation for
calculation of further second-order effects according to the formalism
of Gough \& Thompson; this is used if {\tt irotsl} $\neq 1$.
When {\tt irotsl} $= 1$, the first-order perturbations to the
eigenfrequencies and eigenfunctions are including, according to the
formalism of Soufi {\etal} (1998).

The equation of hydrostatic support in this case is
$$
{\dd p \over \dd r } = - g_{\rm e} \rho \; , \qquad {\rm with}
\quad g_{\rm e} = g - {2 \over 3} r \Omega^2 \; .
\eqno (A.43)$$
Here and in the following
$$
g = {G m \over r^2} \; .
\eqno(A.44)
$$
In accordance with the evolution code, we introduce $\tilde q$ by
$$
\tilde q = {g_{\rm e} \over g} q \; ,
\eqno(A.45)
$$
where, as usual, $q = m/M$.
Thus equation (A.43) may also be written
$$
{\dd p \over \dd r } = - {G M \tilde q \over r^2} \rho \; .
\eqno(A.46)
$$

This redefinition of the equilibrium structure is formally equivalent
to the case considered in Section A.3.1, of including a turbulent
body force.
Thus we similarly 
modify the definitions of the equilibrium variables $A_1 , A_2 $ 
and $A_4$; instead of equations (5.1), we use
$$
\eqalign{
& A_1 \equiv {R^3  \over G M } {g_{\rm e} \over r }
\equiv {\tilde q  \over x^3} \; , \cr
& A_2 = \tilde V_g \equiv - {1 \over \Gamma_1 } {\dd \ln p \over \dd \ln r} 
 \equiv {r \rho g_{\rm e}  \over \Gamma_1 p} \; , \cr
& A_4 = \tilde A \equiv {1 \over \Gamma_1} {\dd \ln p  \over \dd \ln r} -
{\dd \ln \rho  \over \dd \ln r} \; ,
}
\eqno (A.47)
$$
which are identical to equations (A.16),
together with $A_3$ and $A_5$ which are unchanged.
In addition, we introduce the new variable $A_6$ by
$$
A_6 = {q \over \tilde q} = {g \over g_{\rm e}} \; .
\eqno(A.48)
$$
(As noted in Section 5, the inclusion of $A_6$ in the model data is
flagged by setting $D_8 = 20$.)
Finally, we introduce again
$$
V_g = {r \rho g  \over \Gamma_1 p} \; ,\qquad  
A = - V_g - {\dd \ln \rho  \over \dd \ln r} \; .
\eqno (A.49)$$
Clearly in the absence of rotation $A_1 - A_5$ reduce
to the variables defined in Section~5, and $A_6 = 1$.

The relations between the variables defined here and
more ``physical'' variables in equation (5.3) have to
be modified in this case:
$$
p = {G M^2  \over 4 \pi R^4}  {x^2 A_1^2 A_5 A_6 \over A_2 A_3} \; , \qquad
{\dd p \over \dd r }= - {G M^2  \over 4 \pi R^5} x A_1^2 A_5 A_6 \; , \qquad
\rho = {M \over 4 \pi R^3}  A_1 A_5 A_6 \; . 
\eqno (A.50)
$$
The expressions for the buoyancy frequency, Lamb frequency and
sound speed, equations (5.4), (5.5) and (5.7), are unchanged
in terms of $A_1 - A_5$.

The dimensionless perturbation variables $y_1 - y_4$ are defined
as in Section~2.1; in particular the actual gravitational acceleration
$g$ is used in the definition of $y_3$ ({\cf} eq.\ 2.5a).

\subsect
{\it A.4.1 Equations with no explicit inclusion of rotation}

In this case the equations are essentially identical
to those presented in Section A.3.1;
for completeness, they are repeated here.
For non-radial oscillations 
$$
x {\dd y_1  \over \dd x } = (\tilde V_g - 2 )  y_1 +
\left( 1 - {\tilde V_g  \over \tilde \eta} \right) y_2 
- V_g y_3 \; ,
\eqno (A.51)$$
$$
x {\dd y_2  \over \dd x } = [ l ( l + 1) - \tilde \eta \tilde A ] y_1 
+ ( \tilde A - 1 ) y_2 + \eta \tilde A y_3 \; ,
\eqno (A.52)$$
$$
x {\dd y_3  \over \dd x } = y_3 + y_4 \; ,
\eqno (A.53)$$
and
$$
\eqalign{
x {\dd y_4  \over \dd x } = & - \lambda \tilde A U y_1 -
\lambda U {\tilde V_g   \over \tilde \eta} y_2  \cr
& + \left[ l ( l + 1) + U(A - 2 )
+ \left({\tilde q \over q} - \lambda \right) U V_g \right] y_3 
+ 2(1 - U )  y_4 \; . \cr
}
\eqno (A.54)
$$
Here $\tilde \eta = l ( l + 1) \tilde g / ( \omega^2 r )$.
Note that equation (A.54) differs from the corresponding equation
(A.21), in replacing $(1 - \lambda)$ by $(\tilde q / q - \lambda)$
in the last term multiplying $y_3$.
This deserves an extra check in both equations.

For radial oscillations the equations are
$$
x  {\dd y_1  \over \dd x } = (\tilde V_g - 2 ) y_1 -
\tilde V_g {\sigma^2 x^2   \over \tilde q} y_2 \; ,
\eqno (A.55)$$
and
$$
x  {\dd y_2  \over \dd x } = \left[ x - {1 \over \sigma^2 x^2 }
( \tilde q \tilde A - \lambda q U) \right] y_1 + \tilde A y_2 \; .
\eqno (A.56)$$

To use this option, the equilibrium model must obviously
be set up correctly, as defined in equations (A.47).

\subsect
{\it A.4.2 Including first-order rotation, according to Soufi {\etal} (1998)}

This is based on notes provided by K. Karami, translating between
the notation used by Soufi {\etal} (1998) and the notation used here.
We introduce
$$
F_r(r) = {2 \over 3} r \Omega^2 \; , \qquad
\sigma_r = {F_r(r) \over g} \; , \qquad
C = {x^3 \over q} \; , \qquad
C_r = {C \over 1 - \sigma_r} \; .
\eqno(A.57)
$$
Then, obviously, $1 - \sigma_r = g_{\rm e} / g = \tilde q / q$,
and
$$
C_r = {x^3 \over \tilde q} = A_1^{-1} \; .
\eqno(A.58)
$$
We also need $\sigma_\Omega$, defined by
$$
\sigma_\Omega^2 = {R^3 \over G M} \Omega^2 \; .
\eqno(A.59)
$$
This is for the moment obtained by noting that
$$
A_6^{-1} = {g_{\rm e} \over g} = 1 - {2 \over 3} r \Omega^2 {r^2 \over G m}
= 1 - {2 \over 3} \sigma_\Omega^2 (A_1 A_6)^{-1} \; ,
$$
whence
$$
\sigma_\Omega^2 = {3 \over 2} (A_6 - 1) A_1 \; .
\eqno(A.60)
$$
(Note that this way of calculating $\sigma_\Omega$ introduces significant
loss of precision, since $A_6$ is typically relatively close to unity.
In future, it may be replaced by an extension to the $\{A_i\}$ to
include explicit information about $\Omega$.)

Following Soufi {\etal} (1998), and Karami, we finally introduce
$$
\hat \sigma = \sigma + m \sigma_\Omega \; , \qquad
\alpha = 2 m {\sigma_\Omega \over \hat \sigma} \; , \qquad
\zeta = {\Lambda \over \Lambda - \alpha} {\Lambda \over C_r \hat \sigma^2} \; ,
\qquad
h_1 = {\Lambda \alpha \over \Lambda - \alpha} \; ,
\eqno(A.61)
$$
where $\Lambda = l(l+1)$.
Here $\zeta$ may also be written
$$
\zeta = {\Lambda \over \Lambda - \alpha} 
{l(l+1) \tilde q \over \hat \sigma^2 x^3}
= {\Lambda \over \Lambda - \alpha}  
\left(\sigma \over \hat \sigma \right)^2 \tilde \eta \; .
\eqno(A.62)
$$
Then we finally obtain the relevant form of the non-radial equations:
$$
x {\dd y_1  \over \dd x } = (\tilde V_g - 2 +h_1)  y_1 +
( \zeta - \tilde V_g) {y_2 \over \tilde \eta} 
- V_g y_3 \; ,
\eqno (A.63)$$
and
$$
x {\dd y_2  \over \dd x } = 
\left[ l ( l + 1) \left({\hat \sigma \over \sigma} \right)^2  
- \tilde \eta \left(\tilde A + {h_1^2 \over \zeta} \right) \right] y_1 
+ ( \tilde A - 1 - h_1) y_2 + \eta \tilde A y_3 \; ;
\eqno (A.64)$$
equations (A.53) and (A.54) are unchanged.
Also, of course, the equations for radial oscillation are the same
as above.


\end
