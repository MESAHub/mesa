\magnification=\magstephalf
\def\nwl{\hfill\break}
\def\note #1]{{\bf #1]}}
\def\etal{{\it et al.}}
\def\eg{{\it e.g.}}
\def\cf{{\it cf.}}
\def\ie{{\it i.e.}}
\def\etc{{\it etc.}}
\def\viz{{\it viz.}}
\def\dd{{\rm d}}
\def\newpage{\vfill\eject}
\def\muHz{\,\mu {\rm Hz}}
\def\draft{\headline{\bf File: \jobname\hfill DRAFT\hfill\today}}

\def\today{\ifcase\month\or
  January\or February\or March\or April\or May\or June\or
  July\or August\or September\or October\or November\or December\fi
  \space\number\day, \number\year}
\font\fourtbf=cmbx10 scaled \magstep2

%  section macros, designed to avoid new page after section headings.
\def\mainsect #1\par{\interlinepenalty=100\bigskip\bigskip\noindent
\hangindent=1 true cm
\hangafter=1
{#1}\par%
\penalty10000\par\bigskip\par\penalty10000\noindent}
%\def\mainsect #1\par{\bigskip\bigskip\noindent{#1}\par%
%\penalty10000\par\bigskip\par\penalty10000\noindent}

\def\subsect #1\par{\bigskip\noindent{#1}\par%
\penalty10000\bigskip\par\penalty10000\noindent}
\def\intsect #1\par{\par\penalty10000\bigskip\penalty10000\noindent{#1}\par%
\penalty10000\par\bigskip\par\penalty10000\noindent}
\def\ref{\par\noindent
	\hangindent=0.7 true cm
	\hangafter=1}
\parskip = 5pt
\centerline{\fourtbf Notes on using the solar models and }
\centerline{\fourtbf revised adiabatic pulsations package}
\medskip
\centerline{\today}
\medskip
\centerline{J. Christensen-Dalsgaard}

\mainsect
{\bf 1 Introduction}

These notes give some information on the programs and data provided with the
package.
It is assumed that the package has been installed as described in
the {\tt README} file provided, and that the reader is familiar with
the basic concepts discussed in {\it Notes on adiabatic oscillation
programme} (in the following {\it Notes}),
particularly the input format and output files.
All directory names are given relative to the root where
the package was unpacked and installed.

The basic way that all the programs are run is through {\tt csh}
scripts provided in the directory {\tt bin};
to get easy access to the scripts, {\tt bin}
(with its appropriate prefix) must be included in the 
command search path.
These call the executables (denoted by ending in {\tt .x}).
Parameters to the executable may be provided in one of two ways:
\medskip
\item{--} as arguments to the script
\item{--} in a {\it control file} which is provided as as argument 
to the script; this is typically used in more complex cases.
The control file sets up parameters and input and output files
for the program.
\medskip\noindent
In most cases the proper arguments to the scripts, with a brief
description, is provided by calling the script either
with no argument or with the argument {\tt -help},

{\tt script -help}


Sample control files are provided in the {\tt models.tar}
distribution, to appear in the {\tt models} directory.
The structure for these control files is similar to the
structure of the control file to the basic pulsation code,
as described in {\it Notes}, section 7.2.1.

A note on program naming: for ease of reference I have in almost
all cases kept the same names as in my own installation.
In particular, I have kept the ``{\tt .d}'' in the most of the names,
indicating that the programs are in double precision (8 byte reals).

\subsect
{\it 1.1 Notes on revision}

The present set of programs replaces the set released in 1994.
There have been numerous changes in several of the programs and
corresponding control files; for this reason, unfortunately,
backwards compatibility has not been ensured.
However, none of the functionality in the old package has been lost.
Also, the formats of the basic data products are the same,
so that there should be no difficulty in using the new 
package with old files of modes or models.
It is suggested that the user completely replaces the old package with the
new, rather than to try to combine them.
\newpage

\mainsect
{\bf 2 Computing adiabatic modes}

The program is started by the command

{\tt adipls.n.d <control file>}

\noindent
The sample control file recomputes a modest set of modes, at selected 
degrees, based on reading trial frequencies from a grand summary.
Only eigenfrequencies are output.
This is useful for testing that the program works correctly. 
The computed frequencies can be compared with the frequencies
by using the program {\tt freqdif.d} (see below).
It will presumably be comparatively easy to modify the
control file to consider other cases.
A full list of control parameters is provided in {\it Notes}.

If pulsations are to computed for other models than those provided,
the model must be transformed into the appropriate format,
as described in {\it Notes}, Sections 5 and 7.1.
Note in particular that for a physical model (in contrast to
a polytrope) $D_7$ must be set to $-1$ (this affects the
way the surface boundary condition is treated).

In general, the mesh resulting from, {\eg}, an evolution
calculation is not optimal for oscillation calculations.
The mesh can be reset with the program {\tt redistrb.d}
described below.

\mainsect
{\bf 3 Auxiliary programs}

The package contains a set of programs for manipulating the output
of the pulsation program and the model files. 
These are provided in the directory {\tt adiajobs} as Fortran source code,
and are compiled and linked with the {\tt make} command
(see the {\tt README} file).
The following nomenclature is used, for any program {\tt program}:
\medskip
\item{--} {\tt program.d.f} is the Fortran source code. 
\item{--} {\tt program.d.x} is the executable program.
\item{--} {\tt program.d} is the {\tt csh} script
(residing in {\tt bin}) used to call the program.
\medskip
Unlike the pulsation code, there are no separate notes documenting the
use of these programs.
Below follows a brief list of the most important of them,
with indications of what they do and how they are called.
In general, a fairly detailed description of the input parameters
and operation of the program is provided in the source code.

\medskip
\item{$\bullet$}
{\tt form-amdl.d}:
Transforms between binary and ASCII versions of models for pulsation program.
It is called with the command

{\tt form-amdl.d <case> <input file> <output file>}

\item{}
Here {\tt <case>} determines the direction of the transformation:

\itemitem{--} {\tt case = 1}: transform from binary to ASCII
\itemitem{--} {\tt case = 2}: transform from ASCII to binary


\medskip
\item{$\bullet$}
{\tt form-agsm.d}:
Transforms between binary and ASCII versions of grand summaries.
It is called with the command

{\tt form-agsm.d <case> <input file> <output file> [diag]}

\item{}
Here {\tt <case>} determines the direction of the transformation:
\itemitem{--} {\tt case = 1}: transform from binary to ASCII
\itemitem{--} {\tt case = 2}: transform from ASCII to binary
\item{} {\tt [diag]} controls the amount of output, as in
the {\tt scan-agsm.d} command (see below).


\medskip
\item{$\bullet$}
{\tt redistrb.d}:
Resets mesh in model for pulsation program. It is called with the command

{\tt redistrb.d <control file>}

\item{}
A sample control file ({\tt redistrb.in}) is provided.
This is set up to set a mesh suitable for p-mode calculations.
Although it is possible to set individual parameters, the following
two choices should be appropriate for computing modes in solar models:
\itemitem{--}
{\tt icase = 11} for p modes.
\itemitem{--}
{\tt icase = 12} for g modes.
\smallskip
\item{}
I use the convention of denoting p mode models
with 600, 1200, 2400, 4800, $\ldots$ by the file trailers
{\tt p, p1, p2, p3}, $\ldots$, with a similar notation for g modes.

\medskip
\item{$\bullet$}
{\tt scan-amdl.d}: prints summary of pulsation model file.
It is called with the command

{\tt scan-amdl.d <pulsation model file>}

\medskip
\item{$\bullet$}
{\tt set-asscal.d}: Adds dimensionless integral of $\dd r/c$ and
dimensionless asymptotic scaling factor to pulsation model file.
See Appendix A for precise definitions.
It is called with the command

{\tt set-asscal.d <input model file> <output model file> [truncation radius]}

where

\itemitem{--} {\tt truncation radius} (default photospheric radius)
is upper limit of scaling integral.

\medskip
\item{$\bullet$}
{\tt compamod.n.d}:
Finds differences between two pulsation-model files.
It is called with the command

{\tt compamod.n.d <control file>}

\item{}
A sample control file ({\tt compamod.n.in}) is provided.

\medskip
\item{$\bullet$}
{\tt fgong-amdl.d}:
Converts model in ASCII GONG format (see Section~5)
to pulsation model file.
It is called with the command

{\tt fgong-amdl.d <Input GONG file> <Output pulsation-model file>}

\medskip
\item{$\bullet$}
{\tt diff-fgong.d}:
Finds differences between two models in ASCII GONG format (see Section~5).
It is called with the command

{\tt diff-fgong.d <First input model> <Second input model> <Output file> [case]}

\item{}
Use {\tt diff-fgong.d -help} to get details about the different cases.
The default is difference at fixed fractional radius $r/R$.

\bigskip
\item{$\bullet$}
{\tt scan-agsm.d}: prints summary of grand summary file.
It is called with the command

{\tt scan-agsm.d <grand summary file>}

\item{}
or

{\tt scan-agsm.d <grand summary file> 1}

\item{}
In the first case all modes are listed, in the second only the
first mode for each degree.

\medskip
\item{$\bullet$}
{\tt scan-ssm.d}: prints summary of short summary file.
It is called with the command

{\tt scan-ssm.d <short summary file>}

\medskip
\item{$\bullet$}
{\tt scan-amde.d}: prints summary of eigenfunction file.
It is called with the command

{\tt scan-amde.d <eigenfunction file> <case>}

\item{}
Here {\tt case = 1} for a file containing the full set
of eigenfunctions, {\tt case = 2} for file containing
restricted set
(for formats for eigenfunction files, see
{\it Notes}, section 8.5).

\medskip
\item{$\bullet$}
{\tt set-obs.d}: sets ASCII formatted file of mode data
from grand or short summary, with each record having
the form $l, n, \nu$, where $\nu$ is frequency in $\muHz$.
(This can be thought of as frequencies in the form of
observed data.)
It is called with the command

{\tt set-obs.d <case> <input file> <output file>}

\item{}
Here the input file can be either a grand summary or a 
short summary.
{\tt <case>} depends on the type of input file and, for
input from a grand summary, determines the choice of frequency:
\itemitem{--}
{\tt case = 1}: grand summary, variational frequency.
\itemitem{--}
{\tt case = 2}: short summary.
\itemitem{--}
{\tt case = 4}: grand summary, from eigenfrequency in {\tt cs(20)}.
Note that this allows setting Cowling
approximation frequency.
\itemitem{--}
{\tt case = 5}: grand summary, from Richardson extrapolation frequency
in {\tt cs(37)}, if this is set.
\itemitem{--}
Otherwise variational frequency is used.
\item{}
If {\tt case > 10}, set frequencies according to {\tt case - 10},
but include also mode energy ({\cf} {\it Notes}, section~4.3) in the file.

\medskip
\item{$\bullet$}
{\tt diff-sum.d}:
Compares two mode sets, assumed to be ordered properly,
outputting modes that appear in one set but not the other.
The mode sets can
be of the form of grand or short summaries,
or observed data.
It is called with the command

{\tt diff-sum.d <case 1> <case 2> <input file 1> <input file 2> 
$\backslash$\nwl
\hbox{\null} \hskip 10em <excess 1> <excess 2>}

\item{}
Here {\tt case 1} and {\tt case 2}
determine the types of mode sets (which may be different):
\itemitem{--}
{\tt case = 1}: grand summary.
\itemitem{--}
{\tt case = 2}: short summary.
\itemitem{--}
{\tt case = 3}: observed frequencies.

Also
\itemitem{--} {\tt <excess 1>}: Output file of modes in file 1 but not in file 2
\itemitem{--} {\tt <excess 2>}: Output file of modes in file 2 but not in file 1

Note: {\tt excess} files are output in same form as corresponding input file.



\medskip
\item{$\bullet$}
{\tt res-amde.d}: sets restricted eigenfunction file from full file.
It is called with the command

{\tt res-amde.d <input file> <output file> <case>}

\item{}
Here {\tt case = 1} produces an output file with $y_1, y_2$,
whereas {\tt case = 2} produces an output file with $\hat z_1, \hat z_2$
(for formats for eigenfunction files, see
{\it Notes}, section 8.5).

\medskip
\item{$\bullet$}
{\tt freqdif.d}:
Computes frequency differences between various types of
frequency sets.
It is called with the command

{\tt freqdif.d <control file>}

\item{}
A sample control file ({\tt freqdif.in}) is provided,
to compute differences between the original modes provided
in a grand summary, and the modes computed with the
control file for the adiabatic pulsation program.

\item{}
The frequency sets may be in the form of grand or short summaries,
or observed frequencies, as specified by the appropriate input
parameters (a rudimentary set of notes is provided in
{\tt freqdif.in}).
{\tt freqdif.d} automatically selects the overlapping modes between the
two sets.
If the first mode set is in the form of a grand summary, 
the frequency differences can be automatically scaled by the mode
inertia ratio $Q_{nl}$.

\medskip
\item{$\bullet$}
{\tt selsum.d}:
Selects a subset of a mode set, determined by
windowing  and/or by matching modes in another set.
Both input and selection sets can, independently,
be of the form of grand or short summaries,
or observed data.
It is called with the command

{\tt selsum.d <control file>}

\item{}
A sample control file ({\tt selsum.in}) is provided.

\medskip
\item{$\bullet$}
{\tt mer-sum.d}:
Combines two mode sets, assumed to be ordered properly.
The mode sets can
be of the form of grand or short summaries,
or observed data.
It is called with the command

{\tt mer-sum.d <case> <input file 1> <input file 2> <output file>}

\item{}
Here {\tt case} determines the type of mode set:
\itemitem{--}
{\tt case = 1}: grand summary.
\itemitem{--}
{\tt case = 2}: short summary.
\itemitem{--}
{\tt case = 3}: observed frequencies, without errors.
\itemitem{--}
{\tt case = 4}: observed frequencies, with errors.

\medskip
\item{$\bullet$}
{\tt sortsum.d}:
Combines and reorders several mode sets,
possibly applying windowing to them.
The mode sets can
be of the form of grand or short summaries,
or observed data.
It is called with the command

{\tt sortsum.d <control file>}

\item{}
A sample control file ({\tt sortsum.in}) is provided.
This is typically useful if several runs of the pulsation code
is required to produce a complete mode set.
To merge two mode sets, already ordered,
the simpler program {\tt mer-sum.d} can be used (see above).

\medskip
\item{$\bullet$}
{\tt setexec.d}:
Scans for missing modes in a mode set file,
estimating their frequencies from interpolation, and writing
the estimated mode parameters as a short summary file on unit 3.
This may subsequently be used as trial input for a new attempt to
determine the modes.
The input file may be in the form of a short summary or grand summary.
The pogramme is called with the command

{\tt setexec.d <control file>}

\item{}
A sample control file ({\tt setexec.in}) is provided.

\medskip
It is obvious that this is only a sample of the programmes
supplied in {\tt adiajobs}.
The user is invited to explore.

\mainsect
{\bf 4 IDL procedures}

The package contains a small set of IDL procedures
to read products of the pulsation code.
They are in the directory {\tt idl\_pro}.
Some documentation on their use is contained at the start of
each procedure.

The following procedures are provided:

\item{$\bullet$}
{\tt read\_amde.pro}: reads an eigenfunction from eigenfunction file.
\item{$\bullet$}
{\tt read\_amdes.pro}: reads several eigenfunctions from eigenfunction file.
\item{$\bullet$}
{\tt read\_amdl.pro}: reads pulsation model from file.
\item{$\bullet$}
{\tt read\_gsm.pro}: reads a single grand summary from file.
\item{$\bullet$}
{\tt read\_gsms.pro}: reads all grand summaries on file.
\item{$\bullet$}
{\tt read\_fgong.pro}: reads model in ASCII GONG format (see below).

\mainsect
{\bf 5 Models}

The {\tt models.tar} distribution file contains
calibrated models of the present Sun,
frequencies for one of these models and input control
files for the programmes.

\subsect
{\it 5.1 Some notes on naming conventions}

I have kept my internal notation for the files,
even though this may, in isolation, seem rather artificial.
This involves a fixed naming convention
for products of the computation.
This is obviously arbitrary, but provides an immediate
overview of the results:

\item{$\bullet$}
{\tt amdl.<model descriptor>}: Pulsation-model file on the original mesh
used in the evolution calculation.
The variables and file format are described in the Appendix.
\item{$\bullet$}
{\tt amdl.<model descriptor>.p<n>}: 
Pulsation-model file on a mesh suitable for p-model calculations;
{\tt p1}, {\tt p2}, {\tt p3}, $\ldots$ indicate meshes with 1200, 2400,
4800, $\ldots$ points.
\item{$\bullet$}
{\tt amdl.<model descriptor>.g<n>}: 
Pulsation-model file on a mesh suitable for g-model calculations;
{\tt g1}, {\tt g2}, {\tt g3}, $\ldots$ indicate meshes with 1200, 2400,
4800, $\ldots$ points.
\item{$\bullet$}
{\tt fgong.<model descriptor>}: extensive set of model variables,
in the standard ASCII GONG format.
The variables, and the format, are described in the notes
contained in the file \nwl
{\tt file-format.tex} in the {\tt notes} directory.

\medskip

\item{$\bullet$}
{\tt agsm.<model descriptor>.p<n>}: grand summary of modes computed
for model \nwl
{\tt amdl.<model descriptor>.p<n>}.
Different sets are distinguished by adding yet another trailer.
\item{$\bullet$}
{\tt ssm.<model descriptor>.p<n>}: short summary of modes computed
for model \nwl
{\tt amdl.<model descriptor>.p<n>}.
\item{$\bullet$}
{\tt amde.<model descriptor>.p<n>}: full eigenfunction file computed
for model \nwl
{\tt amdl.<model descriptor>.p<n>}.
\item{$\bullet$}
{\tt amde.<model descriptor>.p<n>.y}: restricted eigenfunction file,
providing $y_1, y_2$, computed
for model {\tt amdl.<model descriptor>.p<n>}.
\item{$\bullet$}
{\tt amde.<model descriptor>.p<n>.z}: restricted eigenfunction file,
providing $\hat z_1, \hat z_2$, computed
for model {\tt amdl.<model descriptor>.p<n>}.

\medskip
ASCII versions of binary files (see the {\tt form-amdl.d} and
{\tt form-agsm.d} commands above) are indicated by adding ``{\tt .for}''
to the names of the binary files.


\subsect
{\it 5.2 Notes on the models and frequencies}

Most of the results are for Model S, used as reference in the GONG
series of {\it Science} articles (see Christensen-Dalsgaard {\etal} 1996),
or closely related models.
In addition, for consistency with an earlier
distribution, I have included a couple of models
computed by Christensen-Dalsgaard, Proffitt \& Thompson (1993).
Full descriptions of the physics used in the calculations
is given in the relevant papers.

The models are 
provided in terms of pulsation variables ({\tt amdl} models)
and ASCII GONG variables ({\tt fgong} models).
They are identified by my internal model descriptors.
This naming may appear a little unwieldy; however, I strongly
recommend recording these names for later unique reference to the models.
The following cases are provided:
\medskip
\item{$\bullet$} 
{\tt l5bi.d.15}: Model S of Christensen-Dalsgaard {\etal} (1996). 
OPAL(1992) opacity, OPAL equation of state,
including helium and heavy-element settling. Age of present Sun 4.6 Gyr.
\item{$\bullet$} 
{\tt l5bi.d.24}: OPAL(1992) opacity, OPAL equation of state,
including helium and heavy-element settling. Age of present Sun 4.52 Gyr.
\item{$\bullet$} 
{\tt l5bi.d.28}: OPAL(1995) opacity, OPAL equation of state,
including helium and heavy-element settling. Age of present Sun 4.6 Gyr.
\medskip
\item{$\bullet$} 
{\tt l4b.14}: Model without helium settling;
MHD equation of state, OPAL(1992) opacities.\nwl
(Model 1 of Christensen-Dalsgaard {\etal} 1993, Table 1).
\item{$\bullet$} 
{\tt l4b.d.18}: Model with helium settling;
MHD equation of state, OPAL(1992) opacities.\nwl
(Model 2 of Christensen-Dalsgaard {\etal} 1993, Table 1).

\medskip
In addition to the model files on the original mesh,
for Model S I have also included the model on a p-mode mesh
with 2400 points ({\tt amdl.l5bi.d.15.p2}).

\bigskip\noindent
Files of oscillation results:
\medskip

\item{$\bullet$} 
{\tt agsm.l5bi.d.15.p2}: Grand summary of
a moderate set of modes, at selected degrees
(use {\tt scan-agsm.d} to find out which).
This set is useful for testing the code and initial investigations
of the effects of changes in the model.

\item{$\bullet$} 
{\tt agsm.l5bi.d.15.p2.md}: Grand summary of
an extensive set of modes, 
including all p modes below the acoustic cut-off frequency at
degrees below 300.
This is probably adequate for analysis of the GONG data
(but, as {\it Notes} will show, there are easy ways to
produce complete sets to even higher degree).

\item{$\bullet$} 
{\tt obs.l5bi.d.15.p2.md}: ASCII file (``observational format'') for
an extensive set of modes, 
including all p modes below the acoustic cut-off frequency at
degrees below 300.

\item{$\bullet$} 
{\tt agsm.l5bi.d.15.g3}: 
Grand summary of a fairly extensive set of g and f modes of
degree 1~--~7.

\medskip

\item{$\bullet$} 
{\tt agsm.l4b.14.p1}: 
Grand summary of a moderate set of modes, at selected degrees.

\item{$\bullet$} 
{\tt agsm.l4b.14.p1.md}: 
Grand summary of an extensive set of modes, 
including all p modes below the acoustic cut-off frequency at
degrees below 200, as well as modes at selected higher degree.

\newpage

\mainsect
{\bf References}

\ref
Christensen-Dalsgaard, J., Proffitt, C. R. \& Thompson, M. J., 1993.
[Effects of diffusion on solar models and their oscillation frequencies].
{\it Astrophys. J.}, {\bf 403}, L75 -- L78.
\ref
Christensen-Dalsgaard, J., D\"appen, W., Ajukov, S. V., Anderson, E. R.,
Antia, H. M., Basu, S., Baturin, V. A., Berthomieu, G., Chaboyer, B.,
Chitre, S. M., Cox, A. N., Demarque, P., Donatowicz, J., Dziembowski, W. A.,
Gabriel, M., Gough, D. O., Guenther, D. B., Guzik, J. A., Harvey, J. W.,
Hill, F., Houdek, G., Iglesias, C. A., Kosovichev, A. G., Leibacher, J. W.,
Morel, P., Proffitt, C. R., Provost, J., Reiter, J., Rhodes Jr., E. J.,
Rogers, F. J., Roxburgh, I. W., Thompson, M. J., Ulrich, R. K., 1996.
[The current state of solar modeling].
{\it Science}, {\bf 272}, 1286 -- 1292.

\vfill\eject
\mainsect
{\bf Appendix. The structure of the pulsation model file}

For convenience, I repeat the information about the model-file
structure here (see also {\it Notes}, Section~5).
In addition, I here list the additional variables set up
by calling {\tt set-asscal.d}:

The model variables is defined in the array {\tt data(1:8)}
of global parameters, {\tt x(1:nn)} defining the mesh
and {\tt aa(1:ia, 1:nn)} giving the variables at each mesh point.
The basic set of variables consists of
$$
\eqalignno{
{\tt x(n)} & = x \equiv r / R \; , & (A.1) \cr
{\tt aa(1,n)} & = A_1 
\equiv q / x^3  , \qquad \hbox{\rm where } q = m / M \; , & (A.2) \cr
{\tt aa(2,n)} & = A_2 = V_g 
\equiv -  {1 \over \Gamma_1} {\dd \ln p  \over \dd \ln r}
= {G m \rho   \over \Gamma_1 p r} \; ,  & (A.3) \cr
{\tt aa(3,n)} & = A_3 \equiv \Gamma_1 \; ,  & (A.4) \cr
{\tt aa(4,n)} & = A_4 = A 
\equiv {1 \over \Gamma_1} {\dd \ln p \over \dd \ln r} - 
{\dd \ln \rho   \over \dd \ln r} \; ,  & (A.5) \cr
{\tt aa(5,n)} & = A_5 = U \equiv {4 \pi \rho r^3  \over m } \; . & (A.6) \cr
\noalign{\noindent In addition, in some models}
{\tt aa(6,n)} & \quad \hbox{\rm  is a variable dealing with turbulent pressure} 
\; .\cr
\noalign{\noindent In models extended with {\tt set-asscal.d}}
{\tt aa(7,n)} & = A_7 \equiv \int_0^x {\dd x' \over \tilde c(x')} \; ,  & (A.7) \cr
{\tt aa(8,n)} & = A_8 \equiv \int_x^{x_s} 
\left[1 - \left({\tilde c(x') x \over \tilde c(x) x'}\right)^2 \right]^{-1/2}
{\dd x' \over \tilde c(x')} \; . & (A.8) \cr
}
$$
Here $\tilde c = (R / G M)^{1/2} c$ is the dimensionless sound speed.
The quantities $A_1 - A_8$ are clearly all dimensionless.
In addition we use the following ``global'' quantities for the model:
$$
\eqalign{
{\tt data(1)} = & D_1 \equiv M \; , \cr
{\tt data(2)} = & D_2 \equiv R \; , \cr
{\tt data(3)} = & D_3 \equiv p_{\rm c} \; , \cr
{\tt data(4)} = & D_4 \equiv \rho_{\rm c} \; , \cr
{\tt data(5)} = & D_5 \equiv 
- \left( {1 \over \Gamma_1 p }  {\dd^2 p \over \dd x^2} \right)_{\rm c} \; , \cr
{\tt data(6)} = & D_6  \equiv 
 - \left(  {1 \over \rho } {\dd^2 \rho  \over \dd x^2} \right)_{\rm c} \; , \cr
{\tt data(7)} = & D_7  \equiv \mu \; , \cr
{\tt data(8)} = & D_8: \hbox{see below} \; .\cr
} \eqno(A.9)
$$
Here $R$ and $M$ are photospheric radius and mass of the model (the photosphere
being defined as the point where the temperature equals the effective
temperature). In a complete model 
$p_{\rm c}$ and $\rho_{\rm c}$ are central pressure and 
density, and $D_5$ and $D_6$ are evaluated at the centre.
In an envelope model (that does not include the centre) $D_3$ and
$D_4$ should be set to the values of pressure and density at the
innermost mesh point, and $D_5$ and $D_6$ may be set to zero.
The dimensional variables ({\ie}, $D_1 - D_4$) must be given in $cgs$ units.
For a physical model $D_7$ is set to $-1$;
in a model with a polytropic surface layer, $D_7$ is the polytropic
index of the surface region.
The notation is otherwise standard. The model may include an
atmosphere (for solar models a simplified atmosphere
extending out to roughly the temperature minimum is typically used). Thus
at the surface possibly $x > 1$.

These variables are convenient when the equations are formulated as
by {\eg} Dziembowski; but it should be possible to derive any set
of variables required for {\it adiabatic} oscillation calculations from them.
$D_5$ and $D_6$ are needed
for the expansion of the solution around the centre. 

The quantity $D_8$ is used to flag for a different number
of variables in the file, or otherwise a different structure.
Currently the only non-standard options are
\medskip
\item{--} $D_8 = 10$: the file contains 6 variables $A_1 - A_6$,
\item{--} $D_8 = 100$: the file contains 8 variables $A_1 - A_8$,
\medskip\noindent
as defined above.
The value of $D_8$ is checked when the file is opened for read;
hence the structure must be the same for all models in a given file.

\end
